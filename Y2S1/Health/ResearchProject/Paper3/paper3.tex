\documentclass[11pt,a4paper]{article}

% Essential packages
\usepackage[utf8]{inputenc}
\usepackage[T1]{fontenc}
\usepackage[margin=1in]{geometry}
\usepackage{amsmath,amsfonts,amssymb}
\usepackage{graphicx}
\usepackage{booktabs}
\usepackage{url}
\usepackage{hyperref}
\usepackage{natbib}
\usepackage{setspace}


% Additional useful packages
\usepackage{fancyhdr}
\usepackage{subcaption}
\usepackage{tikz}
\usepackage{algorithm}
\usepackage{algorithmic}
\usepackage{listings}
\usepackage{xcolor}

% Configure hyperref
\hypersetup{
    colorlinks=true,
    linkcolor=blue,
    filecolor=magenta,      
    urlcolor=cyan,
    citecolor=red,
    pdftitle={Drug_Search},
    pdfauthor={Tate Mason},
}

% Set line spacing
\onehalfspacing

% Configure headers and footers
\pagestyle{fancy}
\fancyhf{}
\rhead{\thepage}
\lhead{\textit{Drug Search and Physician Hazard}}
\renewcommand{\headrulewidth}{0.4pt}

% Title page information
\title{Drug Search and Physician Hazard: \\
An Investigation into Addict Behavior and Policy Remedies}

\author{
    Tate Mason\thanks{Tate.Mason@uga.edu} \\
    \textit{University of Georgia} \\
    \textit{Athens, Georgia} \\
}

\date{\today}

% Abstract environment
\newenvironment{abstract}%
{\cleardoublepage\null \vfill\begin{center}%
\bfseries \abstractname \end{center}}%
{\vfill\null}

\begin{document}

% Title page
\maketitle
\thispagestyle{empty}

% Abstract
\begin{abstract}
\noindent % Write a concise summary (150-250 words) covering: research problem, methodology, key findings, and significance. State your main contributions clearly.

\vspace{0.3cm}
\noindent \textbf{Keywords:} % List 3-6 keywords relevant to your research
\end{abstract}

\newpage
\setcounter{page}{1}

% Main content
\section{Introduction}
\label{sec:introduction}

% Provide a broad introduction to your field and gradually narrow to your specific research question.
% Motivate why this work is necessary and important.

\subsection{Background and Motivation}
\label{subsec:background}

Over the last few decades, prescription drug abuse has become a significant and growing public health concern across the globe. The misuse of prescription drugs, specifially opioids, benzodiazepines, and stimulants, has led to the question of how to mitigate the practice of
physician search. Physician search refers to the practice of patients seeking multiple doctors in the hope of "scoring" a presription to continue their addiction. While there are preventative measures in place, like registries of offenders, the practice persists. Coninciding
with this question, it would be of great use to ascertain the incentive for physicians to enable the misuse of drugs, gaining a repeat source of revenue.

This paper seeks to understand the interplay of addicts and physician search, in hope of policy remedies which are more effective than a list. Further, the analysis of prescriber responsibility is also of great importance. Being able to understand how morals and fiscal
incentives contrast in this case could help to implement policy which negates the opportunity for prescribers to mis-prescribe a sensitive drug. Finally, this paper will look to gain an insight into potential rehabilitation remedies which may help take the onus off both
addicts and physicians.



% Provide context for your research area. Explain why this problem is important and worth investigating.

\subsection{Problem Statement}
\label{subsec:problem}

To ascertain the interplay between physician and addict behavior, it would be of interest to model physician WTP, in a sense, with their reputation. That is, how much risk are physicians willing to onboard
for the profits associated with increased and repeat prescriptions? This would be interesting for many reasons, since physicians have many reasons to uphold their standing from liability, licensure, and even
word of mouth advertisement. The last one may be increased as addicts see a pathway to drug access, but that would also increase liabilty and risk of loss of their licensure.

For addicts, it is interesting to inquire about their search cost allocation, as well as habit based expenditure on health. Tolerance necessarily grows with prolonged use of drugs, thus modeling this sort
of depletion of both search and health budgets is of interest to understand the impact on patients of finding a willing prescriber.prescriber

From these two avenues, I think there is a worthwhile question to answer. That is, what is the risk threshold physicians will take on in exchange for profit as well as understanding time and health costs applied
to addicts once they find a willing prescriber.

% Clearly articulate the specific problem or research question you're addressing. Be precise and focused.

\subsection{Contributions}
\label{subsec:contributions}

% List your main contributions clearly and concisely:
% - What novel methods, insights, or results do you provide?
% - How do they advance the state of the art?
% - What specific problems do they solve?

\subsection{Paper Organization}
\label{subsec:organization}

% Briefly outline how the paper is structured. Guide readers through your narrative flow.

\section{Related Work}
\label{sec:related_work}

% Comprehensively review relevant prior work. Organize thematically, not chronologically.
% Show how your work builds on, differs from, or improves upon existing approaches.

\subsection{Previous Approaches}
\label{subsec:previous_approaches}

% Survey existing methods and approaches relevant to your problem.
% Group similar approaches and explain their key ideas, strengths, and weaknesses.

\subsection{Limitations of Existing Work}
\label{subsec:limitations}

% Identify specific gaps, limitations, or open problems in existing approaches.
% This motivates why your work is needed and positions your contributions.

\section{Methodology}
\label{sec:methodology}

\subsection{Theoretical Framework}
\label{subsec:framework}

\subsubsection*{Agents and Functions}
\label{subsubsec:agents&functions}
There will be two classes of agents in this model, physicians and addicts. 
\begin{gather*}
  \text{addicts} \in \mathcal{A} = \{a_1, a_2, \ldots, a_n\} \\
  \text{physicians} \in \mathcal{M} = \{m_1, m_2, \ldots, m_m\}
\end{gather*}

The objective function of the addict is given by:

\begin{gather*}
  \max_p \sum\limits_{t=1}^{T} \delta^t_i p_id_i_{\mathbb{1}(d=1)}\gamma_i^t \\
  \text{s.t.} \\
  p_id_i\gamma_i^t = s_i \\
  s_i = x_ivj + t + r_i
  x_i = \gamma_i^t\cdot w_i
\end{gather*}

Where $\delta_i^t$ is discount factor of addicts, $\gamma_i^T$ is a tolerance parameter, indicating how deeply addicted the agent is, $p_i$ is utility from prescriptions, $d_i$ is an indicator for if doctors precscribe or not, 
$s_i$ is a search cost comprised of $x_i$, a willingness to pay parameter, $v_j$, a visit cost for each doctor, $t$, a fixed time cost, and $r_i$ a risk cost of being reported for the behavior. $x_i$ is comprised
of one's tolerance scaling their wealth to determine how much they're willing to pay for a chance at drugs.

The objective function of the prescriber is:

\begin{gather*}
  \max_x x_j\cdot v_i \\
  \text{s.t.} \\
  x_j\cdot v_i = i_i \\
  v_i = p_jx_j - x_{-j} - r_i \\
\end{gather*}

Where $i_i$ is the cost of insurance for a physician. All other variables are defined above. 

\subsubsection*{Comparative Statics and Counterfactuals}
\label{subsubsec:comparative&counterfactuals}

Understanding how willingness to pay and price of a visit affect the equilibrium would be of importance, as these would have a direct effect on the equilibrium matching.
For a counterfactual, I am interested in modeling some sort of rehabilitation "punishment" for being caught searching. This would be of policy relevance due to the ongoing national drug epidemic.
Further, I think another comparative static of interest would be the risk probability changing, and responses of both addicts and prescribers.

\subsubsection*{Equilibrium and Demand/Supply Behavior}
\label{subsubsec:eq&behavior}

Addicts will have monotonic demand, demanding more and more until death due to tolerance parameter $\gamma_i^t$. Physicians will demand new patients until $r_i = p_jx_j$, or until they are not willing to take on any
more risk in exchange for monetary gain. Equilibrium is achieved when all doctors are matched to either an addict or nothing. Preference ordering of doctors will be dependent on risk tolerance and WTP of addicts.
For addicts, preferences are determined by willingness to prescribe and proximity. To accomplish these facets, I will need to specify a matching algorithm and a search function for addicts.

\subsubsection*{To-Do}
\label{subsubsec:todo}

Still under consideration are how to model things like the risk probability, utility from prescriptions, and if the penalty of being caught as an addict removes them from the mechanism or if it
just makes search cost increase.

% Describe your approach in sufficient detail for reproducibility.
% Include mathematical formulations, algorithms, and theoretical analysis as needed.

\subsection{Problem Formulation}
\label{subsec:formulation}

% Formally define the problem you're solving using mathematical notation.
% Establish assumptions, constraints, and objectives clearly.

\subsection{Proposed Approach}
\label{subsec:approach}

% Detail your methodology including:
% - Core algorithms and their rationale
% - Mathematical derivations and proofs
% - Implementation details crucial for reproducibility
% - Complexity analysis if relevant

\section{Experimental Evaluation}
\label{sec:experiments}

% Present your experimental design, results, and analysis.
% Be objective and thorough in reporting both positive and negative results.

\subsection{Experimental Setup}
\label{subsec:setup}


\subsection{Datasets}
\label{subsec:datasets}

\subsection{Results}
\label{subsec:results}

% Present results clearly and objectively:
% - Use tables and figures effectively
% - Report statistical significance where appropriate
% - Include error bars or confidence intervals
% - Compare against all relevant baselines
% - Report both quantitative metrics and qualitative observations

\subsection{Analysis}
\label{subsec:analysis}

% Provide deeper analysis of your results:
% - Statistical significance tests
% - Ablation studies showing contribution of different components
% - Error analysis and failure cases
% - Computational complexity and runtime analysis
% - Sensitivity analysis for key parameters

\section{Discussion}
\label{sec:discussion}

% Interpret your findings, discuss broader implications, and acknowledge limitations honestly.

\subsection{Interpretation of Results}
\label{subsec:interpretation}

% Explain what your results mean in the context of your research questions:
% - How do findings relate to your hypotheses?
% - What insights do the results provide?
% - How do they compare to previous work?
% - What are the practical implications?

\subsection{Limitations}
\label{subsec:limitations}

% Honestly discuss limitations of your work:
% - Methodological limitations
% - Dataset limitations or biases
% - Assumptions that may not hold in practice
% - Scope limitations
% - Computational or resource constraints

\subsection{Future Work}
\label{subsec:future_work}

\end{document}
