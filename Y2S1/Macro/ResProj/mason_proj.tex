\documentclass[11pt,a4paper]{article}

% Essential packages
\usepackage[utf8]{inputenc}
\usepackage[T1]{fontenc}
\usepackage[margin=1in]{geometry}
\usepackage{amsmath,amsfonts,amssymb}
\usepackage{graphicx}
\usepackage{booktabs}
\usepackage{url}
\usepackage{hyperref}
\usepackage{natbib}
\usepackage{setspace}


% Additional useful packages
\usepackage{fancyhdr}
\usepackage{subcaption}
\usepackage{tikz}
\usepackage{algorithm}
\usepackage{algorithmic}
\usepackage{listings}
\usepackage{xcolor}

% Configure hyperref
\hypersetup{
    colorlinks=true,
    linkcolor=blue,
    filecolor=magenta,      
    urlcolor=cyan,
    citecolor=red,
    pdftitle={Your Paper Title},
    pdfauthor={Your Name},
}

% Set line spacing
\onehalfspacing

% Configure headers and footers
\pagestyle{fancy}
\fancyhf{}
\rhead{\thepage}
\lhead{\textit{Location and Career Search}}
\renewcommand{\headrulewidth}{0.4pt}

% Title page information
\title{Your Academic Paper Title: \\
A Comprehensive Investigation}

\author{
    Tate Mason\thanks{Tate.Mason@uga.edu} \\
    \textit{John Munro Godfrey Sr. Department of Economics} \\
    \textit{The University of Georgia} \\
    \textit{Athens, Georgia} \\
    \\
}

\date{\today}

% Abstract environment
\newenvironment{abstract}%
{\cleardoublepage\null \vfill\begin{center}%
\bfseries \abstractname \end{center}}%
{\vfill\null}

\begin{document}

% Title page
\maketitle
\thispagestyle{empty}

% Abstract
\begin{abstract}
\noindent % Write a concise summary (150-250 words) covering: research problem, methodology, key findings, and significance. State your main contributions clearly.

I would like to do research into location search as a facet of optimal career search for young adults, modeling how the search cost to find an optimal landing spot at 
what could be thought of as the building phase of life. More simply, I want to model where young adults migrate to and from, utilizing IPUMS or PSID data, to analyze how
much investment in moving cost and how much risk they are willing to take on to find the location which provides the most optimal situation for lifetime earnings and employment
outcomes. Particularly, I am most interested in migration to major metropolitan areas, and want to understand how much cost through either monetary cost of uprooting or the cost
of uncertainty agents will invest to achieve their most desired outcome. This could be modeled with uncertainty surrounding employment, as is common, as well as uncertainty around
future migration as in some of the work from James Kennan. Ideally, this work will analyze the trade off between early investment into migration versus end of life outcomes around
wealth and earnings. The two primary questions are, first, how many moves are needed, on average, to find one's best place and, secondly, what are end of life returns to varying levels 
of investment in moving?

\vspace{0.3cm}
\noindent \textbf{Keywords:} % List 3-6 keywords relevant to your research
Location search, career search, earnings outcomes, career outcomes.
\end{abstract}

\newpage
\setcounter{page}{1}

% Main content
\section{Introduction}
\label{sec:introduction}

% Provide a broad introduction to your field and gradually narrow to your specific research question.
% Motivate why this work is necessary and important.

\subsection{Background and Motivation}
\label{subsec:background}

% Provide context for your research area. Explain why this problem is important and worth investigating.

\subsection{Problem Statement}
\label{subsec:problem}

% Clearly articulate the specific problem or research question you're addressing. Be precise and focused.

\subsection{Contributions}
\label{subsec:contributions}

% List your main contributions clearly and concisely:
% - What novel methods, insights, or results do you provide?
% - How do they advance the state of the art?
% - What specific problems do they solve?

\subsection{Paper Organization}
\label{subsec:organization}

% Briefly outline how the paper is structured. Guide readers through your narrative flow.

\section{Related Work}
\label{sec:related_work}

% Comprehensively review relevant prior work. Organize thematically, not chronologically.
% Show how your work builds on, differs from, or improves upon existing approaches.

\subsection{Previous Approaches}
\label{subsec:previous_approaches}

% Survey existing methods and approaches relevant to your problem.
% Group similar approaches and explain their key ideas, strengths, and weaknesses.

\subsection{Limitations of Existing Work}
\label{subsec:limitations}

% Identify specific gaps, limitations, or open problems in existing approaches.
% This motivates why your work is needed and positions your contributions.

\section{Methodology}
\label{sec:methodology}

% Describe your approach in sufficient detail for reproducibility.
% Include mathematical formulations, algorithms, and theoretical analysis as needed.

\subsection{Problem Formulation}
\label{subsec:formulation}

% Formally define the problem you're solving using mathematical notation.
% Establish assumptions, constraints, and objectives clearly.

\subsection{Proposed Approach}
\label{subsec:approach}

% Detail your methodology including:
% - Core algorithms and their rationale
% - Mathematical derivations and proofs
% - Implementation details crucial for reproducibility
% - Complexity analysis if relevant

\section{Experimental Evaluation}
\label{sec:experiments}

% Present your experimental design, results, and analysis.
% Be objective and thorough in reporting both positive and negative results.

\subsection{Experimental Setup}
\label{subsec:setup}

Describe your experimental methodology, including:
\begin{itemize}
    \item Datasets used
    \item Evaluation metrics
    \item Baseline methods
    \item Implementation details
    \item Hardware/software specifications
\end{itemize}

\subsection{Datasets}
\label{subsec:datasets}

Provide details about the datasets used in your evaluation.

\begin{table}[htbp]
\centering
\caption{Dataset Statistics}
\label{tab:datasets}
\begin{tabular}{@{}lrrr@{}}
\toprule
Dataset & Training Samples & Test Samples & Features \\
\midrule
Dataset 1 & 10,000 & 2,500 & 784 \\
Dataset 2 & 50,000 & 10,000 & 3,072 \\
Dataset 3 & 1,000,000 & 100,000 & 128 \\
\bottomrule
\end{tabular}
\end{table}

\subsection{Results}
\label{subsec:results}

% Present results clearly and objectively:
% - Use tables and figures effectively
% - Report statistical significance where appropriate
% - Include error bars or confidence intervals
% - Compare against all relevant baselines
% - Report both quantitative metrics and qualitative observations

\subsection{Analysis}
\label{subsec:analysis}

% Provide deeper analysis of your results:
% - Statistical significance tests
% - Ablation studies showing contribution of different components
% - Error analysis and failure cases
% - Computational complexity and runtime analysis
% - Sensitivity analysis for key parameters

\section{Discussion}
\label{sec:discussion}

% Interpret your findings, discuss broader implications, and acknowledge limitations honestly.

\subsection{Interpretation of Results}
\label{subsec:interpretation}

% Explain what your results mean in the context of your research questions:
% - How do findings relate to your hypotheses?
% - What insights do the results provide?
% - How do they compare to previous work?
% - What are the practical implications?

\subsection{Limitations}
\label{subsec:limitations}

% Honestly discuss limitations of your work:
% - Methodological limitations
% - Dataset limitations or biases
% - Assumptions that may not hold in practice
% - Scope limitations
% - Computational or resource constraints

\subsection{Future Work}
\label{subsec:future_work}

% Suggest concrete directions for future research:
% - How could your approach be extended or improved?
% - What new research questions arise from your work?
% - How could limitations be addressed?
% - What are the next logical steps?

\section{Conclusion}
\label{sec:conclusion}

% Summarize your main contributions, key findings, and their significance.
% Avoid simply repeating the abstract - provide a synthesis that reinforces your contributions.

In this paper, we presented [brief summary of contribution]. Our experimental evaluation demonstrates [key findings]. The implications of this work include [broader impact]. Future research directions include [future work].

\section*{Acknowledgments}

% Thank people who contributed to the work but aren't co-authors.
% Acknowledge funding sources, institutional support, or resources used.

% Bibliography
\bibliographystyle{plainnat}
\bibliography{references} % Make sure you have a references.bib file

% Appendices (if needed)
\appendix
\section{Additional Experimental Results}
\label{app:additional_results}

% Include supplementary results that support your main findings but are too detailed for the main text:
% - Extended result tables
% - Additional ablation studies
% - Results on additional datasets
% - Detailed parameter sensitivity analysis

\section{Mathematical Proofs}
\label{app:proofs}

% Include detailed mathematical proofs and derivations:
% - Theoretical analysis too lengthy for the main text
% - Convergence proofs
% - Complexity analysis details
% - Complete derivations of key equations

\end{document}
