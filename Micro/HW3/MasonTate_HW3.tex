%%% Homework 3 - Advanced Micro %%%
\documentclass[10pt, a4paper]{article}
\usepackage[top=3cm, bottom=4cm, left=3.5cm, right=3.5cm]{geometry}
\usepackage{amsmath,amsthm,amsfonts,amssymb,amscd, fancyhdr, color, comment, graphicx, environ}
\usepackage{float}
\usepackage{mathtools}
\usepackage{mathrsfs}
\usepackage[math-style=ISO]{unicode-math}
\DeclareSymbolFont{\mathnormal}{letters}
\usepackage{lastpage}

%%%%%%%%%%%%%%%%%%%%%%%%%%%%%%%%%%%%%%%%%%%%%%%%%%%%%%%%%%%%%%%%%%
%%%%%%%%%%%%%%%%%%%%%%%%%%%%%%%%%%%%%%%%%%%%%%%%%%%%%%%%%%%%%%%%%%
%Fill in the appropriate information below
\newcommand{\norm}[1]{\left\lVert#1\right\rVert}     
\newcommand\course{ECON - 8010}                            % <-- course name   
\newcommand\hwnumber{ 3}                                 % <-- homework number
\newcommand\Information{Tate Mason}                        % <-- personal information
%%%%%%%%%%%%%%%%%%%%%%%%%%%%%%%%%%%%%%%%%%%%%%%%%%%%%%%%%%%%%%%%%%
%%%%%%%%%%%%%%%%%%%%%%%%%%%%%%%%%%%%%%%%%%%%%%%%%%%%%%%%%%%%%%%%%%
%Page setup
\pagestyle{fancy}
\headheight 35pt
\lhead{\today}
\rhead{}
\lfoot{}
\pagenumbering{arabic}
\cfoot{\small\thepage}
\rfoot{}
\headsep 1.2em
\renewcommand{\baselinestretch}{1.25}
%%%%%%%%%%%%%%%%%%%%%%%%%%%%%%%%%%%%%%%%%%%%%%%%%%%%%%%%%%%%%%%%%%
%%%%%%%%%%%%%%%%%%%%%%%%%%%%%%%%%%%%%%%%%%%%%%%%%%%%%%%%%%%%%%%%%%
%Add new commands here
\renewcommand{\labelenumi}{\alph{enumi})}
\newcommand{\L}{\mathcal L}
\newcommand{\Z}{\mathbb Z}
\newcommand{\R}{\mathbb R}
\newcommand{\Q}{\mathbb Q}
\newcommand{\NN}{\mathbb N}
\newcommand{\PP}{\mathbb P}
\DeclareMathOperator{\Mod}{Mod} 
\renewcommand\lstlistingname{Algorithm}
\renewcommand\lstlistlistingname{Algorithms}
\def\lstlistingautorefname{Alg.}
\newtheorem*{theorem}{Theorem}
\newtheorem*{lemma}{Lemma}
\newtheorem{case}{Case}
\newcommand{\assign}{:=}
\newcommand{\infixiff}{\text{ iff }}
\newcommand{\nobracket}{}
\newcommand{\backassign}{=:}
\newcommand{\tmmathbf}[1]{\ensuremath{\boldsymbol{#1}}}
\newcommand{\tmop}[1]{\ensuremath{\operatorname{#1}}}
\newcommand{\tmtextbf}[1]{\text{{\bfseries{#1}}}}
\newcommand{\tmtextit}[1]{\text{{\itshape{#1}}}}

\newenvironment{itemizedot}{\begin{itemize} \renewcommand{\labelitemi}{$\bullet$}\renewcommand{\labelitemii}{$\bullet$}\renewcommand{\labelitemiii}{$\bullet$}\renewcommand{\labelitemiv}{$\bullet$}}{\end{itemize}}
\catcode`\<=\active \def<{
\fontencoding{T1}\selectfont\symbol{60}\fontencoding{\encodingdefault}}
\catcode`\>=\active \def>{
\fontencoding{T1}\selectfont\symbol{62}\fontencoding{\encodingdefault}}
\catcode`\<=\active \def<{
\fontencoding{T1}\selectfont\symbol{60}\fontencoding{\encodingdefault}}

%%%%%%%%%%%%%%%%%%%%%%%%%%%%%%%%%%%%%%%%%%%%%%%%%%%%%%%%%%%%%%%%%%
%%%%%%%%%%%%%%%%%%%%%%%%%%%%%%%%%%%%%%%%%%%%%%%%%%%%%%%%%%%%%%%%%%
%Begin now!

\begin{document}
  \begin{titlepage}
    \begin{center}
      \vspace*{3cm}
            
        \vspace{1cm}
        \huge
        Homework \hwnumber
            
        \vspace{1.5cm}
        \Large
            
        \textbf{\Information}                      % <-- author
            
        \vfill
        
        An \course \ Homework Assignment
            
        \vspace{1cm}
        \Large

        
        \today
            
    \end{center}
  \end{titlepage}

  \newpage

  \section{Question 3.I.5}
    \subsection{Problem}
      Show that if $u(x)$ is quasilinear with respect to the first good ($p_1$ fixed at 1), then $CV(p^0,p^1,w)=EV(p^0,p^1,w)$ for any $(p^0,p^1,w)$.
    \subsection{Solution}
  \section{Question 5.C.9}
    \subsection{Problem}
      Derive the profit function $\pi(p)$ and supply function $y(p)$ for the single output technologies whose production functions $f(z)$ are given by:

      (b) $f(z)=\sqrt{\min\{z_1,z_2\}}$

      (c) $f(z)=(z_1^{\rho}z_2^{\rho})^{\frac{1}{\rho}}$ for $\rho\leq1$ 
    \subsection{Solution}
      (c) 
      \begin{center}
        ${\max \atop{z_1,z_2}} p((z_1z_2)^{\rho})^{\frac{1}{\rho}}-w_1z_1-w_2z_2$ \\ 
      \end{center}
      F.O.C's
      \begin{center}
        $p(\frac{1}{\rho})(z_1^{\rho}z_2^{\rho})^{\frac{1}{\rho}-1}\cdot(\rho z_1^{\rho-1}z_2^{\rho})-w_1 \Rightarrow$ \\
        $w_1 = p(\frac{1}{\rho})(z_1^{\rho}z_2^{\rho})^{\frac{1}{\rho}-1}\cdot(\rho z_1^{\rho-1}z_2^{\rho})$ \\
        $w_2 = p(\frac{1}{\rho})(z_1^{\rho}z_2^{\rho})^{\frac{1}{\rho}-1}\cdot(\rho z_1^{\rho}z_2^{\rho-1})$ \\ 
        $\frac{w_1}{w_2}=\frac{z_1^{-1}}{z_2^{-1}}$ \\
        $\frac{w_1}{w_2}=\frac{z_2}{z_1}$ \\
        $z_2 = \frac{w_1z_1}{w_2}$ \\
        

  \section{Question 5.C.10}
    \subsection{Problem}
      
      Derive the cost function $c(w,q)$ and conditional function demand functions (or correspondences) $z(w,q)$ for each of the following single-output constant return technologies with production functions:

      (b) $f(z)=\min\{z_1,z_2\}$ (Leontief technology)

      (c) $f(z)=(z_1^{\rho}z_2^{\rho})^{\frac{1}{\rho}}$ for $\rho\leq1$ (CES technology) 
  \section{Question 5.C.11}
    \subsection{Problem}
      Show that $\frac{\partial z_l(w,q)}{\partial q}>0$ if and only if marginal cost at q is increasing in $w_l$.
  \section{Question 5}
    \subsection{Problem}
      A firm uses 2 inputs, $z_1$ and $z_2$, which it purchases at prices $w_1$ and $w_2$ to produce a single output. The firm's technology is described by production function $f$ which is strictly increasing and obeys the Inada conditions ${\lim\atop{z_1\righarrow0}}\frac{\partial f(z_1,z_2)}{\partial z_1}={\lim\atop{z_1\rightarrow0}}\frac{\partial f(z_1, z_2)}{\partial z_2}=\infty$ for each $x$. (Hence, the firm will always choose to use a strictly positive quantity of each input.)

      (a) Set up firm's cost minimization problem, write down its Lagrangian, find firm's first order conditions for cost minimization.

      (b) Use the envelope theorm to five an expression (possibly involving a Lagrange multiplier) for the firm's marginal cost $\frac{\partial c(w,q)}{\partial q}$.

      (c) An economist wishes to measure the firm's markup-ratio of price of output, p, to its marginal cost $\frac{\partial c(w,q)}{\partial q}$. However, she does not know what kind of competition the firm faces inthe production market. In fact, the only data she has are:
        
        - the marginal product of input 1 at the input fix selected by the firm:
        \begin{center}
          $\frac{\partial f(z(w,q))}{\partial z_1}$
        \end{center}

        - the price of input 1, $w_1$

        - the price of firm's output $p$.

      How can she use these data to recover the firm's markup?
  \section{Question 6.B.2}
    \subsection{Problem}
      Show that if the preference relation $\succeq$ on $\mathcal{L}$ is represented by a utility function $U(\cdot)$ that has the expected utility form, then $\succeq$ satisfies the independence axiom. 

\end{document}
