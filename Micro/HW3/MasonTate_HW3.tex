%%% Homework 3 - Advanced Micro %%%
\documentclass[10pt, a4paper]{article}
\usepackage[top=3cm, bottom=4cm, left=3.5cm, right=3.5cm]{geometry}
\usepackage{amsmath,amsthm,amsfonts,amssymb,amscd, fancyhdr, color, comment, graphicx, environ}
\usepackage{float}
\usepackage{mathtools}
\usepackage{mathrsfs}
\usepackage[math-style=ISO]{unicode-math}
\DeclareSymbolFont{\mathnormal}{letters}
\usepackage{lastpage}

%%%%%%%%%%%%%%%%%%%%%%%%%%%%%%%%%%%%%%%%%%%%%%%%%%%%%%%%%%%%%%%%%%
%%%%%%%%%%%%%%%%%%%%%%%%%%%%%%%%%%%%%%%%%%%%%%%%%%%%%%%%%%%%%%%%%%
%Fill in the appropriate information below
\newcommand{\norm}[1]{\left\lVert#1\right\rVert}     
\newcommand\course{ECON - 8010}                            % <-- course name   
\newcommand\hwnumber{ 3}                                 % <-- homework number
\newcommand\Information{Tate Mason}                        % <-- personal information
%%%%%%%%%%%%%%%%%%%%%%%%%%%%%%%%%%%%%%%%%%%%%%%%%%%%%%%%%%%%%%%%%%
%%%%%%%%%%%%%%%%%%%%%%%%%%%%%%%%%%%%%%%%%%%%%%%%%%%%%%%%%%%%%%%%%%
%Page setup
\pagestyle{fancy}
\headheight 35pt
\lhead{\today}
\rhead{}
\lfoot{}
\pagenumbering{arabic}
\cfoot{\small\thepage}
\rfoot{}
\headsep 1.2em
\renewcommand{\baselinestretch}{1.25}
%%%%%%%%%%%%%%%%%%%%%%%%%%%%%%%%%%%%%%%%%%%%%%%%%%%%%%%%%%%%%%%%%%
%%%%%%%%%%%%%%%%%%%%%%%%%%%%%%%%%%%%%%%%%%%%%%%%%%%%%%%%%%%%%%%%%%
%Add new commands here
\renewcommand{\labelenumi}{\alph{enumi})}
\newcommand{\Z}{\mathbb Z}
\newcommand{\R}{\mathbb R}
\newcommand{\Q}{\mathbb Q}
\newcommand{\NN}{\mathbb N}
\newcommand{\PP}{\mathbb P}
\DeclareMathOperator{\Mod}{Mod} 
\renewcommand\lstlistingname{Algorithm}
\renewcommand\lstlistlistingname{Algorithms}
\def\lstlistingautorefname{Alg.}
\newtheorem*{theorem}{Theorem}
\newtheorem*{lemma}{Lemma}
\newtheorem{case}{Case}
\newcommand{\assign}{:=}
\newcommand{\infixiff}{\text{ iff }}
\newcommand{\nobracket}{}
\newcommand{\backassign}{=:}
\newcommand{\tmmathbf}[1]{\ensuremath{\boldsymbol{#1}}}
\newcommand{\tmop}[1]{\ensuremath{\operatorname{#1}}}
\newcommand{\tmtextbf}[1]{\text{{\bfseries{#1}}}}
\newcommand{\tmtextit}[1]{\text{{\itshape{#1}}}}

\newenvironment{itemizedot}{\begin{itemize} \renewcommand{\labelitemi}{$\bullet$}\renewcommand{\labelitemii}{$\bullet$}\renewcommand{\labelitemiii}{$\bullet$}\renewcommand{\labelitemiv}{$\bullet$}}{\end{itemize}}
\catcode`\<=\active \def<{
\fontencoding{T1}\selectfont\symbol{60}\fontencoding{\encodingdefault}}
\catcode`\>=\active \def>{
\fontencoding{T1}\selectfont\symbol{62}\fontencoding{\encodingdefault}}
\catcode`\<=\active \def<{
\fontencoding{T1}\selectfont\symbol{60}\fontencoding{\encodingdefault}}

%%%%%%%%%%%%%%%%%%%%%%%%%%%%%%%%%%%%%%%%%%%%%%%%%%%%%%%%%%%%%%%%%%
%%%%%%%%%%%%%%%%%%%%%%%%%%%%%%%%%%%%%%%%%%%%%%%%%%%%%%%%%%%%%%%%%%
%Begin now!

\begin{document}
  \begin{titlepage}
    \begin{center}
      \vspace*{3cm}
            
        \vspace{1cm}
        \huge
        Homework \hwnumber
            
        \vspace{1.5cm}
        \Large
            
        \textbf{\Information}                      % <-- author
            
        \vfill
        
        An \course \ Homework Assignment
            
        \vspace{1cm}
        \Large

        
        \today
            
    \end{center}
  \end{titlepage}

  \newpage

  \section{Question 3.I.5}
    \subsection*{Problem}
      Show that if $u(x)$ is quasilinear with respect to the first good ($p_1$ fixed at 1), then $CV(p^0,p^1,w)=EV(p^0,p^1,w)$ for any $(p^0,p^1,w)$.
    \subsection*{Solution}
    \begin{proof}
      Because we know that initial price of good 1 is fixed, we know that Hicksian demand is independent of wealth. Therefore, there is no minimum utility level, $\bar u$ which needs to be reached. Because of this, we can write $x(p,w)=x(p)=h(p)=h(p,w)$. Further, $e(p,u) = p\cdot h(p)$ in this case. So, we see that $CV = h(p^1)$ and $EV = h(p^0)$. Since prices are fixed, we can say that $CV(p^0,p^1,w)=EV(p^0,p^1,w)$ for any $(p^0,p^1,w)$. 
    \end{proof}
  \section{Question 5.C.9}
    \subsection*{Problem}
      Derive the profit function $\pi(p)$ and supply function $y(p)$ for the single output technologies whose production functions $f(z)$ are given by

      (b) $f(z)=\sqrt{\min\{z_1,z_2\}}$

      (c) $f(z)=(z_1^{\rho}z_2^{\rho})^{\frac{1}{\rho}}$ for $\rho\leq1$ \\

    \subsection*{Solution}
      (b) $y = \sqrt{\min z_1,z_2}$. In this case, $z_1=z_2$ so,

      \begin{center}
        $y = \sqrt{z_1}$ \\
        $\pi(p) = p\sqrt{z_1}-z_1(w_1+w_2)$ \\
        $z_1: \frac{1}{2} p z_1^{-\frac{1}{2}}-w_1-w_2=0$ \\
        $\frac{1}{2}pz_1^{-\frac{1}{2}}=w_1+w_2$ \\
        $pz_1^{-\frac{1}{2}}=2(w_1+w_2)$ \\
        $z_1 = (\frac{p}{2(w_1+w_2)})^2$ \\
        $\pi(p) = \frac{p^2}{2(w_1+w_2)}-(\frac{p}{2(w_1+w_2)})^2(w_1+w_2)$ \\
        $\pi(p)=\frac{p^2}{2(w_1+w_2)}-\frac{p^2}{4(w_1+w_2)}$ \\
        $\Aboxed{\pi(p) = \frac{p^2}{4(w_1+w_2)}}$
      \end{center}

      (c) $y = (z_1^{\rho}+z_2^{\rho})^{\frac{1}{\rho}}$ for $\rho\leq1$
      
      \begin{center}
        $\pi(p)=p(z_1^{\rho}+z_2^{\rho})^{\frac{1}{\rho}}-w_1z_1-w_2z_2$ \\
      \end{center}
      Finding FOC's
      \begin{center}
        $z_1: p\cdot\frac{1}{\rho}(z_1^{\rho}+z_2^{\rho})^{\frac{1}{\rho-1}}\cdot\rho z_1^{\rho-1} - w_1 = 0$ \\
        $z_2: p\cdot\frac{1}{\rho}(z_1^{\rho}+z_2^{\rho})^{\frac{1}{\rho-1}}\cdot\rho z_2^{\rho-1} - w_2 = 0$ \\
      \end{center}
      Isolating $z_1$:
      \begin{center}
        $\frac{w_1}{w_2}=(\frac{z_1}{z_2})^{\rho-1}$ \\
        $z_1 = z_2(\frac{w_1}{w_2})^{\frac{1}{\rho-1}}$
      \end{center}
      Put back into production function
      \begin{center}
        $y = z_2^{\rho}((\frac{w_1}{w_2})^{\frac{\rho}{\rho-1}}+1)^{\frac{1}{\rho}}$ 
      \end{center}
      Solving for $z_2$
      \begin{center}
        $z_2 = y(\frac{w_2^{\frac{\rho}{\rho-1}}}{(w_1+w_2)^{\frac{\rho}{\rho-1}}})^{\frac{1}{\rho}}$ \\
        $C = y((w_1+w_2)^{\frac{\rho}{\rho-1}})^{\frac{\rho-1}{\rho}}$ \\
        $\pi = py-y((w_1+w_2)^{\frac{\rho}{\rho-1}})^{\frac{\rho-1}{\rho}}$ \\
      \end{center}
      Got lost around solving for $z_2$. Going to speak to Dr. Yoder at office hours. Think I made a calculus mistake.


\section{Question 5.C.10}
    \subsection*{Problem}
      Derive the cost function $c(w,q)$ and conditional function demand functions (or correspondences) $z(w,q)$ for each of the following single-output constant return technologies with production functions:

      (b) $f(z)=\min\{z_1,z_2\}$ (Leontief technology)
        
      (c) $f(z)=(z_1^{\rho}+z_2^{\rho})^{\frac{1}{\rho}}$ for $\rho\leq1$ (CES technology) 
    \subsection*{Solution}
      (b) 
      \begin{center}
        ${\min\atop{\vec{z}\geq0}}\vec{w}\cdot\vec{z}$
      \end{center}
      s.t.
      \begin{center}
        $\min\{z_1,z_2\}\leq q$ \\
        $\vec{z}\geq0$ \\
      \end{center}

      \begin{center}
        $z_1=z_2=q$ \\
        $\Aboxed{C(w,q)=q(w_1+w_2)}$ \\
        $\Aboxed{z_1(w,q) = w_1q}$ \\
        $\Aboxed{z_2(w,q) = w_2q}$ \\
      \end{center}
      (c)
      \begin{center}
        ${\min\atop{\vec{z}\geq0}}\vec{w}\cdot\vec{z}$ \\
      \end{center}
      s.t.
      \begin{center}
        $(z_1^{\rho}+z_2^{\rho})^{\frac{1}{\rho}}\leq q$ \\
        $\vec{z}\geq 0$
      \end{center}

      \begin{center}
        $\vec{w}=\lambda\begin{bmatrix}p\frac{1}{\rho}(z_1^{\rho}+z_2^{\rho})^{\frac{1}{\rho}-1}\cdot(\rhoz_1^{\rho-1}) \\ p\frac{1}{\rho}(z_1^{\rho}+z_2^{\rho})^{\frac{1}{\rho}-1}\cdot(\rho z_2^{\rho-1})\end{bmatrix} + \vec{\mu}$
      \end{center}
      s.t.
      \begin{center}
        $\vec{\mu}\cdot\vec{z}=0$ \\
        $\lambda(q-f(\vec{z}))=0$ \\
      \end{center}
      Solving for $z_1$
      \begin{center}
        $\frac{w_1}{w_2}=(\frac{z_1}{z_2})^{\rho-1}\Rightarrow z_1 = (\frac{w_1}{w_2})^{\frac{1}{\rho-1}}z_2$ \\
      \end{center}
      Plug back into $f(\vec{z})$
      \begin{center}
        $((z_2(\frac{w_1}{w_2})^{\frac{1}{\rho-1}})^{\rho}+z_2^{\rho})^{\frac{1}{\rho}}=q$ \\
        $((z_2(\frac{w_1}{w_2})^{\frac{1}{\rho-1}})^{\rho}+z_2^{\rho} = q^{\rho}$ \\
        $(z_2((\frac{w_1}{w_2})^{\frac{1}{\rho-1}}+1))=q$ \\
        $z_2(\frac{w_1}{w_2})^{\frac{1}{\rho-1}}=q-1$ \\
        $\Aboxed{z_2(w,q) = (\frac{w_2}{w_1})^{\frac{1}{\rho-1}}q-1}$ \\
        $\Aboxed{z_1(w,q) = (\frac{w_1}{w_2})^{\frac{1}{\rho-1}}q-1}$ \\
        $\Aboxed{C(w,q) = q-1(w_1(\frac{w_1}{w_2})^{\frac{1}{\rho-1}})+w_2(\frac{w_2}{w_1})^{\frac{1}{\rho-1}}}$ \\ 
      \end{center}
  \section{Question 5.C.11}
    \subsection*{Problem}
      Show that $\frac{\partial z_l(w,q)}{\partial q}>0$ if and only if marginal cost at q is increasing in $w_l$.
    \subsection*{Solution}
      \begin{proof}
        This can be proven using Shephard's Lemma such that $z_l(w,q) = \frac{\partial C(w,q)}{\partial w_l}$. Marginal cost, then, is given as $\frac{\partial c(w,q)}{\partial q}$. So, we should show that $\frac{\partial z_l (w,q)}{\partial q}\Leftrightarrow \frac{\partial^2c(w,q)}{\partial q\partial w_l}$. Using Shephard's Lemma and symmetry of second derivatives, it can be shown that $\frac{\partial z_l(w,q)}{\partial q} = \frac{\partial^2 c(w,q)}{\partial w_l\partial q}=\frac{\partial^2 c(w,q)}{\patial q\partial w_l}$. Thus, $\frac{\partial z_l(w,q)}{\partial q}>0\Leftrightarrow \frac{\partial^2c(w,q)}{\partial q\partial w_l}>0$, showing that conditional factor demand for input $l$ increases with output if and only if the marginal cost is increasing in the price of input $l$.
      \end{proof}
  \section{Question 5}
    \subsection*{Problem}
      A firm uses 2 inputs, $z_1$ and $z_2$, which it purchases at prices $w_1$ and $w_2$ to produce a single output. The firm's technology is described by production function $f$ which is strictly increasing and obeys the Inada conditions ${\lim\atop{z_1\righarrow0}}\frac{\partial f(z_1,z_2)}{\partial z_1}={\lim\atop{z_1\rightarrow0}}\frac{\partial f(z_1, z_2)}{\partial z_2}=\infty$ for each $x$. (Hence, the firm will always choose to use a strictly positive quantity of each input.)

      (a) Set up firm's cost minimization problem, write down its Lagrangian, find firm's first order conditions for cost minimization.

      (b) Use the envelope theorem to find an expression (possibly involving a Lagrange multiplier) for the firm's marginal cost $\frac{\partial c(w,q)}{\partial q}$.

      (c) An economist wishes to measure the firm's markup-ratio of price of output, p, to its marginal cost $\frac{\partial c(w,q)}{\partial q}$. However, she does not know what kind of competition the firm faces inthe production market. In fact, the only data she has are:
        
        - the marginal product of input 1 at the input fix selected by the firm:
        \begin{center}
          $\frac{\partial f(z(w,q))}{\partial z_1}$
        \end{center}

        - the price of input 1, $w_1$

        - the price of firm's output $p$.

      How can she use these data to recover the firm's markup?
    \subsection*{Solution}
      (a)
      \begin{center}
        $\min_{z_1,z_2} w_1z_1+w_2z_2$
      \end{center}
      s.t.
      \begin{center}
        $f(z_1,z_2)\geq q$ \\
        $z_1,z_2\geq0$
      \end{center}
      Lagrangian:
      \begin{center}
        $\mathcal{L} = w_1z_1+w_2z_2 - \lambda(f(z_1,z_2)-q)$ \\
      \end{center}
      FOC's:
      \begin{center}
        $\frac{\partial\mathcal{L}}{\partial z_1}=w_1-\lambda\frac{\partial f}{\partial z_1}=0\Rightarrow w_1 = \lambda\frac{\partial f}{\partial z_1}$ \\
        $\frac{\partial\mathcal{L}}{\partial z_2}=w_2-\lambda\frac{\partial f}{\partial z_2}=0\Rightarrow w_2 = \lambda\frac{\partial f}{\partial z_2}$ \\
        $\frac{\partial\mathcal{L}}{\partial \lambda}=q-f(z_1,z_2)=0$ \\
      \end{center}

      (b) Envelope theorem:
      \begin{center}
        $\Aboxed{\frac{\partial C(w,q)}{\partial q}=\frac{\mathcal{L}}{\partial q}=\lambda}$
      \end{center}

      This implies that $\lambda$ is the marginal cost.

      (c) 
      $z_1$'s FOC gives us the price of input 1: 
      \begin{center}
        $w_1 = \lambda\frac{\partial f}{\partial z_1}$ \\
        $\lambda = \frac{w_1}{\frac{\partial f}{\partial z_1}}$ \\
      \end{center}
      As established above, $\lambda$ is marginal cost. The markup ratio is $\frac{p}{\lambda}$. So, we can define markup as follows,
      \begin{center}
        Markup = $\Aboxed{p\times\frac{\frac{\partial f}{\partial z_1}}{w_1}}$
      \end{center}

      So, we can conclude by saying that the economist can find markup by multiplying price by the ratio of product 1's marginal product and price. 
  \section{Question 6.B.2}
    \subsection*{Problem}
      Show that if the preference relation $\succeq$ on $\mathcal{L}$ is represented by a utility function $U(\cdot)$ that has the expected utility form, then $\succeq$ satisfies the independence axiom. 
    \subsection*{Solution}
      \begin{proof}
        Assume there exists a lottery with utility of the form $U(L)=\sum p_iu(x_i)$ such that $u(x_i)$ is the utility of outcome $x_i$. Next, allow for $L, L', L''\in \mathcal{L}$ and $\alpha\in(0,1)$. Now, assume $L\succeq L'$. This implies $U(L)\geq U(L')$. Consider a compound lottery in which $\alpha L+(1-\alpha)L'' \Rightarrow \alpha U(L)+(1-\alpha)U(L'')$. Then, consider the compound lottery $\alpha L' +(1-\alpha)L''\Rightarrow \alpha U(L')+(1-\alpha)U(L'')$. Because $U(L)\geq U(L')$, we can say that $\alpha U(L)+(1-\alpha)U(L'')\geq \alpha U(L')+(1-\alpha)U(L'')$. Then, $\alpha L+(1-\alpha)L''\succeq \alpha L'+(1-\alpha) L''$. The reverse can be shown via the same process. This shows that if one lottery is preferred to another, the compound lottery in which a third, less preferred, lottery is included will not change the preference ordering.  
      \end{proof}
\end{document}
