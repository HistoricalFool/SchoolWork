\documentclass[10pt,a4paper]{article}
\usepackage[top=3cm,bottom=4cm,left=3.5cm,right=3.5cm]{geometry}
\usepackage{amsmath,amsthm,amsfonts,amssymb,amscd}
\usepackage{fancyhdr,color,comment,graphicx,environ,float,mathtools,mathrsfs,bbm,listings}
\newcommand{\norm}[1]{\left\lVert#1\right\rVert}

% Custom headers
\pagestyle{fancy}
\lhead{ECON - 8050}
\chead{}
\rhead{Tate Mason}
\lfoot{}
\cfoot{Homework 4}
\rfoot{\thepage}

\begin{document}

\title{Homework 4}
\author{ECON 8050: Macroeconomics II \\ Tate Mason}
\date{}
\maketitle

The process for $y = log(income)$ is:
\begin{align*}
    y_{t+1} = \mu + \rho y_t + \sigma \varepsilon_{t+1}
\end{align*}
where $\varepsilon \sim N(0, 1)$

\begin{enumerate}
    \item[(1)] Set $\mu = 0$, $\rho = 0.9$ and $\sigma = 0.0242$. Discretize the process for $y$ with $9$ points. Download the Matlab code ghquad.m to compute Gauss-Hermit grids and weights. Use $10,000$ as maxit input. As an output, print out the vector of discretized $y$ and the transition matrix.
    
    \item[(2)] Simulate the Markov chain and compute the implied autocorrelation coefficient ($\hat{\rho}$). Note: use 1 million observations to simulate a persistent AR process. Disregard first 1000 observations. Report both $\hat{\rho}$ and $\hat{\sigma}$ computed from the simulated data.
\end{enumerate}

\section*{Solution 1}
  y - vector:
    \begin{gather*}
      \begin{pmatrix} -0.1092 & -0.776 & -0.503 & -0.0248 & 0.0000 & 0.0248 & 0.0503 & 0.0776 & 0.1092 \end{pmatrix}
    \end{gather*}
  Transition Matrix:
    \begin{gather*}
      \begin{bmatrix}
        0.5755 & 0.3551 & 0.0649 & 0.0044 & 0.0001 & 0.0000 & 0.000 & 0.0000 & 0.0000 \\
        0.1572 & 0.4517 & 0.3116 & 0.0729 & 0.0063 & 0.0002 & 0.0000 & 0.0000 & 0.0000 \\
        0.0179 & 0.1945 & 0.4222 & 0.2881 & 0.0708 & 0.0063 & 0.0002 & 0.0000 & 0.0000 \\
        0.0009 & 0.0349 & 0.2212 & 0.4099 & 0.2669 & 0.0623 & 0.0048 & 0.0001 & 0.0000 \\
        0.0000 & 0.0028 & 0.0499 & 0.2441 & 0.4063 & 0.2441 & 0.0499 & 0.0028 & 0.0000 \\
        0.0000 & 0.0001 & 0.0048 & 0.0623 & 0.2659 & 0.4099 & 0.2212 & 0.0349 & 0.0009 \\
        0.0000 & 0.0000 & 0.0002 & 0.0063 & 0.0708 & 0.2881 & 0.4222 & 0.1945 & 0.0179 \\
        0.0000 & 0.0000 & 0.0000 & 0.0002 & 0.0063 & 0.0729 & 0.3116 & 0.4517 & 0.1572 \\
        0.0000 & 0.0000 & 0.0000 & 0.0000 & 0.0001 & 0.0044 & 0.0649 & 0.3551 & 0.5755
      \end{bmatrix}
    \end{gather*}
\section*{Solution 2}
  \begin{gather*}
    \hat{\rho} = 0.8857 \\
    \hat{\sigma} = 0.0239
  \end{gather*}
\begin{lstlisting}
  %% PS4 - Tauchen-Hussey
  clear;
  clc;

  %% Setting parameters
  mu = 0;
  rho = 0.9;
  sigma = 0.0242;
  n = 9;

  maxit = 10000;

  %% Calling GHQuad
  [x,w] = ghquad(n, maxit);

  %% Problem 1

  %Calculating Unconditional Distribution
  uncon_mean = mu / (1-rho);
  uncon_sd = sigma / sqrt(1-rho^2);

  %Calculate Steps and Discretize Space
  m = 3;

  y_state = zeros(n,1);
  for i = 1:n
    y_state(i) = uncon_mean + sqrt(2)*sigma*x(i);
  end

  % Transition Probability
  P = zeros(n, n);

  for i = 1:n
    mean_next = mu + rho*y_state(i);
    for j = 1:n
      P(i,j) = w(j)/sqrt(pi)*exp(-(0.5)*((y_state(j) - rho*y_state(i))/sigma)^2)/exp(-x(j)^2);
    end
    dens = sum(P(i,:));
    P(i,:) = P(i,:)/dens;
  end

  %Print results
  disp('Discretized State Space for y: ');
  disp(y_state);

  disp('Transition Matrix: ');
  disp(P);

  %% Problem 2

n = 1000000;
dis = 1000;

y_sim = zeros(n, 1);

init = randi(length(y_state));
y_sim(1) = y_state(init);

current_state = init;

for t = 2:n
  cum_prob = cumsum(P(current_state, :));
  r = rand();
  next_state = find(r <= cum_prob, 1, 'first');
  y_sim(t) = y_state(next_state);
  current_state = next_state;
end

y_sim_final = y_sim(dis+1:end);

y_t = y_sim_final(1:end-1);
y_tp1 = y_sim_final(2:end);

y_mean = mean(y_sim_final);
y_demean = y_t - y_mean;
y_tp1_demean = y_tp1 - y_mean;

rho_hat = (y_demean' * y_tp1_demean)/(y_demean'*y_demean);

e_hat = y_tp1 - (mu + rho_hat*y_t);
sigma_hat = sqrt(mean(e_hat.^2));

fprintf('Estimated autocorrelation coefficient (rho_hat): %.6f\n', rho_hat);
fprintf('Estimated variance (sigma_hat): %.6f\n', sigma_hat);
\end{lstlisting}
\end{document}
