\documentclass[10pt,a4paper]{article}
\usepackage[top=3cm,bottom=4cm,left=3.5cm,right=3.5cm]{geometry}
\usepackage{amsmath,amsthm,amsfonts,amssymb,amscd,fancyhdr,color,comment,graphicx,environ,float,mathtools,mathrsfs,unicode-math}

% Custom Commands
\newcommand{\norm}[1]{\left\lVert#1\right\rVert}
\newcommand{\sumt}{$\sum\limits_{t=0}^{\infty}$}

% Document Information
\title{Homework 2}
\author{Tate Mason}
\date{ECON - 8050}

\begin{document}

\maketitle

%%%%%%%%%%%%%%%%%%%%%%%%%%%%%%%%%%%%%%%%%%%%%
%% Problem 1 - Business Cycle and whatnot  %%
%%%%%%%%%%%%%%%%%%%%%%%%%%%%%%%%%%%%%%%%%%%%%
\section*{Problem 1: Costs of Business Cycle}
  Let utility be given by:
  \begin{equation*}
      E_{-1} \sum_{t=0}^{\infty} \beta^t U(c_t)
  \end{equation*}
  where the utility function is CRRA:
  \begin{equation*}
      U(c_t) = \frac{c_t^{1-\gamma}}{1- \gamma}
  \end{equation*}

  The consumption process is
  \begin{equation*}
      c_t = c_{t-1}^{\alpha} \varepsilon_t \exp(\mu)
  \end{equation*}
  where
  \begin{equation*}
      \mu = \frac{-\sigma_{\varepsilon}^2 (1-\alpha)}{2 (1-\alpha^2)}, \quad \log \varepsilon_t \sim N(0, \sigma_{\varepsilon}^2) \text{ and i.i.d.}
  \end{equation*}
  Thus, the log of consumption follows an AR(1) process:
  \begin{equation*}
      \log c_t = \mu + \alpha \log c_{t-1} + \log \varepsilon_t
  \end{equation*}

  \subsection*{Part A}
    Find the unconditional mean of $c_t$, $E(c_t)$. (Hint: recall the properties of the lognormal distribution).

  \subsection*{Part B}
    Define lifetime utility before any uncertainty is realized as:
    \begin{equation*}
        V_0 = E_{-1} \sum_{t=0}^{\infty} \beta^t U(c_t)
    \end{equation*}
    Assume $c_0$ is drawn from the invariant (unconditional) distribution of $c$.
    Now define:
    \begin{equation*}
        V(\lambda) = E_{-1} \sum_{t=0}^{\infty} \beta^t U [c_t(1 + \lambda)]
    \end{equation*}
    This is lifetime utility when every period consumption is increased by $(1+\lambda)$. Express $V(\lambda)$ as a function of $\mu, \sigma_{\varepsilon}^2, \alpha, \gamma, \beta$.

  \subsection*{Part C}
    Denote $V_0$ as the lifetime utility when $c_t$ is deterministic and equal to its unconditional mean found in part A). Find the compensation $\lambda$ such that $V(\lambda) = V_0$. Find how much compensation the consumer has to be given in order to be indifferent between the stochastic and deterministic cases, Provide economic intuition.

  \subsection*{Part D}
    Denote the interest rate as $r$. Find consumption $c_t$.
%%%%%%%%%%%%%%%%%%%%%%%%%%%%%%%%%%%%%%%%%%%%%
%% Problem 2 - Non-Expected Utility        %%
%%%%%%%%%%%%%%%%%%%%%%%%%%%%%%%%%%%%%%%%%%%%%
\section*{Problem 2: Non-Expected Utility Framework}
  This problem follows the Kreps and Porteus (1978), Epstein and Zin (1991), and Weil (1990) frameworks.

  Let remaining lifetime utility at time $t$, once $c_t$ is known, be given by $v_t$, satisfying:
  \begin{equation}
      v_t = \left[ (1-\beta) c_t^{\rho} + \beta (E_t v_{t+1}^{\alpha})^{\frac{\rho}{\alpha}} \right]^{\frac{1}{\rho}}
  \end{equation}
  where $1-\alpha$ represents risk aversion and $1-\rho$ represents the inverse of the intertemporal elasticity of substitution. In standard expected utility, $\alpha = \rho$.

  Denote pre-realization lifetime utility at time $t$ as $U_t$, where:
  \begin{equation*}
      U_t = (E_t v_t^{\alpha})^{\frac{1}{\alpha}}
  \end{equation*}

  \subsection*{Part A}
    Prove that multiplying $c_t$ by $\lambda$ for all $t = 0, 1, \dots, \infty$ is equivalent to multiplying $v_t$ by $\lambda$.
    (Hint: start by assuming this holds, substitute into equation (1), and show $v_t$ scales linearly.)

  \subsection*{Part B}
    Suppose for all $t$, we replace $c_t$ with a deterministic constant $\bar{c} = E[c_t]$. Compare welfare in this case with uncertain $c_t$. Specifically, find $\eta$ such that multiplying $c_t$ by $(1+ \eta)$ makes ex-ante welfare $U_0$ equal to that in the deterministic case. Express $\eta$ in terms of $U_0$ and $\bar{c}$.

  \subsection*{Part C}
    Suppose consumption follows one of two sequences: with probability $\frac{1}{2}$, $c_t = c_l$ for all $t$, and with probability $\frac{1}{2}$, $c_t = c_h$ for all $t$. The sequence is revealed at $t = 0$. Find $\eta$ and analyze its dependence on $\rho$ and $\alpha$.

  \subsection*{Part D}
    Now assume $c_t$ is i.i.d., where each period $c_t = c_l$ with probability $\frac{1}{2}$ and $c_h$ with probability $\frac{1}{2}$.
    \begin{enumerate}
        \item Derive an implicit equation for $U_0$.
        \item Analyze whether $\eta$ depends on $\alpha$ and $\rho$.
    \end{enumerate}

  \subsection*{Part E}
    Solve for $U_0$ numerically using Matlab with given parameters: $\beta = 0.95$, $c_l = e^{0.98}$, $c_h = e^{1.02}$. Compute $\eta$ for:
    \begin{itemize}
        \item $\alpha = 1, 0.5, -1$
        \item $\rho = 1, 0.5, -1$
    \end{itemize}
    Report results in a table and provide economic intuition.
    (Hint: Use an iterative approach to solve $U_0 = f(U_0)$ until convergence with tolerance $10^{-8}$.)
\section{Solution 1}
  \subsection*{(A)}
    If $c_t\sim N(m,v)$ for some mean $m$ and variance $v$, $\log c_t$ has a log normal distribution such that $\mathbb{E}[c_t] = \exp[m + \frac{v}{2}]$. Due to $c_t$ exhibiting an AR(1) process, we can say that $m = \mu + \alpha m + 0\rightarrow m(1-\alpha) = \mu\rightarrow m = \frac{\mu}{1-\alpha}$. The unconditional variance can be found by the following $v = \alpha^2v + \sigma_{\epsilon}^2\rightarrow v=\frac{\sigma_{\epsilon}^2}{(1-\alpha^2)}$. Thus, $\mathbb{E}[c_t] = \exp[\frac{\mu}{1-\alpha} + \frac{\sigma_{\epsilon}^2}{2(1-\alpha^2)}]$ Subbing in the given $\mu$, $ \exp[-\frac{\sigma_{\epsilon}^2\mu}{2(1-\alpha^2)(1-\alpha)} + \frac{\sigma_{\epsilon}^2}{2(1-\alpha^2)}]\rightarrow\exp(0) = 1 = \mathbb{E}[c_t]$   
  \subsection*{(B)}
    \begin{gather*}
      V_0 = \mathbb{E}_{-1}\sumt\beta^tU(c_t) \\
      V(\lambda) = \mathbb{E}_{-1}\sumt\beta^tU(c_t(1+\lambda)) \\
      U(c_t(1+\lambda)) = (1+\lambda)^{1-\gamma}\frac{c_t^{1-\gamma}}{1-\gamma} \\
      V(\lambda) = (1+\lambda)^{1-\gamma}\mathbb{E}_{-1}\sumt\beta^t\frac{c_t^{1-\gamma}}{1-\gamma} = (1+\lambda)^{1-\gamma}V_0 \\
    \end{gather*}
    Now, we apply the same distribution as in (A) to $c_{t}^{1-\gamma}$.
    \begin{gather*}
      \mathbb{E}[c_t^{1-\gamma}] = \exp[(1-\gamma)m + \frac{(1-\gamma)v}{2}] \\
      \mathbb{E}[c_t^{1-\gamma}] = \exp[\frac{(1-\gamma)\mu}{1-\alpha} + \frac{(1-\gamma)^2\sigma_{\epsilon}^2}{2(1-\alpha^2)}] \\
      \therefore \ V_0 = \frac{1}{1-\gamma}\sumt\beta^t\mathbb{E}[c_t^{1-\gamma}] \\
      = \frac{1}{1-\gamma}\frac{1}{1-\beta}\exp[\frac{(1-\gamma)\mu}{1-\alpha} + \frac{(1-\gamma)^2\sigma_{\epsilon}^2}{2(1-\alpha^2)}] \\
      V(\lambda) = (1+\lambda)^{1-\gamma}\frac{1}{1-\gamma}\frac{1}{1-\beta}\exp[\frac{(1-\gamma)\mu}{1-\alpha} + \frac{(1-\gamma)\sigma_{\epsilon}^2}{2(1-\alpha^2)}]
    \end{gather*}
  \subsection*{(C)}
    \begin{gather*}
      V_0 = \sumt\beta^tU(1) = \sumt\beta^t\frac{1}{1-\gamma} = \frac{1}{(1-\beta)(1-\gamma)}
    \end{gather*}
    Indifference implies that $V_0 = V(\lambda)$. Therefore:
    \begin{gather*}
      (1+\lambda)^{1-\gamma}\exp[\frac{(1-\gamma)\mu}{1-\alpha} + \frac{(1-\gamma)\sigma_{\epsilon}^2}{2(1-\alpha^2)}] = 1 \\
      \Rightarrow (1-\gamma)\log(1+\gamma)+\frac{(1-\gamma)\sigma_{\epsilon}^2}{(1-\alpha)} + \frac{(1-\gamma)^2\sigma_{\epsilon}^2}{2(1-\alpha^2)} = 0 \\
      \log(1+\lambda) = -\frac{\mu}{1-\alpha} + \frac{(1-\gamma)\sigma_{\epsilon}^2}{2(1-\alpha^2)} \\
      1+\lambda = \exp[-\frac{\mu}{1-\alpha} + \frac{(1-\gamma)\sigma_{\epsilon}^2}{2(1-\alpha^2)}] \\
      \lambda = \exp[-\frac{\mu}{1-\alpha} + \frac{(1-\gamma)\sigma_{\epsilon}^2}{2(1-\alpha^2)}] - 1 \\
      \lambda = \exp[\frac{\sigma_{\epsilon}^2}{2(1-\alpha^2)} - \frac{(1-\gamma)\sigma_{\epsilon}^2}{2(1-\alpha^2)}] - 1 \\
      \lambda = \exp[\frac{\sigma_{\epsilon}^2}{2(1-\alpha^2)} (1-1+\gamma)] -1 \\
      \lambda = \exp[\frac{\gamma\sigma_{\epsilon}^2}{2(1-\alpha^2)}] - 1 \\
    \end{gather*}
  \subsection*{(D)}
    \begin{gather*}
      c_t^{-\gamma} = \beta(1+r)\mathbb{E}_t(c_{t+1}^{-\gamma}) \\
      1 = \beta(1+r)\mathbb{E}_t(\frac{c_t^{\alpha}\epsilon_{t+1}\exp[\mu]}{c_{t-1}\epsilon_{t}\exp[\mu]})^{-\gamma} \\
      \beta(1+r)\mathbb{E}_t(\alpha\log\frac{c_{t+1}}{c_t} + \log(\frac{\epsilon_{t+1}}{\epsilon_{t}})+1) = 1 \\
      \beta(1+r)\mathbb{E}_t(\alpha\Delta\log c_{t+1} + 1) \ \text{s.t. $\Delta\log c_{t+1} \sim N(\mathbb{E}_t\Delta\log c_{t+1}, v_t\Delta\log c_{t+1})$} \\
      \mathbb{E}_t(-\gamma\alpha\Delta\log c_{t+1} + 1) = (-\gamma\alpha\Delta\log c_{t+1}+\frac{1}{2}(\gamma\alpha)^2v_t\Delta\log c_{t+1}) \\
      \mathbb{E}_t\Delta\log c_{t+1} = \frac{\log\beta(1+r)}{\gamma\alpha} + \frac{1}{2}\gamma\alpha v_t\Delta\log c_{t+1} \\
    \end{gather*}
\section*{Solution 2}
  \subsection*{(A)}
    \begin{proof}
      $v_t = [(1-\beta)(\lambda c_t)^{\rho} + \beta(\mathbb{E}[\lambda v_{t+1}]^{\alpha})^{\frac{\rho}{\alpha}}]^{\frac{1}{\rho}}$  This allows us to move the $\lambda$ term out such that $v_t = [\lambda^{\rho}(1-\beta)(c_t^{\rho}) + \lambda^{\rho}\beta(\mathbb{E}[v_t]^{\alpha})^{\frac{\rho}{\alpha}}]^{\frac{1}{\rho}}$. Finally, we can show that utility is linearly scaled by lambda such that $v_t = \lambda[(1-\beta)c_t^{\rho}+\beta(\mathbb{E}c_t^{\alpha})^{\frac{\rho}{\alpha}}]^{\frac{1}{\rho}}$.
    \end{proof}
  \subsection*{(B)}
    In the deterministic case, $v_t=v_{t+1}=U_0^d$. This allows us to write the uncertain case as $U_0^c(1+\eta)=U_0^d \Rightarrow \eta=\frac{U_0^d}{U_0^c}-1\Rightarrow \eta = \frac{\bar{c}}{U_0^c}$. 
  \subsection*{(C)}
    Using the probabilities $\Pi_L = 0.5, \ \Pi_H = 0.5$, we can derive value functions for the cases in which you get a low draw and a high draw.
    \begin{gather*}
      \bar{c} = \Pi_Lc_L + \Pi_Hc_H \\
      U_0 = (\mathbb{E}[v_0^{\alpha}])^{\frac{1}{\alpha}} = [\Pi_Lc_L^{\alpha} + \Pi_Hc_H^{\alpha}]^{\frac{1}{\alpha}} \\
      \shortintertext{Plugging into $\eta$ formula derived above} \\
      \eta = \frac{\Pi_Lc_L + \Pi_Hc_H}{[\Pi_Lc_L^{\alpha} + \Pi_Hc_H^{\alpha}]^{\frac{1}{\alpha}}} -1 \\
    \end{gather*}
    This shows that $\eta$ depends on the risk aversion parameter $\alpha$. $\rho$ does not appear as at period $t=0$, consumption is constant and sees no uncertainty. The agent feels no disutility from pushing consumption to a future period $t$. 
  \subsection*{(D)}
    Let's start with the case in which $c_t = c_L$.
    \begin{gather*}
      v_L = [(1-\beta)c_L^{\rho} + \beta(\mathbb{E}[v_{t+1}^{\alpha}])^{\frac{\rho}{\alpha}}]^{\frac{1}{\rho}} \\
    \end{gather*}
    Next, in the case in which $c_t = c_H$,
    \begin{gather*}
      v_H = [(1-\beta)c_H^{\rho} + \beta(\mathbb{E}[v_{t+1}^{\alpha}])^{\frac{\rho}{\alpha}}]^{\frac{1}{\rho}} \\
    \end{gather*}
    Now, allow $R = \mathbb{E}[v_{t+1}^{\alpha}]^{\frac{1}{\alpha}} = [\Pi_Lv_L^{\alpha} + \Pi_Hv_H^{\alpha}]^{\frac{1}{\alpha}}$. From here, we can update the value functions in each state:
    \begin{gather*}
      v_L = [(1-\beta)c_L^{\rho} + \beta R^{\rho}]^{\frac{1}{\rho}} \\
      v_H = [(1-\beta)c_H^{\rho} + \beta R^{\rho}]^{\frac{1}{\rho}} \\
    \end{gather*}
    Now, we can use these functions to update $U_0$:
    \begin{gather*}
      U_0 = (\mathbb{E}_0[v_0^{\alpha}])^{\frac{1}{\alpha}} -R \\
      U_0 = \{\Pi_L[(1-\beta)c_L^{\rho} + \beta U_0]^{\frac{\alpha}{\rho}} + \Pi_H[(1-\beta)c_H^{\rho}+\beta U_0^{\rho})]^{\frac{\alpha}{\rho}}\} \\
    \end{gather*}
    Here, in this case, $\alpha, \ \rho$ are present due to the fact that consumption can now vary in each period. This causes the agent to experience disutility $\rho$ for creating a buffer stock as well as their risk aversion $\alpha$. 
  \subsection*{(E)}
  From the tables below, it can be seen that risk aversion $\alpha$ has very little effect, staying stagnant regardless of the level of the parameter. $\rho$, however, decreases $\eta$ as it decreases as well. Thus, the disutility of pushing consumption plays a factor in determining $\eta$.  
\newpage
    \begin{tabular}{ccc}
    Alpha & Eta    & Iterations \\
    \hline
    1     & 2.7188 & 312        \\
    0.5   & 2.7188 & 312        \\
    -1    & 2.7188 & 312       
    \end{tabular}
    \quad
    \begin{tabular}{ccc}
    Rho & Eta    & Iterations \\
    \hline
    1   & 2.7188 & 312        \\
    0.5 & 2.7186 & 316        \\
    -1  & 2.7178 & 331       
    \end{tabular}
\end{document}
