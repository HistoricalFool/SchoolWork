\documentclass[10pt,a4paper]{article}
\usepackage[top=3cm,bottom=4cm,left=3.5cm,right=3.5cm]{geometry}
\usepackage{amsmath,amsthm,amsfonts,amssymb,amscd}
\usepackage{fancyhdr,color,comment,graphicx,environ,float,mathtools,mathrsfs}
\newcommand{\norm}[1]{\left\lVert#1\right\rVert}

% Custom headers
\pagestyle{fancy}
\lhead{ECON - 8020}
\chead{}
\rhead{Tate Mason}
\lfoot{}
\cfoot{Assignment 4}
\rfoot{\thepage}

\begin{document}

\title{Assignment 4}
\author{Tate Mason}
\date{Due: March 20th, 11:59pm}
\maketitle

\section*{Question 1 (8pts)}
  Suppose there are $N$ agents and let $\langle M, x, t \rangle$ be a dominant-incentive compatible mechanism.

  \begin{enumerate}
      \item[(a)] What does it mean for $\langle M, x, t \rangle$ to be dominant-incentive compatible? Express the definition mathematically.
      \item[(b)] Prove the revelation principle for dominant-incentive compatible mechanisms.
  \end{enumerate}

\section*{Question 2 (10pts)}
  Prove Theorem 6.2 in the lecture notes. That is, prove that any VCG mechanism (e.g. mechanism satisfying Definition 6.1) is dominant-incentive compatible.

\section*{Question 3 (12pts)}
  Suppose there are three agents: $I_1$, $I_2$, and $I_3$. There is a mechanism designer in possession of a single nuclear weapon, which he would like to allocate. He can either give it to one of the agents OR he may choose to not allocate the weapon at all. Furthermore, the designer has opted to run a VCG mechanism to select the outcome. The agents' values over these outcomes is the following:

  \begin{center}
    \begin{tabular}{|c|c|c|c|c|}
      \hline
      & \textbf{$I_1$ Receives} & \textbf{$I_2$ Receives} & \textbf{$I_3$ Receives} & \textbf{Not Allocated} \\
      \hline
      $I_1$ & 95 & -90 & -10 & 0 \\
      \hline
      $I_2$ & -50 & 90 & -25 & 0 \\
      \hline
      $I_3$ & -60 & 10 & 40 & 0 \\
      \hline
    \end{tabular}
  \end{center}

  Notice that the payoffs for some of the agents are negative. This is an example of an environment with negative externalities. An agent does not just have a value over the outcome where she receives the weapon; she is affected if the weapon is allocated to a different agent. Agents can choose whether or not to participate in the mechanism, but cannot avoid the payoffs associated with each allocation.

  \begin{enumerate}
      \item[(a)] Suppose all agents choose to participate in the mechanism. Determine the allocation and payment rule (show all your work). When determining the payment, indicate if the agent is paying or receiving a payment. What are the net payoffs for all the agents?
      \item[(b)] Suppose $I_2$ decided to not participate in the VCG mechanism, and so only $I_1$ and $I_3$ were participants. If VCG were run with just these two agents, what outcome would be selected? Under this outcome, what would $I_2$'s payoff be? If the other two agents were participating, would $I_2$ unilaterally choose to not participate?
      \item[(c)] Repeat part (b) for the other two agents.
  \end{enumerate}

\section*{Question 4 (10pts)}
  Consider a standard mechanism design environment with $n$ agents and a set of possible outcomes $\mathcal{X}$. Each agent $i$ has a private type $\theta_i \in \Theta_i$, where $\Theta_i$ is some set. Agent $i$'s value for outcome $x$ when her type is $\theta_i$ is $v_i(x, \theta_i)$.

  Let $\mathcal{X}_{-i}$ denote the set of feasible outcomes when agent $i$ does not exist. Assume that $\mathcal{X}_{-i} \subset \mathcal{X}$ for each $i$. That is, if agent $i$ did not exist, the set of feasible outcomes stays the same or decreases.

  Prove that if there are no negative-externalities $v_i(x, \theta_i) \geq 0$ for all $i$ and $x \in \mathcal{X}_{-i}$, then the VCG mechanism is ex-post individually rational.

\section*{Question 5 (20pts)}
  Consider the problem of providing an indivisible public good with two agents with quasilinear utilities. The two agents have privately observed values $v_1, v_2 \in [0, \bar{v}]$ for the public good.

  The cost of providing the public good is $c \in (\bar{v}, 2\bar{v})$. The provision mechanism must have a balanced budget: the agents' monetary contributions must add up to $c$ when the good is provided, and to 0 otherwise.

  For this problem, we are restricting attention to mechanisms that are dominant-strategy incentive-compatible and ex-post individually rational. Mechanisms are permitted to be "random". In other words, the probability of providing the public good when $(v_1, v_2)$ is reported is $x(v_1, v_2) \in [0, 1]$. We will call $(v_1, v_2)$ the state.

  \begin{enumerate}
      \item[(a)] Write the dominant-strategy incentive compatibility condition for a mechanism. Compute the equilibrium payoffs of agent $i$ when agent $i$'s value is $v_i$.
      \item[(b)] Using your answer to (a), write the condition that the two expected payoffs must add up to the total expected surplus in that state.
      \item[(c)] Suppose $x(v_1, v_2)$ is twice continuously differentiable. What property must $x$ have in order to satisfy the condition in (b)?
      \item[(d)] A fixed-contribution mechanism is one which provides a public good if and only if each agent $i$ is willing to contributed $c_i$, where $c_1 + c_2 = c$. Show that any mechanism considered in (c) is equivalent to a randomized fixed-contribution mechanism.
  \end{enumerate}

\section*{Solution 1}
  \subsection*{Part (A)}
    To be dominant incentive compatible is to be strategy-proof. Otherwise stated as the mechanism is optimal for an agent no matter the moves of others. Mathematically, this can be represented as follows. For Bayesian Incentive compatible:
    \begin{gather*}
      \mathbb{E}[v_i(x_i(\theta_i, \theta_{-i}),\theta) - t_i(\theta_i,\theta_{-i})|\theta_{-i}] \geq \mathbb{E}[v_i(x_i(\hat{\theta_i}, \theta_{-i}),\theta) - t_i(\hat{\theta_i},\theta_{-i})|\theta_{-i}]
    \end{gather*}
    For DIC:
    \begin{gather*}
      v_i(x_i(\theta_i,\theta_{-i},\theta) - t_i(\theta_i,\theta_{i}) \geq v_i(x_i(\hat{\theta_i},\theta_{-i})) - t_i(\hat{\theta_i},\theta_{-i}) \forall \theta_i, \theta_{-i}, \hat{\theta}
    \end{gather*}
  \subsection*{Part (B)}
    \begin{proof}
      Consider arbitrary mechanism $\langle M, x, t \rangle$ with dominant strategy equilibrium $s^*(\theta) = (s_1\theta_1, s_2\theta_2, ..., s_n\theta_n)$. Because $\exists$ equilibrium $s^*(\theta)$, we can use the revelation principle to show direct mechanism $\langle \Theta, x', t' \rangle$ such that $\Theta$ is the type space, $x'(\theta)$ is the allocation rule, and $t'(\theta)$ is the payment rule. Under the direct mechanism, it is opitmal for players to report truthfully. 

      Any game which has a dominant strategy has at least one equilibrium in which all players will play said strategy. If $\langle M, x, t \rangle$ is DIC, the equilibrium is that all players will play their type truthfully. Thus, the indirect and direct mechanisms are equivalent in the case in which both agents choose to report truthfully.
    \end{proof}
\section*{Solution 2}
  \begin{proof}
    Consider an agent i with type $\theta_i$. Further, consider the agent has the option to report their type as $\hat{\theta}_i$, a type which differs from their true type. When reporting truthfully, agent $i$ has utility:
    \begin{equation}
      u_i(\theta_i,\theta_{-i}) = v_i(x(\theta_i,\theta_{-i}),\theta_i) - t_i(\theta_i, \theta_{-i})
    \end{equation}
    The payment rule given in definition 6.1 can now be subbed in such that:
    \begin{equation}
      u_i(\theta_i,\theta_{-i}) = v_i(x(\theta_i,\theta_{-i}),\theta_i) + \sum\limits_{j\ne i}v_j(x(\theta_i, \theta_{-i}), \theta_j)-h_j(\theta_{-i})
    \end{equation}
    Using the same steps, we can produce utility from reporting something besides their type:
    \begin{equation}
      u_i(\hat{\theta}_i, \theta_{-i}) = v_i(x(\hat{\theta}_i,\theta_{-i}),\theta_i) - t_i(\hat{\theta}_i, \theta_{-i})
    \end{equation}
    \begin{equation}
      u_i(\hat{\theta}_i,\theta_{-i}) = v_i(x(\hat{\theta}_i,\theta_{-i}),\theta_i) + \sum\limits_{j\ne i}v_j(x(\hat{\theta}_i, \theta_{-i}), \theta_j)-h_j(\theta_{-i})
    \end{equation}
    To show that truthful reporting is optimal, $u(\theta_i,\theta_{-i})\geq u(\hat{\theta_i},\theta_{-i})$ needs to be shown. 
    \begin{equation}
      v_i(x(\theta_i,\theta_{-i}),\theta_i) + \sum\limits_{j\ne i}v_j(x(\theta_i, \theta_{-i}), \theta_j)-h_j(\theta_{-i}) \geq v_i(x(\hat{\theta}_i,\theta_{-i}),\theta_i) + \sum\limits_{j\ne i}v_j(x(\hat{\theta}_i, \theta_{-i}), \theta_j)-h_j(\theta_{-i})
    \end{equation}
    It should be apparent that the function $h_j(\theta_{-i})$, a function which relates to the reports of other agents, cancels out leaving us with: 
    \begin{equation}
      v_i(x(\theta_i,\theta_{-i}),\theta_i) + \sum\limits_{j\ne i}v_j(x(\theta_i, \theta_{-i}), \theta_j)\geq v_i(x(\hat{\theta}_i,\theta_{-i}),\theta_i) + \sum\limits_{j\ne i}v_j(x(\hat{\theta}_i, \theta_{-i}), \theta_j)
    \end{equation}
    This leaves an inequality that shows that it is optimal to report truthfully regardless of the reports of others, shown by the step in which $h_j(\theta_{-i})$, otherwise showing that any VCG mechanism is DIC. 
  \end{proof}
\section*{Solution 3}
  \subsection*{Part (A)}
    Beginning the allocation:
    
    If all agents participate, the good will be allocated to agent 2.

    Now, for transfers:
    \begin{gather*}
      t_1 = (90+10) - (90+10) = 0 \\
      t_2 = (95-60) - (-90+10) = 115 \\
      t_3 = (95-50) - (-90+90) = 45 \\
    \end{gather*}
    Finally, net transfers are as follows:
    \begin{gather*}
      n_1 = -90 - 0 = -90 \\
      n_2 = 90 - 115 = -25 \\
      n_3 = 10 - 45 = -35
    \end{gather*}
  \subsection*{Part (B)}
    The transfers for $I_1, \ I_3$ can be described as 
    \begin{gather*}
      t_1 = 40 - (-60) = 100 \\
      t_3 = 95 - 95 = 0 \\
    \end{gather*}
    And net payoffs as
    \begin{gather*}
      n_1 = 95 - 100 = -5 \\
      n_3 = -60 -0 = -60 \\
    \end{gather*}
    $I_2$ would incur a negative externality of value $-50$, the value of $I_1$ receiving the weapon. Because their net pay is $-25$, they will choose to stay in the allocation game since the cost to participate is lower than the cost to sit out. 
  \subsection*{Part (C)}
    \subsubsection*{Agent 1}
    Transfers can be determined as the following:
    \begin{gather*}
      t_2 = 40-10 = 30 \\
      t_3 = 90-90 = 0
    \end{gather*}
    Net payoffs:
    \begin{gather*}
      n_2 = 90-30 = 60 \\
      n_3 = 10 - 0 = 10 \\
    \end{gather*}
    $I_1$ would experience a negative payoff of $-90$ when the weapon is allocated to $I_2$ , which is equal to their net payoff when participating. This would cause them to continue to participate as they would not experience a negative payoff. 
    \subsubsection*{Agent 3}
      Transfers:
      \begin{gather*}
        t_1 = 90-(-50) = 140 \\
        t_2 = 95-95 = 0 \\
      \end{gather*}
      Net pay:
      \begin{gather*}
        n_1 = 90 - 140 = -50 \\
        n_2 = -50 - 0 = -50 \\
      \end{gather*}
      $I_3$ would experience a negative payoff of $-60$, with the weapon allocated to $I_1$, when they abstain. They will choose to participate as their net utility is higher than the utility of sitting out. 
\section*{Solution 4}
  \begin{proof}
    As shown before, we can write the VCG payoff function as follows:
    \begin{equation}
      v_i(x(\theta_i),\theta_i) - t_i(\theta_i) 
    \end{equation}
    Let's plug in for $t_i(\theta_i)$
    \begin{equation}
      v_i(x(\theta_i),\theta_i) - {\max\atop{s\in\Chi_{-i}}}\sum\limits_{j\ne i}v_j*s, \theta_j) + \sum\limits_{j\ne i}v_i(x,\theta_j)
    \end{equation}
    Now, we can transform the equation as shown in class to be 
    \begin{equation}
      =\sum\limits_{i=1}^nv_i(x(\theta_i),\theta_i) - {\max\atop{s\in\chi_{-i}}}\sum\limits_{j\ne i}v_j(s, \theta_j)
    \end{equation}
    Since $\mathcal{X}_{-i}\subset\mathcal{X}$
    \begin{equation}
      \sum\limits_{i=1}^nv_i(x(\theta_i),\theta_i) \geq {\max\atop{s\in\chi_{-i}}}\sum\limits_{j\ne i}v_j(s, \theta_j)
    \end{equation}
    The left side includes all outcomes, and thus the right side must be less than or equal. This is since the right side only considers a subset of all outcomes. Further, 
    \begin{equation}
      {\max\atop{s\in\mathcal{X}_{-i}}}\sum\limits_{i=1}^nv_i(s(\theta_i),\theta_i) \geq {\max\atop{s\in\chi_{-i}}}\sum\limits_{j\ne i}v_j(s, \theta_j)
    \end{equation}
    The left side necessarily has more agents. Because there are no negative externalities, it must also be true that 
    \begin{equation}
      \sum\limits_{i=1}^nv_i(x(\theta_i),\theta_i)\geq {\max\atop{s\in\chi_{-i}}}\sum\limits_{j\ne i}v_j(s, \theta_j)
    \end{equation}
    This states that the payoff for any agent i is non-negative, also implying the VCG is ex-post IR.
  \end{proof}

\section*{Solution 5}
  \subsection*{Part (A)}
    For agent one, the DSIC is as follows:
    \begin{equation}
      x(v_1,v_2)v_1 - t_1(v_1,v_2) \geq x(\hat{v_1},v_2)v_1 - t_1(\hat{v_1},v_2)
    \end{equation}
    For agent two, the DSIC is much the same:
    \begin{equation}
      x(v_1,v_2)v_2 - t_2(v_1,v_2) \geq x(v_1,\hat{v_2})v_2 - t_2(v_1,\hat{v_2})
    \end{equation}
    Further, we know that the mechanism must be budget balanced such that 
    \begin{equation}
      \sum\limits_{i=1}^2 t_i(v_1,v_2) = \begin{cases}
        c & if \ x(v_1,v_2) = 1, \\
        0 & otherwise
      \end{cases}
    \end{equation}
    Now, let's calculate payoffs:
    \subsubsection*{Agent 1:}
      \begin{gather*}
        U_1(v_1,v_2) = x(v_1,v_2)v_1 - t_1(v_1,v_2) \\
        \shortintertext{Using the envelope theorem:} \\
        \frac{\partial U_1(v_1,v_2)}{\partial v_1} = x(v_1,v_2) \\
        \shortintertext{Integrating:} \\
        U_1(v_1,v_2) = U_1(0,v_2) + \int\limits_0^{v_1}x(s,v_2)ds
      \end{gather*}
    \subsubsection*{Agent 2:}
      \begin{gather*}
        U_2(v_1,v_2) = x(v_1,v_2)v_2 - t_2(v_1,v_2) \\
        \shortintertext{Using the envelope theorem:} \\
        \frac{\partial U_2(v_1,v_2)}{\partial v_2} = x(v_1,v_2) \\
        \shortintertext{Integrating:} \\
        U_2(v_1,v_2) = U_2(v_1,0) + \int\limits_0^{v_2}x(v_1,s)ds
      \end{gather*}
  \subsection*{Part (B)}
    Define total expected surplus as follows:
    \begin{equation}
      x(v_1,v_2)(v_1+v_2-c) \\
    \end{equation}
    Set expected surplus equal to the sum of expected utilities:
    \begin{gather*}
      x(v_1,v_2)(v_1+v_2-c) = [U_1(0,v_2) + \int\limits_0^{v_1}x(s,v_2)ds] + [U_2(v_1,0) + \int\limits_0^{v_2}x(v_1,s)ds] \\
    \end{gather*}
  \subsection*{Part (C)}
    Now, Start with taking cross-partial derivative of $x(v_1,v_2)(v_1+v_2-c) = [U_1(0,v_2) + \int\limits_0^{v_1}x(s,v_2)ds] + [U_2(v_1,0) + \int\limits_0^{v_2}x(v_1,s)ds]$.
    \begin{gather*}
      \shortintertext{Start by differentiating each side of the equality w.r.t $v_1$} \\
      \shortintertext{Starting with the left hand side:} \\
      \frac{\partial x(v_1,v_2)}{\partial v_1}[\int\limits_0^{v_1}x(s,v_2)df+\int\limits_0^{v_2}x(v_1,s)ds] \\
      \Rightarrow x(v_1,v_2)+\int\limits_0^{v_2}\frac{\partial x(v_1,s)}{\partial v_1}ds \\
      \shortintertext{Now, the right hand side:} \\
      \frac{\partial x(v_1,v_2)}{\partial v_1}[(v_1+v_2-c)x(v_1,v_2)] \\
      \Rightarrow \ \frac{\partial x(v_1,v_2)}{\partial v_1}(v_1+v_2-c) + x(v_1,v_2) \\
      \shortintertext{Simplifying and grouping yields:} \\
      \frac{\partial x(v_1,v_2)}{\partial v_1}(v_1+v_2-c) = \int\limits_0^{v_2}\frac{\partial x(v_1,s)}{\partial v_1} \\
      \shortintertext{Now, we differentiate that w.r.t $v_2$} \\
      \shortintertext{Again, starting with the left:} \\
      \frac{\partial x(v_1,v_2)}{\partial v_2}[\int\limits_0^{v_2}\frac{\partial x(v_1,s)}{\partial v_2}ds] \\
      \Rightarrow \ \frac{\partial x(v_1,v_2)}{\partial v_1} \\
      \shortintertext{Now, the right:} \\
      \frac{\partial x(v_1,v_2)}{\partial v_2}[\frac{\partial x(v_1,v_2)}{\partial v_1}(v_1+v_2-c)] \\
      \Rightarrow \ \frac{\partial^2x(v_1,v_2)}{\partial v_2\partial v_1}(v_1+v_2-c) + \frac{\partial x(v_1,v_2)}{\partial v_1} \\
      \shortintertext{Now, simplify such that:} \\
      0 = \frac{\partial^2x(v_1,v_2)}{\partial v_2\partial v_1}(v_1+v_2-c) \\
    \end{gather*}
    As given in the conditions $v_1+v_2-c\ne0$ $\therefore$ $\frac{\partial^2x(v_1,v_2)}{\partial v_2\partial v_1}=0$. This states that changes in either variable are independent of the other. This can otherwise be shown via separability (as hinted at in lecture) such that $x(v_1,v_2)=f(v_1)+g(v_2)$. 
  \subsection*{Part (D)}
    Because $x(v_1,v_2)$ is separable, 
    \begin{equation}
      x(v_1,v_2) = \begin{cases}
        1& if \ v_1\geq c_1 \ v_2\geq c_2 \\
        0& otherwise
      \end{cases}
    \end{equation}
    This states that the good will provided when an agents value surpasses costs. This finding makes use of the budget balancing from earlier parts.

\end{document}
