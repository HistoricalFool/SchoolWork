\documentclass[10pt,a4paper]{article}
\usepackage[top=3cm,bottom=4cm,left=3.5cm,right=3.5cm]{geometry}
\usepackage{amsmath,amsthm,amsfonts,amssymb,amscd}
\usepackage{fancyhdr,color,comment,graphicx,environ,float,mathtools,mathrsfs}
\newcommand{\norm}[1]{\left\lVert#1\right\rVert}

% Custom headers
\pagestyle{fancy}
\lhead{ECON - 8020}
\chead{}
\rhead{Tate Mason}
\lfoot{}
\cfoot{Assignment 5}
\rfoot{\thepage}

\begin{document}

\title{Assignment 5}
\author{Tate Mason}
\date{Due: April 3rd, 11:59pm}
\maketitle

\section*{Question 1 (10pts)}
  A town is deciding whether or not to build a public project that costs $c = \frac{3}{4}$. There are two residents ($n = 2$). Each agents type (value for the public project) lies in the set $\{0,1\}$. Types are independent across agents and equally likely.

  Suppose you have a mechanism $\langle x, t \rangle$ that is efficient and balanced from the standpoint that total transfers are 0 when the project is not built and $c$ when it is built. In other words, the transfers always cover the cost of the public project.

  \begin{enumerate}
      \item[(a)] What is the efficient allocation/decision rule?
      \item[(b)] Show that if $\langle x, t \rangle$ is incentive compatible then it can not be interim individually rational.
  \end{enumerate}

\section*{Question 2 (25pts)}
  An economy consists of workers with utility functions of the form $\theta l - \frac{l^2}{2} - t$.

  $\theta$ is the worker's marginal product, $l \geq 0$ is her labor supply, $t$ is the tax she pays to the government.

  Suppose there are three types of workers, with marginal products $\theta_1 < \theta_2 < \theta_3$, and that there are large and equal numbers of each type of workers. A worker's type is his private information, but the government can observe his labor supply $l$. The government wishes to use a (nonlinear) tax/subsidy scheme on labor to maximize the median utility level in the population subject to balancing the budget. Agents can not emigrate, which means there are no participation constraints.

   \begin{enumerate}
        \item[(a)] What is a direct mechanism in this context? Why can we restrict attention to direct mechanisms?
        \item[(b)] In any incentive compatible mechanism, what will be the type of the worker with the median utility level? Explain.
        \item[(c)] Write down the government's optimization problem.
        \item[(d)] Show that labor supply is nondecreasing in type under any taxation scheme.
        \item[(e)] Show that once we include the monotonicity of labor supply constraint then all but two incentive constraints become redundant.
        \item[(f)] Assume $\frac{2\theta_1+\theta_3}{3} < \theta_2 < \frac{\theta_1+2\theta_2}{3}$. Solve for the optimal taxation scheme. Explain why the assumption on the $\theta$'s is necessary.
    \end{enumerate}

\section*{Question 3 (20pts)}
  The economics faculty at a university has to allocate courses for next year. Each faculty member must teach one class. Current classes and preferences are:
  
  \begin{center}
    \begin{tabular}{|l|l|l|}
      \hline
      \textbf{Faculty} & \textbf{Current Class} & \textbf{Preferences} \\
      \hline
      Levin & Market Design (MD) & MD $>$ GT $>$ Macro $>$ Intro \\
      \hline
      Taylor & Intro Econ (Intro) & Macro $>$ Intro $>$ MD $>$ GT \\
      \hline
      Niederle & Game Theory (GT) & Macro $>$ GT $>$ Intro $>$ MD \\
      \hline
      Klenow & Macro (Macro) & MD $>$ GT $>$ Intro $>$ Macro \\
      \hline
    \end{tabular}
  \end{center}

  Klenow proposes random serial dictatorship (RSD) to allocate classes. Under his scheme, a priority order will be chosen randomly and announced, with all orders equally likely. Then at a meeting, faculty will choose classes in order of priority.

  \begin{enumerate}
      \item[(a)] Will the faculty have an incentive to choose truthfully?
      \item[(b)] What is the probability that Levin gets to teach Market Design? Game Theory, Macro? Intro Economics? What are the probabilities for Taylor?
      \item[(c)] Before priorities are announced, Levin and Taylor meet secretly. They discuss choosing their classes strategically and possibly swapping after the meeting. Can you find a deal along these lines that leaves each of them strictly better off in some scenario, and neither worse off in any scenario?
      \item[(d)] Niederle argues that faculty should have property rights over the class they are teaching this year, and proposes Top Trading Cycles as an alternative to RSD. What is the outcome under Top Trading Cycles?
  \end{enumerate}

\section*{Solution 1}
  \subsection*{Part (a)}
    The efficient allocation rule is as follows:
    \begin{gather*}
      x(\theta_1,\theta_2) = \begin{cases}
        1, & if \ \sum\limits_{i=1}^2v_i(\theta_i,\theta_{-i})\geq\frac{3}{4}, \\
        0, & otherwise
      \end{cases}
    \end{gather*}
  \subsection*{Part (b)}
    To be budget balanced in this case implies that 
    \begin{gather*}
      \sum\limits_{i=1}^2 t_i(\theta_i,\theta_{-i}) = \frac{3}{4} = c
    \end{gather*}
    Incentive compatibility implies that misreporting is suboptimal. 
    \begin{gather*}
      \shortintertext{For an agent with $\theta_i = 0$, we can show that:} \\
      0 - \mathbb{E}[t_0(0)] \geq 0 - \mathbb{E}[t_0(1)] \Rightarrow \mathbb{E}[t_0(0)] \leq \mathbb{E}[t_0(1)] \\
      \shortintertext{For an agent with $\theta_i = 1$, we can show that:} \\
      1 - \mathbb{E}[t_1(1)] \geq 0.5 - \mathbb{E}[t_1(0)] \Rightarrow \mathbb{E}[t_1(1)] \leq \mathbb{E}[t_1(0)] + 0.5 \\
    \end{gather*}
    Such that $t_1(\cdot)$ represents the transfer of an agent with value 1 and $t_0(\cdot)$ represents the transfer of an agent with type 0. Now, defining the IR constraint as follows:
    \begin{gather*}
      \shortintertext{For an agent with $\theta_i = 0$, the IR constraint is:} \\
      0 - \mathbb{E}[t_i(0)] \geq 0 \Rightarrow \mathbb{E}[t_i(0)] = 0 \\
      \shortintertext{For an agent with $\theta_i = 1$, the IR constraint is:} \\
      1 - \mathbb{E}[t_i(1)] \geq 0 \Rightarrow \mathbb{E}[t_i(1)] \leq 1 \\
    \end{gather*}
    This states that an individual with value 0 has an expected transfer of 0, and one with a value of 1 has an expected transfer under 1.
    Now, using budget balance as defined above and summing the transfers, we see:
    \begin{gather*}
      \mathbb{E}[t_1(\cdot)+t_2(\cdot)] = 3(\frac{1}{4}\cdot\frac{3}{4}) = \frac{9}{16} \\
      \mathbb{E}[t_1(\cdot) + t_2(\cdot)] = 0.5(\mathbb{E}[t_1(1)] + \mathbb{E}[t_2(1)]) \geq \frac{9}{16} \\
      \mathbb{E}[t_1(\cdot) + t_2(\cdot)] = \mathbb{E}[t_1(1)+t_2(1)] \geq \frac{9}{8} \\
    \end{gather*}
    But, this is a contradiction. The IC and IR constraints state that
    \begin{gather*}
      \mathbb{E}[t_1(1) + t_2(1)] \leq \mathbb{E}[t_1(0) + t_2(0)] + 1 \leq 1
    \end{gather*}
    But, we found that the summation of transfers is $\frac{9}{8}$, which is greater than 1. So, we can conclude that if a mechanism is efficient, budget balanced, and IC, it cannot be IIR.
\section*{Solution 2}
  \subsection*{Part (a)}
    Attention can be restricted to direct mechanisms because there is no incentive to lie about marginal product. 

  \subsection*{Part (b)}
    
\section*{Solution 3}
  \subsection*{Part (a)}
    Yes, they are incentivized to choose truthfully. Because we assume the randomization is "fair", there is no incentive to strategize. This is due to the efficiency and Pareto optimality of an RSD with strict preferences.
  \subsection*{Part (b)}
    To show the probabilities, we will calculate probabilities for Professors Levin and Taylor being in each place, and sum them for each class. Each probability will be multiplied by $\frac{1}{4}$, since there are $4! = 24$ possible orders, each with probability $\frac{1}{24}$. Each professor has a probability of $\frac{6}{24}$ to get each position, thus we multiply by $\frac{1}{4}$. 
    \begin{gather*}
      \shortintertext{Levin:} \\
      \text{MD} \ \frac{1}{4} + \frac{1}{4}\cdot\frac{2}{3} + \frac{1}{4}\cdot\frac{1}{3} + 0 = \frac{1}{2} \\
      \text{Intro} \ \frac{1}{4}\cdot\frac{1}{2} + 0 = \frac{1}{8} \\
      \text{GT} \ 3(\frac{1}{4}\cdot\frac{1}{3}) = \frac{1}{4} \\
      \text{Macro} \ = 0 \\
      \shortintertext{Taylor:} \\
      \text{MD} \ = 0 \\
      \text{Intro} \ 2(\frac{1}{4}\cdot\frac{1}{3}) + 0 = \frac{1}{6} \\
      \text{GT} \  = 0 \\
      \text{Macro} \ \frac{1}{4} + \frac{1}{4}\cdot\frac{2}{3} + \frac{1}{4}\cdot\frac{2}{3} + 0 = \frac{7}{12} \\
    \end{gather*}

    Let's go through explaining each case. Levin will always choose MD if first, is able to choose it if Niederle doesn't go first, and will pick it only in the case in which the order is K-N-T-L, meaning they can teach market design $\frac{12}{24} = \frac{1}{2}$ of the time. Intro is only chosen in the case in which the order is N-T-K-L, T-K-N-L, or T-N-K-L, leading to them choosing it $\frac{1}{8}$ of the time. Game theory is chosen in the cases of Klenow choosing first, T-N-L-K, N-T-L-K, or N-K-T-L, $\frac{1}{4}$ of the time. Macro is either never chosen in the case when Levin has an early pick or unavailable in the case of a later pick. For Taylor, they similarly never choose market design if early or never get to if later in the order. For Intro, they only get it in the cases in which Klenow goes before them, $\frac{1}{6}$ of the time. Macro is their top choice and is chosen if first or when Klenow isn't first or second, $\frac{7}{12}$ of the time. Finally, game theory is either not preferred or unavailable, meaning it is not chosen. 
  \subsection*{Part (c)}
    Yes, collusion for ex-post trading is possible. An example case is as follows: Assume Niederle goes first, taking Macro. Then, Taylor forgoes their own preferences, choosing MD. Assuming Klenow goes next, GT is taken, leaving Intro. This allows Taylor and Levin to trade, granting greater total utility than if they were left unaligned. Taylor gets their second choice no matter what, and Levin gets their top choice, which is normally not feasible. If either professor has the first choice, this does not hold. If the order aligns such that they go sequentially, this does not hold. Also, noticing they have preferences which are very different, their ordering makes this much easier whereas if it was Taylor and Niederle trying to collude, the preference ordering would not be congruent to strategy.
    
  \subsection*{Part (d)}
    By the cycle, Levin will keep market design class. Then we remove them from the chain. After that, the cycle, starting with Niederle, grants Macro to Niederle and Game Theory to Klenow. This closes a loop, leaving Taylor with Intro Econ, the class they already had. This grants professors either their first or second choice for teaching, a seemingly good outcome in terms of "fairness".

\end{document}
