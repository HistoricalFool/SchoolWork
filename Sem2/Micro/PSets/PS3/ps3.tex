\documentclass[10pt,a4paper]{article}
\usepackage[top=3cm,bottom=4cm,left=3.5cm,right=3.5cm]{geometry}
\usepackage{amsmath,amsthm,amsfonts,amssymb,amscd}
\usepackage{fancyhdr,color,comment,graphicx,environ,float,mathtools,mathrsfs}
\newcommand{\norm}[1]{\left\lVert#1\right\rVert}

% Custom headers
\pagestyle{fancy}
\lhead{ECON - 8020}
\chead{}
\rhead{Tate Mason}
\lfoot{}
\cfoot{Assignment 3}
\rfoot{\thepage}

\begin{document}

\title{Assignment 3}
\author{Tate Mason}
\date{Due: February 27th, 11:59pm}
\maketitle

\section*{Question 1: Optimal Auctions (25 pts)}
  In this problem, you will compute the auction that maximizes the auctioneer’s expected revenue.

  There is a seller looking to sell a single object. There are $N$ bidders (indexed by $i \in \{1, \dots, N\}$) for this object. Each bidder $i$ has an i.i.d value $v_i$ drawn from a distribution $F$ with support on $[0,M]$. We assume that the density function $f$ is continuous, where $F(a) = \int_{0}^{a} f(x)dx$. Furthermore, assume that $\frac{f(x)}{1-F(x)}$ is non-decreasing.

  \begin{enumerate}
      \item[(a)] Write down the seller’s optimization problem. Hint: use the revelation principle we discussed in class.
      \item[(b)] Take a given bidder $i$ and the corresponding incentive constraints. Write these constraints as the solution to a maximization problem.
      \item[(c)] Reformulate the constraints using the envelope theorem and derive an expression for $t(v_i)$.
      \item[(d)] Solve the optimization problem for the seller. What is the allocation and transfer rule that maximizes expected revenue?
      \item[(e)] Interpret your answer in (d). What kind of auction is this?
  \end{enumerate}

\section*{Question 2 (15 pts)}
  A seller is selling a single object. There are $N$ bidders. Each bidder $i$ has an i.i.d value $v_i$ drawn from a distribution $F$ with support $[0,M]$.

  Let $OPT(N)$ be the expected revenue from the optimal auction with $N$ bidders. Let $S(N+1)$ denote the expected revenue from a second-price auction with $N+1$ bidders.
  \begin{enumerate}
      \item[(a)] Prove that $S(N+1) \geq OPT(N)$.
      \item[(b)] Interpret the result in (a). What does it mean? What is the takeaway?
  \end{enumerate}

\section*{Question 3: Correlated Values (10 pts)}
  There are two bidders with private values $v_i \in \{1,2\}$. The values are correlated:
  \begin{itemize}
      \item Probability $\frac{1}{3}$: both bidders have a value of 1.
      \item Probability $\frac{1}{3}$: both bidders have a value of 2.
      \item Probability $\frac{1}{6}$: one has 2, the other has 1.
  \end{itemize}
  \begin{enumerate}
      \item[(a)] Suppose the auctioneer runs a second-price auction with random tie-breaking. Is truthful bidding still weakly dominant? Find expected revenue and bidder surplus.
      \item[(b)] Construct an allocation and transfer rule that extracts full surplus while keeping the allocation rule unchanged.
  \end{enumerate}

\section*{Question 4 (25 pts)}
  This question involves the Principal-Agent problem we examined in Lecture.

  The principal (seller) sells a quantity $x \geq 0$ of a good for payment $t$. The cost function is $c(x)$. The agent’s utility is $v(x, \theta) - t$, where $\theta$ is private information.
  \begin{enumerate}
      \item[(a)] Write down the seller’s optimization problem (objective, IC, and IR constraints).
      \item[(b)] Rewrite the constraints using ICFOC and monotonicity.
      \item[(c)] Solve for the optimal mechanism (menu/contract).
      \item[(d)] Under what conditions on $v(x,\theta)$ does marginal markup decrease?
      \item[(e)] Show that constant marginal cost implies quantity discounting.
      \item[(f)] Suppose $v(x, \theta) = \theta \gamma(x)$. Show that a power-law distributed $\theta$ leads to a two-part tariff.
  \end{enumerate}

\section*{Solution 1:}
  \subsection*{(a)}
    \begin{gather*}
      {\max\atop{x(\cdot)}}\mathbb{E}[\sum\limits_{i=1}^N t_i(v)]
      \shortintertext{s.t.} \\
      \shortintertext{I.C.} u_i(v) \geq u_i(\hat{v}) \ \forall i \\
      \shortintertext{I.R.} u_i(v) \geq 0 \ \forall i \\
    \end{gather*}
  \subsection*{(b)}
    \begin{gather*}
      \shortintertext{The constraints can be written as solutions as follows:} \\
      \shortintertext{IC:} \\
      u(\hat{v}) = u(\underline{v}) + \int\limits_{\underline{0}}^{\bar{v}}\frac{\partial v(x(w),w)}{\partial v} dw \\
      \shortintertext{IR:} \\
      t_i(v_i,v_{-i}) v_ix_i(v_i,v_{-i}) - \int\limits_{0}^{v_i}\frac{\partial v(x(w),w)}{\partial v}dw \\
    \end{gather*}
  \subsection*{(c)}
    Using the envelope theorem and the given properties that $x_i(\cdot)$ 
    is non-decreasing in $v_i$ and that $u_i(0) = 0$ given by the IR 
    constraint, we can do the following,
    \begin{gather*}
      t_i(v_i) = v_i(x_i(v_i, v_{-i})) - \int\limits_0^{v_i}x_i(w)dw \\
      \shortintertext{By law of iterated expectation,} \\
      \mathbb{E}[t_i(v_i)] =\int_0^M[v_ix_i(v_i,v_{-i}) - \int_0^{v_i}x_i(w)dw]f(v_i)dv_i \\
      \shortintertext{Now, we will split this into two pieces and go by parts} \\
      \int_0^Mv_ix_i(v_i,v_{-i})f(v_i)dv_i - 
      \int_0^M(\int_0^{v_i}x_i(w)dw)f(v_i)dv_i \\
      \shortintertext{Using the law of integrated integrals on the second term,} \\
      \int_0^Mx_i(w)[\int_{v_i=w}^Mf(v_i)dv_i]dw \\
      = \int_0^Mx_i(w)[1-F(w)]dw \\
      \shortintertext{Now, let's rejoin the terms,} \\
      \mathbb{E}[t_i(v_i)] = \int_0^Mv_ix_i(v_i,v_{-i})f(v_i)dv_i -
      \int_0^Mx_i(w)[1-F(w)]dw \\
      \shortintertext{Next, we should sum over the entire bidder 
      distribution,} \\
      \sum^N_{i=1}\mathbb{E}[t_i(v_i)] = 
      \sum^N_{i=1}\int_0^Mx_i(v_i, v_{-i})[v_i - \frac{1-
      F(v_i)}{f(v_i)}]f(v_i)dv_i \\
      \shortintertext{Using pointwise maximization,} \\
      {\max\atop{x(\cdot)}}\sum^N_{i=1}\int_0^Mx_i(v_i, v_{-i})[v_i - 
      \frac{1-F(v_i)}{f(v_i)}] \\
    \end{gather*}
  \subsection*{(d)}
    \begin{gather*}
      x_i^*(v_i) = v_i - \frac{1-F(v_i)}{f(v_i)} \\
      v^* = \frac{1-F(v_i)}{f(v_i)} \\
    \end{gather*}
    Optimal allocation is as such
    \begin{gather*}
      x^*(v_i) = \begin{cases}
        1, & if v_i\geq v^* \\
        0, & otherwise \\
      \end{cases}
    \end{gather*}
  \subsection*{(e)}
    The auction is a second price auction which also employs a reserve 
    price, the surplus. This is also known as Myerson's Optimal Auction.
\section*{Solution 2:}
  \subsection*{(a)}
    \begin{proof}
      $S(N+1)$ has an expected revenue equal to the second highest bid, $\mathbb{E}[V^{(2)}]$. From problem 1, we found that the optimal allocation was 
      $V^{(1)}\geq v^*$. This states that the highest bid needs to be higher than the reserve price of the auction, the surplus. So, we can represent 
      expected revenue as $\mathbb{E}[\max\{V^{(1)},v^*\}]$. Or it is the highest bid, greater than the reserve price, leading to a second highest bid 
      $V^{(1)} > V^{(2)}$ being paid. Otherwise, $v^*$ is paid. Because adding a bidder will drive up prices, $\mathbb{E}[V^{(2)}] \geq
      \mathbb{E}[\max\{V^{(1)}, v^*\}]$. Thus, by extension, $S(N+1) \geq OPT(N)$. 
    \end{proof}
  \subsection*{(b)}
    As competition increases, so too does expected revenue. Thus, we can say that as $N$ gets larger, it is optimal to hold a pure second price auction. As $N$ gets smaller, the reserve price modification becomes more beneficial. 

\section*{Solution 3:}
  \subsection*{(a)}
    To verify that bidding truthfully is weakly dominant, let's go through the cases in which a bidder would overbid or underbid. First, let's consider the case 
    of an overbidding agent. The agent with value $v_i$ shades up their bid such that $b_i>v_i$. If $b_i>b_{-i}$, the agent gets utility via winning, but 
    because they paid more than their value, they lose more utility than they gain. Thus, this is not ideal. Next, in the case of a bidder under-bidding, if 
    they bid $b_i<v_i$, they do not win the item and were not optimally bidding. Thus, the better option is to bid truthfully, implying it is still a weakly 
    dominant strategy.

    Expected revenue and surplus of an auction in which bidders bid truthfully is as follows:
    \begin{gather*}
      \mathbb{E}[r] = (\frac{1}{3}\times1) + (\frac{1}{3}\times2) +
      (\frac{1}{3}\times1) = \frac{4}{3} \\ 
      \mathbb{E}[s] = (\frac{1}{3}\times0) + (\frac{1}{3}\times0) +
      (\frac{1}{6}\times1) = \frac{1}{6} \\
    \end{gather*}
    Expected revenue is calculated by multiplying the probability of having a certain value by the payment associated with that value. This leads to an expected 
    revenue of $\frac{4}{3}$. Expected surplus is calculated by subtracting the transfer from the value, then multiplying each value by its probability. This 
    leads to an expected surplus of $\frac{1}{6}$.
  \subsection*{(b)}
    Setting an entry price of $\frac{1}{6}$ will maximize surplus and revenue in expectation. This is because, in expectation, the entry fee comes out and thus 
    the expectation remains the same, granting the expected revenue and surplus found above plus an entry fee of the expected surplus, thus incentivizing agents 
    to participate still.
\section*{Solution 4:}
  \subsection*{(a)}
    \begin{gather*}
      {\max\atop{x(\cdot),t(\cdot)}}\mathbb{E}[t(\theta) - c(x(\theta))] \\
      \shortintertext{s.t.} \\
      v(x(\theta),\theta) - t(\theta) \geq v(x(\hat{\theta}),\theta) - t(\hat{\theta}) \ \forall \ \theta,\hat{\theta}\in[\underline{\theta},\bar{\theta}] \\
      v(x(\theta),\theta) - t(\theta)\geq 0 \forall \ \forall \ \theta\in[\underline{\theta},\bar{\theta}] \\
    \end{gather*}
  \subsection*{(b)}
    \begin{gather*}
      \Phi(\hat{\theta},\theta) = v(x(\hat{\theta}),\theta) - t(\hat{\theta}) \\
      U(\theta) = \Phi(\theta,\theta) \\
      \shortintertext{$\Phi$ has the property of being differentiable almost everywhere in $\theta$. $U(\theta)$ is differentiable almost everywhere} \\
      U'(\hat{\theta}) = \frac{\partial \Phi(\hat{\theta},\hat{\theta})}{\partial\theta} = \frac{\partial v(x(\hat{\theta}),\hat{\theta})}{\partial\theta} \\
      U(\hat{\theta}) = U(\underline(\theta)) + \int\limits_{\underline{\theta}}^{\bar{\theta}} = v(x(\theta),\theta) - 
      \int\limits_{\underline{\theta}}^{\bar{\theta}}v_{\theta}(x(w),w)dw \\
      \shortintertext{So, the seller's maximization problem can be 
      expressed as follows:} \\
      \max\int\limits_0^{\bar{\theta}}[t(\theta) - 
      c(x(\theta))]f(\theta)d\theta \\
      \shortintertext{s.t.} \\
      t(\theta) = U(\underline{\theta}) - 
      \int\limits_{\underline{\theta}}^{\bar{\theta}}x(w)dw \\
      U(\underline{\theta}) = 0 \\
      x(\cdot) \text{is non-decreasing}
    \end{gather*}
  \subsection*{(c)}
    \begin{gather*}
      \max\int\limits_0^{\bar{\theta}}[t(\theta) - 
      c(x(\theta))]f(\theta)d\theta \\
      \shortintertext{s.t.} \\
      t(\theta) = U(\underline{\theta}) - 
      \int\limits_{\underline{\theta}}^{\bar{\theta}}x(w)dw \\
      U(\underline{\theta}) = 0 \\
      x(\cdot) \text{is non-decreasing} \\
    \end{gather*}
    Dropping the monotonicity constraint, we can set up a relaxed form of the problem s.t.
    \begin{gather*}
      v(x,\theta) - t(\theta) = 
      \int\limits_{\underline{\theta}}^{\theta}\frac{\partial v(x(w),
      w)}{\partial\theta}dw \\
      t(\theta) = v(x,\theta) -
      \int\limits_{\underline{\theta}}^{\theta}\frac{\partial v(x(w),
      w)}{\partial\theta}dw \\
    \end{gather*}
    Now, we can restate the maximization as follows
    \begin{gather*}
      \max{\atop{x(),t()}}\int\limits_0^M[v(x,\theta) - 
      \int\limits_0^{\theta}\frac{\partial v(x(w),
      w)}{\partial\theta}dw - c(x(\theta))] f(\theta)d\theta \\
      \shortintertext{Rearranging and using integration by parts:} \\
      \int\limits_0^M[v(x,\theta) - c(x(\theta))]f(\theta)d\theta - \int
      \limits_0^Mf(\theta)\int\limits_0^{\theta}\frac{\partial v(x(w),w)}
      {\partial\theta}dwd\theta \\
      F(\theta)\int\limits_0^M\frac{\partial v(x(w), w)}{\partial\theta}d\theta -\int\limits_0^M f(\theta)\frac{\partial v(x(w),w)}{\partial\theta}dwd\theta \\
      \shortintertext{Now, merging back into a single integral,} \\
      {\max\atop{x}}\int\limits_0^M[v(x(\theta),\theta)-c(x(\theta)) - 
      \frac{1-F(\theta)}{f(\theta)}\frac{\partial v(x(w),w)}{\partial\theta}]f(\theta)d\theta \\
      \shortintertext{Now, maximizing the integral, we can find that} \\
      c'_x(x(\theta)) = v'(x(\theta),\theta) - \frac{1- 
      F(\theta)}{f(\theta)}v''_{\theta x}(x(\theta),\theta)
    \end{gather*}
  \subsection*{(d)}
    The two conditions to satisfy a decreasing marginal markup are such that $v(x(\theta),\theta)$ should be non-decreasing and concave. 
    \begin{proof}
      $t'(x) - c'(x) = v'_{\theta}(x(\theta),\theta) + \frac{1-F(\theta)}{f(\theta)}v_{\theta x}''(x(\theta),\theta)$. For the function to be decreasing it must be the case that $v'_{\theta}(x(\theta),\theta)\geq0$ and $v''_{\theta x}(x(\theta),\theta)\leq0$. These are true when $v(x(\theta),\theta)$ is concave and non-decreasing, thus ensuring a decreasing marginal markup. 
    \end{proof}
  \subsection*{(e)} 
    \begin{gather*}
      t'(x) - c'(x) \ \text{is decreasing in x} \\
      t'(x) - \bar{c} \ \text{is also decreasing in x} \\
      \Rightarrow t'(x) \ \text{must be decreasing in x} \\
    \end{gather*}
    This implies that as $x$ increases, marginal trasnfer per unit decreases, showing quantity discounting. 
  \subsection*{(f)}
    As found in previous sections, when $v$ is concave and non-decreasing,
    \begin{gather*}
      t'(x)-c'(x) = v_{\theta}'(x(\theta),\theta) - \frac{1-
      F(\theta)}{f(\theta)}v''_{\theta x}(x(\theta),\theta) \\
      \therefore \ t'(x) - \bar{c} = \gamma(x) - \frac{1 - (1-\theta^{-
      \delta})}{\delta\theta^{-\delta-1}}\gamma'(x) \\
      t'(x) = \gamma(x) - \frac{1}{\delta}\gamma'(x) + \bar{c} \\
    \end{gather*}
    Thus, we can see that transfers hold both a component that varies with $x$, and a term which linearly increases in $x$. This is a definition of a two-part tariff.
\end{document}
