\documentclass[10pt,a4paper]{article}
\usepackage[top=3cm,bottom=4cm,left=3.5cm,right=3.5cm]{geometry}
\usepackage{amsmath,amsthm,amsfonts,amssymb,amscd}
\usepackage{fancyhdr,color,comment,graphicx,environ,float,mathtools,mathrsfs}
\newcommand{\norm}[1]{\left\lVert#1\right\rVert}

% Custom headers
\pagestyle{fancy}
\lhead{ECON - 8020}
\chead{}
\rhead{Tate Mason}
\lfoot{}
\cfoot{Assignment 2}
\rfoot{\thepage}

\begin{document}

\title{Assignment 2}
\author{Tate Mason}
\date{Due: February 6th, 11:59pm}
\maketitle

\section*{Instructions}
\begin{itemize}
    \item Show all your work and circle your final answers.
    \item Submit as a single document.
\end{itemize}

\section*{Question 1 (10 points)}
There is an island with only two consumers, Tom and Christina. There are two goods, apples ($x$) and bananas ($y$), available on the island. The consumers' utility function over bundles is given by:
\begin{align*}
    \text{Tom: } u(x, y) &= x^\alpha y^{1-\alpha}, \\
    \text{Christina: } u(x, y) &= x^\beta y^{1-\beta}.
\end{align*}
Tom has an endowment of $\omega_{A,T} > 0$ apples and $\omega_{B,T} > 0$ bananas, while Christina has an endowment of $\omega_{A,C} > 0$ apples and $\omega_{B,C} > 0$ bananas. Without loss of generality, the price of bananas is normalized to 1.
\begin{enumerate}
    \item[(a)] What is the Walrasian equilibrium of this exchange economy?
    \item[(b)] In equilibrium, what share of wealth does Tom spend on apples? What share of wealth does Christina spend on apples?
\end{enumerate}

\section*{Question 2 (10 points)}
There is an island with only two consumers, Bob and Alice. There are two goods, apples ($x$) and bananas ($y$), available on the island. The consumers' utility function over bundles is given by:
\begin{align*}
    \text{Bob: } u(x, y) &= x + y, \\
    \text{Alice: } u(x, y) &= \min \{x, y\}.
\end{align*}
Bob's endowment is 1 apple and 0 bananas. Alice's endowment is 2 apples and 1 banana.
\begin{enumerate}
    \item Show mathematically that no Walrasian equilibrium exists in this economy.
\end{enumerate}

\section*{Question 3 (15 points)}
\begin{enumerate}
    \item[(a)] Prove the First Welfare Theorem.
    \item[(b)] Prove an equivalent characterization of Pareto-efficient allocations. Given any set of weights $\mu_1, \dots, \mu_N \geq 0$ such that $\sum \mu_i = 1$, consider the solution $x^*$ to the following problem:
    \begin{align*}
        \max_{x_1, \dots, x_N} \quad \sum_{i=1}^N \mu_i u(x^i) \quad \text{subject to} \quad \sum_{i=1}^N x_i^j \leq \sum_{i=1}^N e_i^j \quad \forall j \in \{1, \dots, M\}.
    \end{align*}
    Prove that any solution $x^*$ is a Pareto-efficient allocation.
    \item[(c)] Provide an interpretation, in words, of what you showed in (b).
\end{enumerate}

\section*{Question 4 (20 points)}
Consider a second-price auction with $N$ bidders. Each bidder has an independent private value $v_i$ drawn from a Uniform Distribution on $[0, 1]$.
\begin{enumerate}
    \item[(a)] What is the expected revenue generated when all bidders bid truthfully? Provide a closed-form solution and show your work.
    \item[(b)] Prove that all bidders bidding truthfully is an equilibrium of the second-price auction game. What is another equilibrium? Prove your example is an equilibrium.
    \item[(c)] If the auctioneer sets a reserve price $r$, is it still a weakly dominant strategy to bid truthfully?
    \item[(d)] Suppose $N = 3$. What is the optimal reserve price $r$ the auctioneer should set? How much smaller/greater is the revenue compared to the auction with no reserve price?
    \item[(e)] A student at another university thought revenue equivalence implied that the expected revenue should be the same. Explain why their reasoning is incorrect.
\end{enumerate}

\section*{Question 5 (15 points)}
Suppose there are $N$ bidders competing for a single object in an all-pay auction. Each bidder has an i.i.d. value $v_i$ for the object drawn from some continuous distribution $F$ with support $[0, M]$.
\begin{enumerate}
    \item[(a)] Show that there is a symmetric equilibrium in increasing strategies.
    \item[(b)] What is the expected revenue generated by this auction in the equilibrium from (a)? Explain your answer.
\end{enumerate}
\section*{Solution 1:}
  \subsection*{(a)}
    Given prices $\{p_x, 1\}$ and allocations $\{x,y,\omega_{A,T}, \omega_{A,C}, \omega_{B, T}, \omega_{B,C}\}$, 
    Tom solves:
    \begin{gather*}
        u_T = x^{\alpha}y^{1-\alpha} \\
        \text{s.t.}
        p_xx + y = p_x\omega_{A,T} + \omega_{B,T}
    \end{gather*}
    and Cristina solves:
    \begin{gather*}
        u_C = x^{\beta}y^{1-\beta} \\
        \text{s.t.}
        p_xx + y = p_x\omega_{A,C} + \omega_{B,C}
    \end{gather*}
  \subsection*{(b)}
    
\section*{Solution 2:}
\section*{Solution 3:}
  \subsection*{(a)}
    \begin{proof}
      Assume there is an allocation $x$ which solve 
    \end{proof}
  \subsection*{(b)}
  \subsection*{(c)}
\section*{Solution 4:}
  \subsection*{(a)}
  \subsection*{(b)}
  \subsection*{(c)}
  \subsection*{(d)}
  \subsection*{(e)}
\section*{Solution 5:}
  \subsection*{(a)}
  \subsection*{(b)}
\end{document}
