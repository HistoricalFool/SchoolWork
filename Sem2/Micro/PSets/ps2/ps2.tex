\documentclass[10pt,a4paper]{article}
\usepackage[top=3cm,bottom=4cm,left=3.5cm,right=3.5cm]{geometry}
\usepackage{amsmath,amsthm,amsfonts,amssymb,amscd}
\usepackage{fancyhdr,color,comment,graphicx,environ,float,mathtools,mathrsfs}
\newcommand{\norm}[1]{\left\lVert#1\right\rVert}

% Custom headers
\pagestyle{fancy}
\lhead{ECON - 8020}
\chead{}
\rhead{Tate Mason}
\lfoot{}
\cfoot{Assignment 2}
\rfoot{\thepage}

\begin{document}

\title{Assignment 2}
\author{Tate Mason}
\date{Due: February 6th, 11:59pm}
\maketitle

\section*{Instructions}
\begin{itemize}
    \item Show all your work and circle your final answers.
    \item Submit as a single document.
\end{itemize}

\section*{Question 1 (10 points)}
There is an island with only two consumers, Tom and Christina. There are two goods, apples ($x$) and bananas ($y$), available on the island. The consumers' utility function over bundles is given by:
\begin{align*}
    \text{Tom: } u(x, y) &= x^\alpha y^{1-\alpha}, \\
    \text{Christina: } u(x, y) &= x^\beta y^{1-\beta}.
\end{align*}
Tom has an endowment of $\omega_{A,T} > 0$ apples and $\omega_{B,T} > 0$ bananas, while Christina has an endowment of $\omega_{A,C} > 0$ apples and $\omega_{B,C} > 0$ bananas. Without loss of generality, the price of bananas is normalized to 1.
\begin{enumerate}
    \item[(a)] What is the Walrasian equilibrium of this exchange economy?
    \item[(b)] In equilibrium, what share of wealth does Tom spend on apples? What share of wealth does Christina spend on apples?
\end{enumerate}

\section*{Question 2 (10 points)}
There is an island with only two consumers, Bob and Alice. There are two goods, apples ($x$) and bananas ($y$), available on the island. The consumers' utility function over bundles is given by:
\begin{align*}
    \text{Bob: } u(x, y) &= x + y, \\
    \text{Alice: } u(x, y) &= \min \{x, y\}.
\end{align*}
Bob's endowment is 1 apple and 0 bananas. Alice's endowment is 2 apples and 1 banana.
\begin{enumerate}
    \item Show mathematically that no Walrasian equilibrium exists in this economy.
\end{enumerate}

\section*{Question 3 (15 points)}
\begin{enumerate}
    \item[(a)] Prove the First Welfare Theorem.
    \item[(b)] Prove an equivalent characterization of Pareto-efficient allocations. Given any set of weights $\mu_1, \dots, \mu_N \geq 0$ such that $\sum \mu_i = 1$, consider the solution $x^*$ to the following problem:
    \begin{align*}
        \max_{x_1, \dots, x_N} \quad \sum_{i=1}^N \mu_i u(x^i) \quad \text{subject to} \quad \sum_{i=1}^N x_i^j \leq \sum_{i=1}^N e_i^j \quad \forall j \in \{1, \dots, M\}.
    \end{align*}
    Prove that any solution $x^*$ is a Pareto-efficient allocation.
    \item[(c)] Provide an interpretation, in words, of what you showed in (b).
\end{enumerate}

\section*{Question 4 (20 points)}
Consider a second-price auction with $N$ bidders. Each bidder has an independent private value $v_i$ drawn from a Uniform Distribution on $[0, 1]$.
\begin{enumerate}
    \item[(a)] What is the expected revenue generated when all bidders bid truthfully? Provide a closed-form solution and show your work.
    \item[(b)] Prove that all bidders bidding truthfully is an equilibrium of the second-price auction game. What is another equilibrium? Prove your example is an equilibrium.
    \item[(c)] If the auctioneer sets a reserve price $r$, is it still a weakly dominant strategy to bid truthfully?
    \item[(d)] Suppose $N = 3$. What is the optimal reserve price $r$ the auctioneer should set? How much smaller/greater is the revenue compared to the auction with no reserve price?
    \item[(e)] A student at another university thought revenue equivalence implied that the expected revenue should be the same. Explain why their reasoning is incorrect.
\end{enumerate}

\section*{Question 5 (15 points)}
Suppose there are $N$ bidders competing for a single object in an all-pay auction. Each bidder has an i.i.d. value $v_i$ for the object drawn from some continuous distribution $F$ with support $[0, M]$.
\begin{enumerate}
    \item[(a)] Show that there is a symmetric equilibrium in increasing strategies.
    \item[(b)] What is the expected revenue generated by this auction in the equilibrium from (a)? Explain your answer.
\end{enumerate}
\section*{Solution 1:}
  \subsection*{(a)}
    Given prices $\{p_x, 1\}$ and allocations $\{x,y,\omega_{A,T}, \omega_{A,C}, \omega_{B, T}, \omega_{B,C}\}$, 
    Tom solves:
    \begin{gather*}
        u_T = x^{\alpha}y^{1-\alpha} \\
        \text{s.t.}
        p_xx + y = E_t \\
        E_t = p_x\omega_{A,T} + \omega_{B,T}
    \end{gather*}
    and Cristina solves:
    \begin{gather*}
        u_C = x^{\beta}y^{1-\beta} \\
        \shortintertext{s.t.}
        p_xx + y = E_c \\
        p_x\omega_{A,C} + \omega_{B,C}
    \end{gather*}
    Tom's Lagrangian and FOC's will be as follows:
    \begin{gather*}
      \mathcal{L}_t = x^{\alpha}y^{1-\alpha} + \lambda(E_t-p_xx - y) \\
      \mathcal{L}_x = \alpha\frac{x^{\alpha-1}y^{1-\alpha}}{p_x} = \lambda \\
      \mathcal{L}_y = (1-\alpha)x^{\alpha}y^{-\alpha} = \lambda \\
      \therefore \ \frac{\alpha x^{\alpha-1}y^{1-\alpha}}{p_x} = (1-\alpha)x^{\alpha}y^{1-\alpha} \\
      \frac{\alpha x^{\alpha-1}y}{p_x} = (1-\alpha)x^{\alpha} \\
      \frac{\alpha y}{p_x} = (1-\alpha)x \\
      y = \frac{(1-\alpha)}{\alpha}p_xx \\
      \shortintertext{Substituting into BC:} \\
      p_xx + \frac{(1-\alpha)}{\alpha}p_xx = E \rightarrow \boxed{x^*_T = \alpha\frac{E_t}{p_x}} \\
      \shortintertext{Plugging into BC:} \\
      y_T^* = E_t - p_xx \rightarrow \boxed{E_t(1-\alpha) = y_T^*}
    \end{gather*}
    Christina's Lagrangian and FOC's will follow the same path, thus I will skip redundant math and give the answers,
    \begin{gather*}
      \mathcal{L}_c = x^{\beta}y^{1-\beta} + \lambda(E_c - p_xx - y) \\
      \shortintertext{Following same steps as above yields...} \\
      \boxed{x^*_c = \beta\frac{E_c}{p_x}, \ y^*_c = E_c(1-\beta)} \\
    \end{gather*}
    Finally, equilibrium prices can be solved via the following steps:
    \begin{gather*}
      x^*_t + x^*_c = \omega_{A,T} + \omega_{A,C} \\
      \frac{E_i}{p_x}(\alpha+\beta) = \omega_{A,T} + \omega_{A,C} \\
      E_i = \omega_{A,i} + \omega_{B,i} \\
      p_x(\omega_{A,T} + \omega_{A,C}) = \alpha(p_x\omega_{A,T} + \omega_{B,T}) + \beta(p_x\omega_{A,C} + \omega_{B,C}) \\
      p_x(\alpha\omega_{A,T} + \beta\omega_{A,C}) + (\alpha\omega_{B,T} + \beta\omega_{B,C}) = p_x(\omega_{A,T}+\omega_{A,C}) \\
      (\alpha\omega_{B,T} + \beta\omega_{B,C}) = p_x[\omega_{A,T}(1-\alpha) + \omega_{A,C}(1-\beta)] \\
      \boxed{p_x^* = \frac{\alpha\omega_{B,T} + \beta\omega_{B,C}}{(1-\alpha)\omega_{A,T} + (1-\beta)\omega_{A,C}}} \\
    \end{gather*}
  \subsection*{(b)}
    Given the preferences are Cobb-Douglas, Tom will spend share $\alpha$ on apples and Christina will spend share $\beta$ on apples.
\section*{Solution 2:}
  \begin{proof}
    There are two consumers, Bob and Alice, and two goods $(x,y)$. Bob's utility is $u_B = x+y$ and Alice's is $u_A = \min\{x,y\}$. They also have endowments $E_B = (1,0)$ and $E_A = (2,1)$, respectively. Assume price of good $x$ is $p_x$ and price of good $y$ is normalized to $p_y=1$. Then, given prices $\{p_x,1\}$ and allocations $\{x,y\}$, both consumers will maximize utility and clear the market. With these conditions laid out, Bob's optimization problem is 
    \begin{gather*}
      u_B = x+y \\
      \shortintertext{s.t.} \\
      p_xx + y = p_x \\
    \end{gather*} Bob's utility function shows perfect substitutes and thus he is indifferent to the good and will base decisions on price. Therefore, if $p_x<1$, he will spend his total wealth on good $x$; if $p_x>1$, he will spend all of his wealth on good $y$; if $p_x = 1$, Bob is fully indifferent. Alice's optimization problem is as follows:
    \begin{gather*}
      u_A = \min\{x,y\} \\
      \shortintertext{s.t.} \\
      p_xx + y = 2p_x + 1 \\
      \shortintertext{but because she displays Leontief preferences, $x^* = y^*$, implying:} \\
      p_xx^* + x^* = 2p_x+1 \\
      \shortintertext{We can rearrange Alice's constraint such that:} \\
      x^*(p_x+1) = 2p_x+1 \rightarrow x^* = \frac{2p_x + 1}{p_x + 1} \\
    \end{gather*} 
    For market clearing, we know that it must be such that $x_A^* + x_B^* = 3$ and $y_A^* + y_B^* = 1$. Now, let us consider Bob's three price cases from above. First, when $p_x<1$, $x^*_B = 1$ and thus $\frac{2p_x + 1}{p_x+1} = 2$. This yields the following $2p_x + 1 = 2p_x + 2$. Subtracting the term $2p_x$ from both sides yields $1=2$, a contradiction. Thus, the case when $p<1$ is not an equilibrium. Now, the second case is when $p_x>1$. In this case, Bob spends nothing on $x$, meaning $y^*_B = p_x$ from Bob's budget constraint. Thus, market clears with $p_x + \frac{2p_x + 1}{p_x + 1} = 1$. Doing some algebra will yield $p_x^2 + 2p_x + p_x + 1 = p_x + 1$. After more rearranging, we get that $p_x^* = -2$, another contradiction. Finally, in the third case, $p_x = 1$, Bob does not care at all. So, $\frac{2+1}{1+1} = 1$. $\frac{3}{2}\ne1$, a contradiction. Therefore, there is no Walrasian equilibrium in this case. 
  \end{proof}
\section*{Solution 3:}
  \subsection*{(a)}
    \begin{proof}
      Assume there is a Walrasian equilibrium given by prices $\{p_x,p_y\}$ and allocations $\{x,y\}$. Assume the optimal bundle in this equilibrium is $(x^*,y^*)$ which clears the market such that $p_xx^* + p_yy^* = pE$. However, there is a bundle $(\bar{x},\bar{y})$ such that $u(\bar{x})>u(x^*)$ and $u(\bar{y}) > u(y^*)$. This can be further interpreted as $p_x\bar{x} > p_xx^*$ and $p_y\bar{y} > p_yy^*$. However, if this is the case, $p_x\bar{x}+p_y\bar{y}>pE$. This is infeasible, and a contradiction. Thus, the equilibrium is maximizing feasible bundles for agent utility. 
    \end{proof}
  \subsection*{(b)}
    \begin{proof}
      Assume $x^*$ is not a Pareto-efficient allocation, let's now assume there exists a $u_i(x')\geq u_i(x^*))$ for all agents $i$ and $u_j(x')>u_j(x^*)$ for at least one agent $j$. Then, it follows that $\sum\limits_{i=1}^N\mu_iu_i(x')\geq\sum\limits_{i=1}^N\mu_iu_i(x^*)$. But, by the same token, it must also be that $\sum\limits_{i=1}^N\mu_iu_i(x')>\sum\limits_{i=1}^N\mu_iu_i(x^*)$. However, if this is the case, then in equilibrium, $\sum\limits_{i=1}^Nx'_i>\sum\limits_{i=1}^Ne_i^j$. This is a contradiction, and an infeasible allocation. Thus, $x^*$ is the Pareto-efficient allocation. 
    \end{proof}
  \subsection*{(c)}
    In part b, we assume that there is an $x'$  which is at least as good as $x^*$ for at least all but 1 agent for whom $x'$ is strictly better than $x^*$. If this is the case, the summation $\sum\limits_{i=1}^N\mu_iu(x')>\sum\limits_{i=1}^N\mu_iu(x^*)$. For this to be the case, it is further true that $x'>e$. This is a contradiction since in equilibrium endowments and allocations will equalize. Thus, this allocation is not Pareto-efficient because it is not feasible.
  \section*{Solution 4:}
  \subsection*{(a)}
    The expected revenue when bidders bid truthfully is the second highest value. This can be represented by deriving the following:
    \begin{gather*}
      f^{(k)}(v) = \frac{N!}{(k-1)!(N-k)!} v^{k-1}(1-v)^{N-k} \\
      \mathbb{E}(v^{(k)}) = \int\limits_0^1 vf^{k}(v)dv \\
      \mathbb{E}(V^{(k)}) = \frac{N!}{(k-1)!(N-k)!} \int\limits_0^1 vv^{k-1}(1-v)^{N-k}dv
      \shortintertext{The integral can be represented as a Beta function such that:} \\
      \textbf{B}(k+1,N-k+1) = \int\limits_0^1 vv^{k-1}(1-v)^{N-k}dv \\
      \shortintertext{This can then be represented as a Gamma function such that:} \\
      \textbf{B}(k+1, N-k+1) = \frac{\Gamma(k+1)\Gamma(N-k+1)}{\Gamma(N+2)} \\
      \shortintertext{By the property $\Gamma(n) = (n-1)!$ we can simplify to} \\
      \textbf{B}(k+1,N-k+1) = \frac{k!(N-k)!}{(N+2)!} \\
      \shortintertext{Plugging this back into the Expected value equation, we get:} \\
      \mathbb{E}[V^{(k)}] = \frac{N!}{(k-1)!(N-k)!} \frac{k!(N-k)!}{(N+1)!} \\
      \shortintertext{After some simplification, we get that:}
      \mathbb{E}[V^{(k)}] = \frac{k}{N+1} \\
      \shortintertext{For the second highest order, $k=N-1$} \\
      \boxed{\mathbb{E}[R] = \frac{N-1}{N+1}} \\
    \end{gather*}
    This is the expected revenue of the second highest bid.
  \subsection*{(b)}
    \begin{proof}
      Consider a second price auction $A$ with $N$ bidders. Each bidder has privately held, independent value $v_i$ which are drawn from a distribution $F$ on $[0,1]$. Consider some bidder $i$ with value $v_i$. Let's assume their strategy is to bid truthfully such that $b_i=v_i$. When employing this strategy, they will win \iff $b_i > \max\{b_j\}$. If they win, they receive a payoff $P = v_i - b^{(2)}$. Thus, they will win the difference between their value and the second highest bid. Let's consider an alternative strategy in which the bidder decides to shade their bid up. In this case $b_i>v_i$ and if they win, their payoff will be $P = v_i - b^{(2)} < 0$ or a negative payoff. This is avoided with a truthful bid. Now, let's consider the case in which the bidder would shade their bid down such that $v_i<b_i$. In this case, the bidder doesn't win if the bid $b^{(2)}\in[b_i, v_i]$. In this case, the bidder has a higher likelihood of winning by bidding truthfully. Because all bidders are held to the same conditions, it is weakly dominant to bid truthfully.
    \end{proof}
    For one case of an alternative equilibrium, let's evaluate the case in which all $b_i = c$ in which $c$ is an arbitrary constant greater than $0$. Let's consider the case in which a bidder $i$ would look to deviate, bidding $b_i>c$. While they may win, they still pay $c$, thus creating the same utility as just bidding $b_i=c$. Lowering the bid such that $b_i<c$ would lead to a payoff of $0$, guaranteeing a loss due to all other players bidding $b_i=c$. These findings lead to a conclusion that in this case there is no incentive to deviate, thus implying $b_i=c$ is an equilibrium. 
  \subsection*{(c)}
    Yes. In second price auctions which are increasing and symmetric, bidding truthfully is a weakly dominant strategy no matter the reserve price.
  \subsection*{(d)}
    The optimal reserve price is determined by finding the price which maximizes $\mathbb{E}[Revenue]$ for the auction. Given $N=3$, and the uniform distribution, we can say that 
    \begin{gather*}
      \mathbb{E}(\pi)=\mathbb{P}[b_1, b_2, b_3 < r]\times\mathbb{E}[\pi|b_1,b_2,b_3<r] + \mathbb{P}[One \ b_i > r]\mathbb{E}[\pi|One \ b_i > r] + \\ \mathbb{P}[Two \ b_i > r]\mathbb{E}[\pi|Two \ b_i > r] + \mathbb{P}[Three \ b_i > r]\mathbb{E}[\pi|Three \ b_i > r] \\
      \mathbb{E}[\pi] = [r^3|0 + [3r^2(1-r)](r) + [3r(1-r)^2](r+(1-r)\frac{1}{3})+[(1-r)^3](r+(1-r)\frac{1}{2})
      \mathbb{E}[\pi] = \frac{1}{2} + r^3 + - \frac{3}{4}r^4 \\
      \shortintertext{Then, we take the derivative to find the optimal reserve price $r^*$} \\
      \frac{\partial \mathbb{E}[\pi]}{\partial r} = 3r^2 - 6x^3 = 0 \\
        3r^2(1-2r) = 0 \\
        r = 0; \ r = \frac{1}{2} \\
    \end{gather*}
    This yields two possible values for $r^*$. In the case in whcih $r^*=0$, would be the same as not setting a reserve price, allowing for revenue to be determined by expected revenue. Similarly, when examining $r^* = \frac{1}{2}$, we see that $\mathbb{E}$ is equal to the expected revenue without a reserve price. This leads to the intuition that the expected revenue is unimpacted by the reserve price, stating that
    \begin{gather*}
      \mathbb{E}[\pi|r^*=0] = \mathbb{E}[\pi|r^*=\frac{1}{2}] = \frac{1}{2}
    \end{gather*}
  \subsection*{(e)}
    This is incorrect because the student does not take into consideration the effects of reserve prices on expected revenues. Further, they fail to understand that the revenue equivalence theorem only holds under certain criteria. These criteria are that bidders values are i.i.d from the same distribution, the bidders are risk neutral, and the auction is symmetric and efficient. Reserve prices can make expected revenue change, thus proving a contradiction to the claim that they are the same.
\section*{Solution 5:}
  \subsection*{(a)}
    Consider an auction $A$ in which there are $N$ bidders, who each hold and i.i.d $v_i$, drawn from distribution $F\sim[0,1]$. The allocation rule for this auction is as such 
    \begin{gather*}
      x_i(b_i) = \begin{cases}
        1 & \text{if $b_i > \max_k b_k$} \\
        0 & \text{otherwise}
      \end{cases}
    \end{gather*}
    The transfer/payment rule is as follows for an all pay auction:
    \begin{gather*}
      t_i(b_i) = b_i
    \end{gather*}
    The strategy for a bidder $i$ is 
    \begin{gather*}
      \beta_i(v_i) = \max\mathbb{E}_{v_{-i}}[v_ix_i(b_i) - t_i(b_i)|\beta_{-i}]
    \end{gather*}
    Then, we can represent the payoff function with variable $z_i$ which is the outward facing value of the bidder $i$.
    \begin{gather*}
      \Pi_i(z_i, v_i) = G(z_i)v_i - m(z_i) \\
      \shortintertext{s.t.} \\
      G(z_i) = Pr(\max v_{-i}<z_i) \\
      m(z_i) = \mathbb{E}_{v_{-i}}(t_i|z_i)
    \end{gather*}
    The, we will differentiate with respect to $G$, thus getting us
    \begin{gather*}
      g(z_i)v_i - m'(z_i) = 0 \\
      m'(z_i) = m(0) \int\limits_0^{v_i}g(x)xdx \\
      m(z_i) = \int\limits_0^{v_i}g(x)xdx \\
    \end{gather*}
    This states that bidding up to your true value is optimal as everyone is going to follow this strategy. Because of this, no one will bid above their value. Therefore, bidders will choose to either bid nothing, or bid their true value.
  \subsection*{(b)}
    By revenue equivalence, the expected revenue of this all pay auction is the second highest bid or $\mathbb{E}[R] = \frac{N-1}{N+1}$
\end{document}
