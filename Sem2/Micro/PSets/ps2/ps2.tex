\documentclass[10pt,a4paper]{article}
\usepackage[top=3cm,bottom=4cm,left=3.5cm,right=3.5cm]{geometry}
\usepackage{amsmath,amsthm,amsfonts,amssymb,amscd}
\usepackage{fancyhdr,color,comment,graphicx,environ,float,mathtools,mathrsfs}
\newcommand{\norm}[1]{\left\lVert#1\right\rVert}

% Custom headers
\pagestyle{fancy}
\lhead{ECON - 8020}
\chead{}
\rhead{Tate Mason}
\lfoot{}
\cfoot{Assignment 2}
\rfoot{\thepage}

\begin{document}

\title{Assignment 2}
\author{Tate Mason}
\date{Due: February 6th, 11:59pm}
\maketitle

\section*{Instructions}
\begin{itemize}
    \item Show all your work and circle your final answers.
    \item Submit as a single document.
\end{itemize}

\section*{Question 1 (10 points)}
There is an island with only two consumers, Tom and Christina. There are two goods, apples ($x$) and bananas ($y$), available on the island. The consumers' utility function over bundles is given by:
\begin{align*}
    \text{Tom: } u(x, y) &= x^\alpha y^{1-\alpha}, \\
    \text{Christina: } u(x, y) &= x^\beta y^{1-\beta}.
\end{align*}
Tom has an endowment of $\omega_{A,T} > 0$ apples and $\omega_{B,T} > 0$ bananas, while Christina has an endowment of $\omega_{A,C} > 0$ apples and $\omega_{B,C} > 0$ bananas. Without loss of generality, the price of bananas is normalized to 1.
\begin{enumerate}
    \item[(a)] What is the Walrasian equilibrium of this exchange economy?
    \item[(b)] In equilibrium, what share of wealth does Tom spend on apples? What share of wealth does Christina spend on apples?
\end{enumerate}

\section*{Question 2 (10 points)}
There is an island with only two consumers, Bob and Alice. There are two goods, apples ($x$) and bananas ($y$), available on the island. The consumers' utility function over bundles is given by:
\begin{align*}
    \text{Bob: } u(x, y) &= x + y, \\
    \text{Alice: } u(x, y) &= \min \{x, y\}.
\end{align*}
Bob's endowment is 1 apple and 0 bananas. Alice's endowment is 2 apples and 1 banana.
\begin{enumerate}
    \item Show mathematically that no Walrasian equilibrium exists in this economy.
\end{enumerate}

\section*{Question 3 (15 points)}
\begin{enumerate}
    \item[(a)] Prove the First Welfare Theorem.
    \item[(b)] Prove an equivalent characterization of Pareto-efficient allocations. Given any set of weights $\mu_1, \dots, \mu_N \geq 0$ such that $\sum \mu_i = 1$, consider the solution $x^*$ to the following problem:
    \begin{align*}
        \max_{x_1, \dots, x_N} \quad \sum_{i=1}^N \mu_i u(x^i) \quad \text{subject to} \quad \sum_{i=1}^N x_i^j \leq \sum_{i=1}^N e_i^j \quad \forall j \in \{1, \dots, M\}.
    \end{align*}
    Prove that any solution $x^*$ is a Pareto-efficient allocation.
    \item[(c)] Provide an interpretation, in words, of what you showed in (b).
\end{enumerate}

\section*{Question 4 (20 points)}
Consider a second-price auction with $N$ bidders. Each bidder has an independent private value $v_i$ drawn from a Uniform Distribution on $[0, 1]$.
\begin{enumerate}
    \item[(a)] What is the expected revenue generated when all bidders bid truthfully? Provide a closed-form solution and show your work.
    \item[(b)] Prove that all bidders bidding truthfully is an equilibrium of the second-price auction game. What is another equilibrium? Prove your example is an equilibrium.
    \item[(c)] If the auctioneer sets a reserve price $r$, is it still a weakly dominant strategy to bid truthfully?
    \item[(d)] Suppose $N = 3$. What is the optimal reserve price $r$ the auctioneer should set? How much smaller/greater is the revenue compared to the auction with no reserve price?
    \item[(e)] A student at another university thought revenue equivalence implied that the expected revenue should be the same. Explain why their reasoning is incorrect.
\end{enumerate}

\section*{Question 5 (15 points)}
Suppose there are $N$ bidders competing for a single object in an all-pay auction. Each bidder has an i.i.d. value $v_i$ for the object drawn from some continuous distribution $F$ with support $[0, M]$.
\begin{enumerate}
    \item[(a)] Show that there is a symmetric equilibrium in increasing strategies.
    \item[(b)] What is the expected revenue generated by this auction in the equilibrium from (a)? Explain your answer.
\end{enumerate}
\section*{Solution 1:}
  \subsection*{(a)}
    Given prices $\{p_x, 1\}$ and allocations $\{x,y,\omega_{A,T}, \omega_{A,C}, \omega_{B, T}, \omega_{B,C}\}$, 
    Tom solves:
    \begin{gather*}
        u_T = x^{\alpha}y^{1-\alpha} \\
        \text{s.t.}
        p_xx + y = E_t \\
        E_t = p_x\omega_{A,T} + \omega_{B,T}
    \end{gather*}
    and Cristina solves:
    \begin{gather*}
        u_C = x^{\beta}y^{1-\beta} \\
        \shortintertext{s.t.}
        p_xx + y = E_c \\
        p_x\omega_{A,C} + \omega_{B,C}
    \end{gather*}
    Tom's Lagrangian and FOC's will be as follows:
    \begin{gather*}
      \mathcal{L}_t = x^{\alpha}y^{1-\alpha} + \lambda(E_t-p_xx - y) \\
      \mathcal{L}_x = \alpha\frac{x^{\alpha-1}y^{1-\alpha}}{p_x} = \lambda \\
      \mathcal{L}_y = (1-\alpha)x^{\alpha}y^{-\alpha} = \lambda \\
      \therefore \ \frac{\alpha x^{\alpha-1}y^{1-\alpha}}{p_x} = (1-\alpha)x^{\alpha}y^{1-\alpha} \\
      \frac{\alpha x^{\alpha-1}y}{p_x} = (1-\alpha)x^{\alpha} \\
      \frac{\alpha y}{p_x} = (1-\alpha)x \\
      y = \frac{(1-\alpha)}{\alpha}p_xx \\
      \shortintertext{Substituting into BC:} \\
      p_xx + \frac{(1-\alpha)}{\alpha}p_xx = E \rightarrow \boxed{x^*_T = \alpha\frac{E_t}{p_x}} \\
      \shortintertext{Plugging into BC:} \\
      y_T^* = E_t - p_xx \rightarrow \boxed{E_t(1-\alpha) = y_T^*}
    \end{gather*}
    Christina's Lagrangian and FOC's will follow the same path, thus I will skip redundant math and give the answers,
    \begin{gather*}
      \mathcal{L}_c = x^{\beta}y^{1-\beta} + \lambda(E_c - p_xx - y) \\
      \shortintertext{Following same steps as above yields...} \\
      \boxed{x^*_c = \beta\frac{E_c}{p_x}, \ y^*_c = E_c(1-\beta)} \\
    \end{gather*}
    Finally, equilibrium prices can be solved via the following steps:
    \begin{gather*}
      x^*_t + x^*_c = \omega_{A,T} + \omega_{A,C} \\
      \frac{E_i}{p_x}(\alpha+\beta) = \omega_{A,T} + \omega_{A,C} \\
      E_i = \omega_{A,i} + \omega_{B,i} \\
      p_x(\omega_{A,T} + \omega_{A,C}) = \alpha(p_x\omega_{A,T} + \omega_{B,T}) + \beta(p_x\omega_{A,C} + \omega_{B,C}) \\
      p_x(\alpha\omega_{A,T} + \beta\omega_{A,C}) + (\alpha\omega_{B,T} + \beta\omega_{B,C}) = p_x(\omega_{A,T}+\omega_{A,C}) \\
      (\alpha\omega_{B,T} + \beta\omega_{B,C}) = p_x[\omega_{A,T}(1-\alpha) + \omega_{A,C}(1-\beta)] \\
      \boxed{p_x^* = \frac{\alpha\omega_{B,T} + \beta\omega_{B,C}}{(1-\alpha)\omega_{A,T} + (1-\beta)\omega_{A,C}}} \\
    \end{gather*}
  \subsection*{(b)}
    Given the preferences are Cobb-Douglas, Tom will spend share $\alpha$ on apples and Christina will spend share $\beta$ on apples.
\section*{Solution 2:}
  \begin{proof}
    There are two consumers, Bob and Alice, and two goods $(x,y)$. Bob's utility is $u_B = x+y$ and Alice's is $u_A = \min\{x,y\}$. They also have endowments $E_B = (1,0)$ and $E_A = (2,1)$, respectively. Assume price of good $x$ is $p_x$ and price of good $y$ is normalized to $p_y=1$. Then, given prices $\{p_x,1\}$ and allocations $\{x,y\}$, both consumers will maximize utility and clear the market. With these conditions laid out, Bob's optimization problem is 
    \begin{gather*}
      u_B = x+y \\
      \shortintertext{s.t.} \\
      p_xx + y = p_x \\
    \end{gather*} Bob's utility function shows perfect substitutes and thus he is indifferent to the good and will base decisions on price. Therefore, if $p_x<1$, he will spend his total wealth on good $x$; if $p_x>1$, he will spend all of his wealth on good $y$; if $p_x = 1$, Bob is fully indifferent. Alice's optimization problem is as follows:
    \begin{gather*}
      u_A = \min\{x,y\} \\
      \shortintertext{s.t.} \\
      p_xx + y = 2p_x + 1 \\
      \shortintertext{but because she displays Leontief preferences, $x^* = y^*$, implying:} \\
      p_xx^* + x^* = 2p_x+1 \\
      \shortintertext{We can rearrange Alice's constraint such that:} \\
      x^*(p_x+1) = 2p_x+1 \rightarrow x^* = \frac{2p_x + 1}{p_x + 1} \\
    \end{gather*} 
    For market clearing, we know that it must be such that $x_A^* + x_B^* = 3$ and $y_A^* + y_B^* = 1$. Now, let us consider Bob's three price cases from above. First, when $p_x<1$, $x^*_B = 1$ and thus $\frac{2p_x + 1}{p_x+1} = 2$. This yields the following $2p_x + 1 = 2p_x + 2$. Subtracting the term $2p_x$ from both sides yields $1=2$, a contradiction. Thus, the case when $p<1$ is not an equilibrium. Now, the second case is when $p_x>1$. In this case, Bob spends nothing on $x$, meaning $y^*_B = p_x$ from Bob's budget constraint. Thus, market clears with $p_x + \frac{2p_x + 1}{p_x + 1} = 1$. Doing some algebra will yield $p_x^2 + 2p_x + p_x + 1 = p_x + 1$. After more rearranging, we get that $p_x^* = -2$, another contradiction. Finally, in the third case, $p_x = 1$, Bob does not care at all. So, $\frac{2+1}{1+1} = 1$. $\frac{3}{2}\ne1$, a contradiction. Therefore, there is no Walrasian equilibrium in this case. 
  \end{proof}
\section*{Solution 3:}
  \subsection*{(a)}
    \begin{proof}
      Assume there is a Walrasian equilibrium given by prices $\{p_x,p_y\}$ and allocations $\{x,y\}$. Assume the optimal bundle in this equilibrium is $(x^*,y^*)$ which clears the market such that $p_xx^* + p_yy^* = pE$. However, there is a bundle $(\bar{x},\bar{y})$ such that $u(\bar{x})>u(x^*)$ and $u(\bar{y}) > u(y^*)$. This can be further interpreted as $p_x\bar{x} > p_xx^*$ and $p_y\bar{y} > p_yy^*$. However, if this is the case, $p_x\bar{x}+p_y\bar{y}>pE$. This is infeasible, and a contradiction. Thus, the equilibrium is maximizing feasible bundles for agent utility. 
    \end{proof}
  \subsection*{(b)}
    
  \subsection*{(c)}
\section*{Solution 4:}
  \subsection*{(a)}
  \subsection*{(b)}
  \subsection*{(c)}
    Yes. In second price auctions which are increasing and symmetric, bidding truthfully is a weakly dominant strategy no matter the reserve price.
  \subsection*{(d)}
  \subsection*{(e)}
    This is incorrect because revenue equivalence states that symmetric auctions have the same basic revenue, the second highest bid.
\section*{Solution 5:}
  \subsection*{(a)}
  \subsection*{(b)}
\end{document}
