%% ECON 8041 %%
\documentclass[10pt, a4paper]{article}
\usepackage[top=3cm, bottom=4cm, left=3.5cm, right=3.5cm]{geometry}
\usepackage{amsmath, amsthm, amsfonts, amssymb, amscd, fancyhdr, color, comment, graphicx, environ}
\usepackage{float}
\usepackage{mathtools}
\usepackage{mathrsfs}
\newcommand{\course}{ECON - 8040}
\newcommand{\hwnumber}{1}
\newcommand{\Information}{Tate Mason}

\pagestyle{fancy}
\fancyhf{}
\fancyhead[L]{\course}
\fancyhead[C]{Assignment \hwnumber}
\fancyhead[R]{\Information}
\begin{document}

\begin{center}
    \Large \textbf{Assignment \hwnumber} \\
    \normalsize Due Date: January 23rd, 11:59pm \\
    Please Show Your Work and Circle Your Final Answer \\
    Submit as a single document
\end{center}

\section*{Question 1: Envelope Theorem}

Consider a constrained optimization problem:
\[
x^*(\theta) = \arg \max_{x \in X(\theta)} f(x, \theta), \quad V(\theta) = \max_{x \in X(\theta)} f(x, \theta)
\]
where $X(\theta) = \{x \in \mathbb{R}^n | g_j(x) \leq \theta_j\}$.

\begin{enumerate}
    \item[(a)] Derive the envelope theorem, computing $\frac{\partial V(\theta)}{\partial \theta_j}$ for each $j \in \{1, \ldots, m\}$.
    \item[(b)] Interpret your answer in (a), especially regarding KKT multipliers.
    \item[(c)] Apply the result to consumer maximization. Define $f(x, \theta)$, constraints $g_j$, and interpret $\theta$ and KKT multipliers.
\end{enumerate}


\section*{Question 2: Topkis' Theorem (Single-Dimension)}

\begin{enumerate}
    \item[(a)] Provide a full proof of Topkis' theorem using the approach from class.
    \item[(b)] Analyze when $q^*(\theta)$ is nondecreasing for a monopolist with inverse demand $p(q)$ and cost $c(q, \theta)$.
    \item[(c)] For a firm minimizing costs with production $f(k, \phi l) = q$, find conditions where optimal $k^*$ is weakly increasing/decreasing in $\phi$.
\end{enumerate}

\section*{Question 3: Topkis' Theorem (Multi-Dimensional)}

\begin{enumerate}
    \item[(a)] Interpret the assumption $\frac{\partial^2 f}{\partial l \partial k} < 0$.
    \item[(b)] Analyze how optimal labor choice changes as wage $w$ increases.
\end{enumerate}

\section*{Question 4: Putting it All Together}

\begin{enumerate}
    \item[(a)] For utility $u(x, y, z) = x^{1/2} y^{1/2} + z$, show optimum $T$ equals 0 or $W$.
    \item[(b)] For $u(x, y, z) = x^\alpha y^\alpha + z$, derive $T$ in terms of prices and $W$.
    \item[(c)] For $u(x, y, z) = x^\alpha y^\beta + h(z)$, show $T$ is weakly increasing in $W$.
\end{enumerate}

\section*{Solution 1:}
  \subsection*{(a)}
    Using the envelope theorem on the optimization problem:
    \begin{gather*}
      V(\theta)=\max_{x\in\mathcal{X}} f(x,\theta) \\
      \text{where} X(\theta) = \{x\in\mathbb{R}^n|g_j(x)\leq\theta_j\} \\
    \end{gather*}
    \begin{proof}
      \begin{center}
        $\frac{\partial v}{\partial\theta_j} = \frac{\partial f(x^*(\theta),\theta_j)}{\partial\theta_j}
        +\sum\limits_{j=1}^m\frac{\partial f}{\partial x_j}\frac{\partial x^*(\theta_j)}{\partial\theta_j}
        = \frac{\partial f}{\partial\theta_j} + \sum\limits_{j=1}^m\lambda\frac{\partial X(\theta_j)}{\partial x_j}
        \cdot\frac{\partial x^*(\theta)}{\partial\theta_j}$ \\
        $\sum\limits_{j=1}^m\lambda\frac{\partial X(\theta_j)}{\partial x_j}\cdot\frac{\partial x^*(\theta)}{\partial\theta_j}
        = \sum\limits_{j=1}^m\frac{\partial X(\theta)}{\partial(x_j)}\cdot\Delta x_j + \frac{\partial X(\theta)}{\partial\theta_j}\cdot\Delta\theta_j$ \\
      \end{center}
      Now, we can apply the following:
      \begin{center}
        $\frac{1}{\Delta\theta_j}\cdot[\sum\limits_{j=1}^m\frac{\partial X(\theta)}{\partial x_j}\Delta x_j +
        \frac{\partial X(\theta_j)}{\partial\theta_j}]$ \\
        $\sum\limits_{j=1}^m\frac{\partial X(\theta)}{\partial x_j}\frac{\Delta x_j}{\Delta\theta_j} +
        \frac{\partial X(\theta)}{\partial\theta_j}$ \\
      \end{center}
      Finally, we can yield an end result:
      \begin{center}
        $\boxed{\frac{V(\theta)}{\partial\theta_j} = \frac{\partial f}{\partial \theta_j} + \lambda[-\frac{\partial X(\theta)}{\partial\theta_j}]}$
      \end{center}
      This is the result since, at the max, $x^*=0$ thus eliminating the first term in the summation and leaving the derivative of the
      constraint with respect to $\theta_j$.
    \end{proof}
  \subsection*{(b)}
    
\end{document}
