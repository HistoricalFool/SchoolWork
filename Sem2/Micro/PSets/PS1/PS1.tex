%% ECON 8041 %%
\documentclass[10pt, a4paper]{article}
\usepackage[top=3cm, bottom=4cm, left=3.5cm, right=3.5cm]{geometry}
\usepackage{amsmath, amsthm, amsfonts, amssymb, amscd, fancyhdr, color, comment, graphicx, environ}
\usepackage{float}
\usepackage{mathtools}
\usepackage{mathrsfs}
\newcommand{\course}{ECON - 8040}
\newcommand{\hwnumber}{1}
\newcommand{\Information}{Tate Mason}

\pagestyle{fancy}
\fancyhf{}
\fancyhead[L]{\course}
\fancyhead[C]{Assignment \hwnumber}
\fancyhead[R]{\Information}
\begin{document}

\begin{center}
    \Large \textbf{Assignment \hwnumber} \\
    \Large \textbf{\Information} \\
    \normalsize Due Date: January 23rd, 11:59pm \\
\end{center}

\section*{Question 1: Envelope Theorem}

Consider a constrained optimization problem:
\[
x^*(\theta) = \arg \max_{x \in X(\theta)} f(x, \theta), \quad V(\theta) = \max_{x \in X(\theta)} f(x, \theta)
\]
where $X(\theta) = \{x \in \mathbb{R}^n | g_j(x) \leq \theta_j\}$.

\begin{enumerate}
    \item[(a)] Derive the envelope theorem, computing $\frac{\partial V(\theta)}{\partial \theta_j}$ for each $j \in \{1, \ldots, m\}$.
    \item[(b)] Interpret your answer in (a), especially regarding KKT multipliers.
    \item[(c)] Apply the result to consumer maximization. Define $f(x, \theta)$, constraints $g_j$, and interpret $\theta$ and KKT multipliers.
\end{enumerate}


\section*{Question 2: Topkis' Theorem (Single-Dimension)}

\begin{enumerate}
    \item[(a)] Provide a full proof of Topkis' theorem using the approach from class.
    \item[(b)] Analyze when $q^*(\theta)$ is nondecreasing for a monopolist with inverse demand $p(q)$ and cost $c(q, \theta)$.
    \item[(c)] For a firm minimizing costs with production $f(k, \phi l) = q$, find conditions where optimal $k^*$ is weakly increasing/decreasing in $\phi$.
\end{enumerate}

\section*{Question 3: Topkis' Theorem (Multi-Dimensional)}

A firm has two inputs, capital ($k$) and labor ($l$), that it can use for production. Cost of capital and labor are $r \text{and} w$ , respectively. The firm's production function is 
given by $f(k, l)$, and the firm sells its product at a fixed price $p$. Unlike part (c) in the previous question, the firm is not constrained to produce a fixed level of output.
Thus, the firm's problem is:
\begin{center}
    $\max_{k,l} pf(k,l) - wl - rk$
\end{center}
Assume that $\frac{\partial f}{\partial k} \text{and} \frac{\partial f}{\partial l}$ exist and are positive, and that $\frac{\partial^2 f}{\partial k\partial l} < 0$.
\begin{enumerate}
    \item[(a)] Provide an interpretation for the assumption that $\frac{\partial^2 f}{\partial k\partial l} < 0$.
    \item[(b)] Analyze how optimal labor choice changes as wage $w$ increases.
\end{enumerate}

\section*{Question 4: Putting it All Together}

\begin{enumerate}
    \item[(a)] For utility $u(x, y, z) = x^{1/2} y^{1/2} + z$, show optimum $T$ equals 0 or $W$.
    \item[(b)] For $u(x, y, z) = x^\alpha y^\alpha + z$, derive $T$ in terms of prices and $W$.
    \item[(c)] For $u(x, y, z) = x^\alpha y^\beta + h(z)$, show $T$ is weakly increasing in $W$.
\end{enumerate}

\section*{Solution 1:}
    \subsection*{(a)}
        Using the envelope theorem on the optimization problem:
        \begin{gather*}
        V(\theta)=\max_{x\in\mathcal{X}} f(x,\theta) \\
        \text{where} X(\theta) = \{x\in\mathbb{R}^n|g_j(x)\leq\theta_j\} \\
        \end{gather*}
        \begin{proof}
        \begin{center}
            $\frac{\partial v}{\partial\theta_j} = \frac{\partial f(x^*(\theta),\theta_j)}{\partial\theta_j}
            +\sum\limits_{j=1}^m\frac{\partial f}{\partial x_j}\frac{\partial x^*(\theta_j)}{\partial\theta_j}
            = \frac{\partial f}{\partial\theta_j} + \sum\limits_{j=1}^m\lambda\frac{\partial X(\theta_j)}{\partial x_j}
            \cdot\frac{\partial x^*(\theta)}{\partial\theta_j}$ \\
            $\sum\limits_{j=1}^m\lambda\frac{\partial X(\theta_j)}{\partial x_j}\cdot\frac{\partial x^*(\theta)}{\partial\theta_j}
            = \sum\limits_{j=1}^m\frac{\partial X(\theta)}{\partial(x_j)}\cdot\Delta x_j + \frac{\partial X(\theta)}{\partial\theta_j}\cdot\Delta\theta_j$ \\
        \end{center}
        Now, we can apply the following:
        \begin{center}
            $\frac{1}{\Delta\theta_j}\cdot[\sum\limits_{j=1}^m\frac{\partial X(\theta)}{\partial x_j}\Delta x_j +
            \frac{\partial X(\theta_j)}{\partial\theta_j}]$ \\
            $\sum\limits_{j=1}^m\frac{\partial X(\theta)}{\partial x_j}\frac{\Delta x_j}{\Delta\theta_j} +
            \frac{\partial X(\theta)}{\partial\theta_j}$ \\
        \end{center}
        Finally, we can yield an end result:
        \begin{center}
            $\boxed{\frac{V(\theta)}{\partial\theta_j} = \frac{\partial f}{\partial \theta_j} +\sum\limits_{j=1}^m \lambda_j[-\frac{\partial X(\theta)}{\partial\theta_j}]}$
        \end{center}
        This is the result since, at the max, $x^*=0$ thus eliminating the first term in the summation and leaving the derivative of the
        constraint with respect to $\theta_j$.
        \end{proof}
    \subsection*{(b)}
        The derivative of the value function with respect to $\theta_j$ is the derivative of the objective function with respect to $\theta_j$ plus the KKT multipliers $\sum\limits_{j=1}^m\lambda_j$
        multiplied by the derivative of the constraint with respect to $\theta_j$. This means that the KKT multipliers have an effect on the value function via the constraint. The KKT multipliers are
        the shadow prices of the constraints, which are the marginal utility of the budget constraint.
    \subsection*{(c)}
        For consumer maximization, the objective function is the utility function $f(x,\theta)$, the constraints are the budget constraint $g_j(x)\leq\theta_j$, and $\theta$ is the budget. The KKT multipliers
        are the shadow prices of the constraints, which are the marginal utility of the budget constraint.
\section*{Solution 2:}
    \subsection*{(a)}
        \begin{proof}
          Let's assume there exist variables $a_H$ and $a_L$ such that $a_H > a_L$. Further, define $f(a_H,\theta') - f(a_L,\theta') \geq f(a_H,\theta)-f(a_L,\theta)$. Next, substitute
          $a_H = \max\{x,x'\} \text{and} a_L = x'$ such that $x\in x^*(\theta) \text{and} x'\in x^*(\theta')$. This yields $f(\max\{x,x'\},\theta') - f(x',\theta') \geq f(\max\{x,x'\},\theta) - f(x',\theta)$.
          Further, define $\theta'>\theta$. If the maximum of $x$ and $x'$ with respect to $\theta$ is $x'$, the right side of the term goes to zero. Otherwise, the inequality holds. Let's
          proceed with the assumption that the right is zeroed out, allowing us the rearrange and get the inequality $f(\max\{x',x\},\theta')\geq f(x',\theta')$. This, therefore, states that 
          $\max\{x',x\} \in x^*(\theta')$. Now, assume that $a_H = x$ and $a_L = \min\{x,x'\}$. This yields $f(x,\theta') - f(\min\{x,x'\},\theta') \geq f(x,\theta) - f(\min\{x,x'\},\theta)$. 
          Now, assuming that $\min\{x,x'\} = x$, the left side equals 0, thus allowing us to rearrange as $f(\min\{x,x'\},\theta)\geq f(x,\theta)$. This result shows that $\min\{x,x'\}\in x^*(\theta)$.
      \end{proof}
    \subsection*{(b)}
      In this case, we can say that $-c^*(q)$ has increasing differences. Derivation follows:
      \begin{gather*}
        p(q)q - c(q,\theta) = \Pi \\
        \text{Applying Topkis' Theorem} \\
        -c(q',\theta')-(-c(q,\theta')) \geq -c(q,\theta)-(-c(q,\theta)) \geq 0 \\
        q\in q^*(\theta); \ q'\in q^*(\theta'); \ \theta' > \theta \\
        c(q,\theta') >_s c(q,\theta)
      \end{gather*}
    \subsection*{(c)}
      
\section*{Question 3:}
    \subsection*{(a)}
      $\frac{\partial^2 f}{\partial l\partial k}<0$ is due to both $\frac{\partial f}{\partial k}, \ \frac{\partial f}{\partial l} >0$. Thus, the cross-partial will be negative due to concavity.
      Further, this makes intuitive sense as an increase in cost in one input would necessitate a decrease in the use of that input to keep costs at a utility maximizing level.
    
    \subsection*{(b)}
      As $w$ increases, labor will be less utilized in favor of capital inputs. Again, this makes intuitive sense trying to keep costs at optimal.

\section*{Question 4:}
    \subsection*{(a)}
      \begin{gather*}
        u(x,y,z) = x^{\frac{1}{2}}y^{\frac{1}{2}}+z \\
        \text{s.t.} \\
        p_xx + p_yy + p_zz = W \\
        p_xx +p_yy = T \\
        \text{Therefore} T+p_zz = W \\
      \end{gather*}
      Case 1:
      \begin{gather*}
        T = 0 \\
        p_zz = W
      \end{gather*}
      In this case, all income is spent on good $z$.

      Case 2:
      \begin{gather*}
        T = W \\
        W + p_zz = W \\
        p_zz = 0 \\
      \end{gather*}
      In this case, no money is spent on good $z$. Instead, the agent favors a bundle of $x,y$.
    \subsection*{(b)}
      \begin{gather*}
        u(x,y,z) = x^{\alpha}y^{\alpha} + z \\
        \text{s.t.} \\
        p_xx + p_yy + p_zz = W \\
        \alpha\in(0,\frac{1}{2})
      \end{gather*}
      Form Lagrangian:
      \begin{gather*}
        \mathcal{L} = x^{\alpha}y^{\alpha} + z + \lambda(W - p_xx + p_yy + p_zz) \\
        \frac{\partial \mathcal{L}}{\partial x}: \alpha x^{\alpha-1}y^{\alpha} = p_x\lambda \\
        \frac{\partial \mathcal{L}}{\partial y}: \alpha x^{\alpha}y^{\alpha-1} = p_y\lambda \\
        \frac{\partial \mathcal{L}}{\partial z}: 1 = p_z\lambda \\
        \frac{\partial \mathcal{L}}{\partial \lambda}: p_xx + p_yy + p_zz = W \\
      \end{gather*}
      
\end{document}