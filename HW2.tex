%%% Homework 2 - Advanced Micro %%%
\documentclass[10pt, a4paper]{article}
\usepackage[top=3cm, bottom=4cm, left=3.5cm, right=3.5cm]{geometry}
\usepackage{amsmath,amsthm,amsfonts,amssymb,amscd, fancyhdr, color, comment, graphicx, environ}
\usepackage{float}
\usepackage{mathtools}
\usepackage{mathrsfs}
\usepackage[math-style=ISO]{unicode-math}
\DeclareSymbolFont{\mathnormal}{letters}
\usepackage{lastpage}

%%%%%%%%%%%%%%%%%%%%%%%%%%%%%%%%%%%%%%%%%%%%%%%%%%%%%%%%%%%%%%%%%%
%%%%%%%%%%%%%%%%%%%%%%%%%%%%%%%%%%%%%%%%%%%%%%%%%%%%%%%%%%%%%%%%%%
%Fill in the appropriate information below
\newcommand{\norm}[1]{\left\lVert#1\right\rVert}     
\newcommand\course{ECON - 8010}                            % <-- course name   
\newcommand\hwnumber{ 2}                                 % <-- homework number
\newcommand\Information{Tate Mason}                        % <-- personal information
%%%%%%%%%%%%%%%%%%%%%%%%%%%%%%%%%%%%%%%%%%%%%%%%%%%%%%%%%%%%%%%%%%
%%%%%%%%%%%%%%%%%%%%%%%%%%%%%%%%%%%%%%%%%%%%%%%%%%%%%%%%%%%%%%%%%%
%Page setup
\pagestyle{fancy}
\headheight 35pt
\lhead{\today}
\rhead{}
\lfoot{}
\pagenumbering{arabic}
\cfoot{\small\thepage}
\rfoot{}
\headsep 1.2em
\renewcommand{\baselinestretch}{1.25}
%%%%%%%%%%%%%%%%%%%%%%%%%%%%%%%%%%%%%%%%%%%%%%%%%%%%%%%%%%%%%%%%%%
%%%%%%%%%%%%%%%%%%%%%%%%%%%%%%%%%%%%%%%%%%%%%%%%%%%%%%%%%%%%%%%%%%
%Add new commands here
\renewcommand{\labelenumi}{\alph{enumi})}
\newcommand{\Z}{\mathbb Z}
\newcommand{\R}{\mathbb R}
\newcommand{\Q}{\mathbb Q}
\newcommand{\NN}{\mathbb N}
\newcommand{\PP}{\mathbb P}
\DeclareMathOperator{\Mod}{Mod} 
\renewcommand\lstlistingname{Algorithm}
\renewcommand\lstlistlistingname{Algorithms}
\def\lstlistingautorefname{Alg.}
\newtheorem*{theorem}{Theorem}
\newtheorem*{lemma}{Lemma}
\newtheorem{case}{Case}
\newcommand{\assign}{:=}
\newcommand{\infixiff}{\text{ iff }}
\newcommand{\nobracket}{}
\newcommand{\backassign}{=:}
\newcommand{\tmmathbf}[1]{\ensuremath{\boldsymbol{#1}}}
\newcommand{\tmop}[1]{\ensuremath{\operatorname{#1}}}
\newcommand{\tmtextbf}[1]{\text{{\bfseries{#1}}}}
\newcommand{\tmtextit}[1]{\text{{\itshape{#1}}}}

\newenvironment{itemizedot}{\begin{itemize} \renewcommand{\labelitemi}{$\bullet$}\renewcommand{\labelitemii}{$\bullet$}\renewcommand{\labelitemiii}{$\bullet$}\renewcommand{\labelitemiv}{$\bullet$}}{\end{itemize}}
\catcode`\<=\active \def<{
\fontencoding{T1}\selectfont\symbol{60}\fontencoding{\encodingdefault}}
\catcode`\>=\active \def>{
\fontencoding{T1}\selectfont\symbol{62}\fontencoding{\encodingdefault}}
\catcode`\<=\active \def<{
\fontencoding{T1}\selectfont\symbol{60}\fontencoding{\encodingdefault}}

%%%%%%%%%%%%%%%%%%%%%%%%%%%%%%%%%%%%%%%%%%%%%%%%%%%%%%%%%%%%%%%%%%
%%%%%%%%%%%%%%%%%%%%%%%%%%%%%%%%%%%%%%%%%%%%%%%%%%%%%%%%%%%%%%%%%%
%Begin now!

\begin{document}
  \begin{titlepage}
    \begin{center}
      \vspace*{3cm}
            
        \vspace{1cm}
        \huge
        Homework \hwnumber
            
        \vspace{1.5cm}
        \Large
            
        \textbf{\Information}                      % <-- author
            
        \vfill
        
        An \course \ Homework Assignment
            
        \vspace{1cm}
        \Large

        
        \today
            
    \end{center}
  \end{titlepage}

  \newpage

  \section{Question 3.C.1}
    \subsection{Problem}
      Verify that the lexicographic ordering is complete, transitive, strongly monotone, and strictly convex.
    \subsection{Solution}
    Assume that $\exists X,Y\in\mathbb{R}^2$ and $X=(x_1,x_2), Y=(y_1,y_2)$. These preferences are complete by nature. $X\succ Y$ is true if either $x_1>y_1$ or $x_1=y_1$ and $x_2\geq y_2$. $y\succ x$ is true if either $y_1>x_1$ or $y_1=x_1$ and $y_2>x_2$. Further, if $x_1=y_1$ and $x_2=y_2$, $X=Y$. These conditions ensure that ordering is complete. Assume that $\exists Z=(z_,z_2)$. Then, if $X\geq Y, Y\geq Z,$ then $X\geq Z$. This can also be seen as $x_2\geq y_2\geq z_2$, which also implies that $X\geq Z$. These condiitons imply that preferences are transitive. Now, we can imply strong monotonicity if $(x_1,x_2)\geq(y_1,y_2)$ and $(x_1,x_2)\ne(y_1,y_2)$. These assertions imply that $x_1>y_1$ or $x_1=y_1$ and $x_2>y_2$. Either way, $X>Y$, showing strong monotonicity. Assume $x\geq z$ and $y\geq z$, we can then say that $\alpha x+(1-\alpha)y\geq z$. Now, assume that $x\ne y\ne z$, this implies that $x_1>z_1$ or $x_1=z_1$ and $x_2>z_2$. Therefore, either $\alpha x_1 +(1-\alpha)y_1>z_1$ or $\alpha x_1 +(1-\alpha)y_1=z_1$ and $\alpha x_2 +(1-\alpha)y_2>z_2$. In any case, we can conclude strict convexity $\alpha x+(1-\alpha)y>z$. 
  \section{Question 3.C.6}
    \subsection{Problem}
      Suppose that in a two commodity world, the consumer's utility function take the form $u(x) = [\alpha_1x_1^{\rho}+\alpha_2x_2^{\rho}]^{\frac{1}{\rho}}$. This utility function is known as the \textit{constant elasticity of substitution} (or \textit{CES})
      utility function.

      (a) Show that when $\rho =1$, indifference curves become linear.

      (b) Show that as $\rho\rightarrow0$, this utility function comes to represent the same preferences as the (generalized) Cobb-Douglas utility function $u(x)=x_1^{\alpha_1}x_2^{\alpha_2}$.

      (c) Show that as $\rho\rightarrow-\infty$, indifference curves become "right angles"; that is, this utility function has in the limit the indifference map of the Leontief utility function $u(x_1,x_2)=\min\{x_1,x_2\}$.
    \subsection{Solution}
      
      (a) $\rho=1$: $u(x)=[\alpha x_1^1+\alpha x_2^1]^1\Rightarrow\alpha[x_1+x_2]$ Which is a linear curve shifted by $\alpha$.

      (b) As $\rho\rightarrow0$: First, let's take the log transform.
        \begin{center}
          $\ln u(x)=\frac{1}{\rho}\ln[\alpha_1x_1^{\rho}+\alpha_2x_2^{\rho}]$
        \end{center}
        Then, using L'Hopital's rule, we can do the following.
        \begin{center}
          ${\lim\atop{p\rightarrow0}}\ln u(x)={\lim\atop{p\rightarrow0}}\frac{\alpha_1 x_1^{\rho}\ln x_1+\alpha_2x_2^{\rho}\ln x_2}{\alpha_1x_1^{\rho}+\alpha_2x_2^{\rho}}$
        \end{center}
        As $p\rightarrow0$, $x_i^{\rho}\rightarrow1$, leading us to:
        \begin{center}
          ${\lim\atop{p\rightarrow0}}\ln u(x)=\frac{\alpha_1\ln x_1+\alpha_2\ln x_2}{\alpha_1+\alpha_2}}$
        \end{center}
        Next, we will assume that $\alpha_1+\alpha_2=1$, granting us
        \begin{center}
          $\ln u(x)=\alpha\ln x_1+\alpha_2\ln x_2$
        \end{center}
        Finally, we can take the exponential of both sides, getting us
        \begin{center}
          $\Aboxed{u(x)=x_1^{\alpha_1}+x_2^{\alpha_2}}$
        \end{center}

      (c) As $\rho\rightarrow-\infty$: if $x_1>x_2$, $x_1^{\rho}\gg x_2^{\rho}$ as $p\rightarrow-\infty$. Similarly, if $x_2<x_1$, $x_2{\rho}\gg x_!^{\rho}$ as $p\rightarrow-\infty$. So, we can write that:
        \begin{center}
          as $p\rightarrow-\infty$: $u(x)\approx(\min\{\alpha_1x_1^{\rho},\alpha_2x_2^{\rho}\})$
        \end{center}
        Next, we will raise both sides by $\rho$, keeping in mind that as $p\rightarrow-\infty$ $\frac{1}{\rho}\rightarrow0$ 
        \begin{center}
          $u(x)^{\rho}\approx\min\{\alpha_1x_1^{\rho},\alpha_2x_2^{\rho}\}$
        \end{center}
        Then as $p\rightarrow-\infty$ this approaches
        \begin{center}
          $\Aboxed{u(x)\approx\min\{x_1,x_2\}}$
        \end{center}
        Assuming that $\alpha_1=\alpha_2=1$. These results yield the Leontief utility function. 
  \section{Question 3.D.5 - (a)\&(c)}
    \subsection{Problem}
      Consider again the CES utility function from 3.C.6, and assume that $\alpha_1=\alpha_2=1$.

      (a) Compute the Walrasian demand and indirect utility functions for this utility function

      (c) Derive the Walrasian demand correspondence and indirect utility function for the case of linear utility and the case of Leontief utility (see exercise 3.C.6). Show that the CES Walrasian demand and indirect utility functions approach these as $\rho$ approaches $1$ and $-\infty$, respectively.

    \subsection{Solution}
    
    (a):
    
    Walrasian:
    \begin{center}
      $L_{x_1}: \frac{1}{\rho}[x_1^{\rho}+x_2^{\rho}]^{\frac{1}{\rho}-1}\cdot \rho x_1^{\rho-1}-\lambda p_1$ \\
      $L_{x_2}: \frac{1}{\rho}[x_1^{\rho}+x_2^{\rho}]^{\frac{1}{\rho}-1}\cdot\rho x_2^{\rho-1}-\lambda p_2$ \\ 
      $(\frac{x_1}{x_2})^{\rho-1}=\frac{p_1}{p_2}$ \\
      $x_1 = (\frac{p_1}{p_2})^{\frac{1}{\rho-1}}x_2$ \\ 
      $x_2(\frac{p_1}{p_2})^{\frac{1}{\rho-1}}p_1+p_2x_2=w$ \\
      $x_2[(\frac{p_1}{p_2})^{\frac{1}{\rho-1}}p_1+p_2]^=w$ \\
      $\Aboxed{x_2 = \frac{wp_2^{\frac{1}{\rho-1}}}{{p_1^{\frac{\rho}{\rho-1}}+p_2^{\frac{\rho}{\rho-1}}}}}$ \\
      $\therefore$
      $\Aboxed{x_1 =\frac{wp_1^{\frac{1}{\rho-1}}}{{p_1^{\frac{\rho}{\rho-1}}+p_2^{\frac{\rho}{\rho-1}}}}}$ \\
    \end{center}
    Indirect Utility Function:
    \begin{center}
      $v(x) = [(\frac{wp_1^{\frac{1}{\rho-1}}}{{p_1^{\frac{\rho}{\rho-1}}+p_2^{\frac{\rho}{\rho-1}}}})^{\rho}(\frac{wp_2^{\frac{1}{\rho-1}}}{p_1^{\frac{\rho}{\rho-1}}+p_2^{\frac{\rho}{\rho-1}}})^{\rho}]^{\frac{1}{\rho}}$ \\
      $\Aboxed{v(x)=(\frac{w}{p_1^{\frac{\rho}{\rho-1}}+p_2^{\frac{\rho}{\rho-1}}})^{\frac{1}{\rho}}}$
    \end{center}

    (c):
    
    Linear utility:
    \begin{center}
      $x_1+x_2=u$
    \end{center}
    if $p_1<p_2$, 
    \begin{center}
      $x_1 = \frac{w}{p_1}$ and $x_2=0$.
    \end{center}
    \begin{center}
      $v = \frac{w}{p_1}$
    \end{center}
    if $p_1=p_2$
    \begin{center}
      $\lambda x_1p_1+(1-\lambda)x_2p_2=w$
    \end{center}
    in this case, any $(x_1,x_2)$ works within the budget line
    \begin{center}
      $V  = \lambda x_1+(1-\lambda)x_2$
    \end{center}
    Let's first consider $\rho\rightarrow1$
    $p_1<p_2$ and let $\beta=\frac{\rho}{\rho-1}$. As $\frac{p_2}{p_1}>1$, $(\frac{p_2}{p_1})^{\beta}\rightarrow0$. This leads us to 
      
    \begin{center}
      ${\lim\atop{\beta\rightarrow-\infty}}\frac{p_1^{\beta-1}w}{p_1^{\beta}+p_2^{\beta}} = {\lim\atop{\beta\rightarrow-\infty}}\frac{w}{p_1(1+(\frac{p_2}{p_1})^{\beta})}=\frac{w}{p_1}$
    \end{center}

    Now, $\frac{p_1}{p_2}<1$, $(\frac{p_1}{p_2})^{\beta}\rightarrow\infty$. This leads us to
    \begin{center}
      ${\lim\atop{\beta\rightarrow-\infty}}\frac{p_2^{\beta-1}w}{p_1^{\beta}+p_2^{\beta}}={\lim\atop{\beta\rightarrow-\infty}}\frac{w}{p_2(1+(\frac{p_1}{p_2})^{\beta}}=0$
    \end{center}

    Therefore, when $p_1<p_2$, there is convergence to a linear indifference curve. CES utility will, similarly, converge to utility of linear indifference curves. This is given by the fact that $(p_1^{\beta}+p_2^{\beta})^{\frac{1}{\beta}}\rightarrow p_2$ as $p_1\leq p_2$.

    The case in which $p_2<p_1$ follows the same logic and thus would be redundant to show.

    The case in which $p_1=p_2$ yields
    \begin{center}
      $\frac{w}{(p_1^{\beta}+p_2^{\beta})(p_1^{\beta-1}p_2^{\beta-1})}=\frac{w}{(p_1^{\beta}+p_2^{\beta})(p_1^{\beta-1}p_2^{\beta-1})}=\frac{w}{2p_1}$
    \end{center}
    Now, consider $\rho\rightarrow-\infty$. $\beta=\frac{\rho}{\rho-1}\rightarrow1$ as $\rho\rightarrow1$. Plugging in $\beta=1$ into the Walrasian 
    \begin{center}
      $\Aboxed{x_1=\frac{w(\frac{p_1}{p_2})^{\beta-1}}{p_1(\frac{p_1}{p_2})^{\beta-1}+p_2}}$ \\

      $\Aboxed{x_2 = \frac{w}{p_1(\frac{p_1}{p_2})^{\beta-1}+p_2}}$
    \end{center}
    yields the Leontief demand function.

  \section{Question 3.D.8}
    \subsection{Problem}
      Show that for all $(p,w)$, $w\frac{\partial v(p,w)}{\partial w}=-p\cdot\nabla_pv(p,w)$.

    \subsection{Solution}
      Walras' law states $p\cdot x(p,w)=w$. Roy's identity states that $0=x(\bar{p},\bar{w})\cdot\frac{\partial v(p,w)}{\partial w}+\nabla_pv(p,w)$. Combining the two, we get $-\frac{w}{p}\cdot\frac{\partial v(p,w)}{\partial w}=\nabla_p v(p,w)$. Doing some algebra gets us $w\frac{\partial v(p,w)}{\partial w}=-p\cdot\nabla_pv(p,w)$
  \section{Question 3.E.5}
    \subsection{Problem}
      Show that if $u(\cdot)$ is homogenous of degree one, then $h(p,u)$ and $e(p,u)$ are homogenous of degree one in $u$ [i.e., they can be written as $h(p,u)=\tilde{h}(p)u$ and $e(p,u)=\tilde{e}(p)u$].

    \subsection{Solution}
    \begin{proof}
      Consider utility function $u(x)$. For $u(x)$ to be homogenous of degree 1, $u(\lambda x)=\lambda u(x)$. Let's apply this first to the Hicksian case. Consider the form $h(,u)=\min p\cdot x$ s.t. $u(x)\geq u$. Now, for any lambda, assume $\min p\cdot x$ s.t. $u(x)\geq\lambda u$. Due to the assumed homoegeneity of $u(x)$, we can rearrange the problem as $\min p\cdot x$ s.t. $u(\frac{x}{\lambda})\geq u$. Now, assume $\exists x^*$ such that $x^*$ is the solution to the original $u(x)$, then $\lambda x^*$ is the solution to the new problem as $u(\frac{\lambda x^*}{\lambda})=u(x^*)\geq u$. To further reinforce $\lambda x^*$ being the solution, assume $\exists y$ such that $p\cdot y<p\cdot (\lambda x^*)$. This would imply $y$ solves the minimization problem rather than $x$. This is a contradiction, therefore $h(p,\lambda u)=\lambda h(p,u)$. This shows that $h(p,u)$ is HD1 in $u$, allowing us to write that $h(p,u)=\tilde{h}(p)u$.
    \end{proof}

    \begin{proof}
      The expenditure function follows much of the same so this proof will be more concise. $e(p,u)=p\cdot h(p,u) = p\cdot(\lambda h(p,u))=\lambda(p\cdot h(p,u))=\lambda e(p,u)$. This sequence is sufficient for showing that $e$ is HD1, and therefore $e(p,u)=\tilde{e}(p)u$.
    \end{proof}
  \section{Question 3.G.12}
    \subsection{Problem}
       What restrictions on the Gorman form correspond to the cases of homothetic and quasilinear preferences?
    \subsection{Solution}
      For homothetic preferences, $a(p)$ must be held constant and for quasilinear preferences, $b(p)$ is held constant. We hold $a(p)$ constant to ensure that $v(p,w)$ is HD1, i.e. increasing wealth by $\alpha$, increases utility by $\alpha$ as well. We hold $b(p)$ constant to ensure that changes in wealth lead to positive but non-increasing levels of utility. 
\end{document}
