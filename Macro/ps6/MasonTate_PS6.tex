%% ECON 8040 - Macro: Problem Set 6 %%
\documentclass[10pt, a4paper]{article}
\usepackage[top=3cm, bottom=4cm, left=3.5cm, right=3.5cm]{geometry}
\usepackage{amsmath,amsthm,amsfonts,amssymb,amscd, fancyhdr, color, comment, graphicx, environ}
\usepackage{float}
\usepackage{mathtools}
\usepackage{mathrsfs}
\usepackage[math-style=ISO]{unicode-math}
\DeclareSymbolFont{\mathnormal}{letters}
\usepackage{lastpage}

%%%%%%%%%%%%%%%%%%%%%%%%%%%%%%%%%%%%%%%%%%%%%%%%%%%%%%%%%%%%%%%%%%
%%%%%%%%%%%%%%%%%%%%%%%%%%%%%%%%%%%%%%%%%%%%%%%%%%%%%%%%%%%%%%%%%%
%Fill in the appropriate information below
\newcommand{\norm}[1]{\left\lVert#1\right\rVert}     
\newcommand\course{ECON - 8040}                            
\newcommand\hwnumber{6}                                 
\newcommand\Information{Tate Mason}                        
%%%%%%%%%%%%%%%%%%%%%%%%%%%%%%%%%%%%%%%%%%%%%%%%%%%%%%%%%%%%%%%%%%
%%%%%%%%%%%%%%%%%%%%%%%%%%%%%%%%%%%%%%%%%%%%%%%%%%%%%%%%%%%%%%%%%%
%Page setup
\pagestyle{fancy}
\headheight 35pt
\lhead{\today}
\rhead{}
\lfoot{}
\pagenumbering{arabic}
\cfoot{\small\thepage}
\rfoot{}
\headsep 1.2em
\renewcommand{\baselinestretch}{1.25}

%%%%%%%%%%%%%%%%%%%%%%%%%%%%%%%%%%%%%%%%%%%%%%%%%%%%%%%%%%%%%%%%%%
%%%%%%%%%%%%%%%%%%%%%%%%%%%%%%%%%%%%%%%%%%%%%%%%%%%%%%%%%%%%%%%%%%
%Add new commands here
\renewcommand{\labelenumi}{\arabic{enumi}.}
\newcommand{\Z}{\mathbb Z}
\newcommand{\R}{\mathbb R}
\newcommand{\Q}{\mathbb Q}
\newcommand{\NN}{\mathbb N}
\newcommand{\PP}{\mathbb P}
\DeclareMathOperator{\Mod}{Mod}
\newtheorem*{theorem}{Theorem}
\newtheorem*{lemma}{Lemma}
\newcommand{\assign}{:=}

\begin{document}
  \begin{titlepage}
    \begin{center}
      \vspace*{3cm}
            
      \vspace{1cm}
      \huge
      Homework \hwnumber
            
      \vspace{1.5cm}
      \Large
            
      \textbf{\Information}
            
      \vfill
      
      Collaboration to varying degrees with Timothy Duhon, Josephine Hughes, Abdul Khan, Kasra Lak, Rachel Lobo, Mingzhou Wang, Wenyi Wang
      \vspace{1cm}
      
      Due on Tuesday October 29, by 11:49pm

      \vspace{1cm}

      An \course \ Homework Assignment
            
      \vspace{1cm}
      \Large
      
      \today
            
    \end{center}
  \end{titlepage}

\section*{Question 1}
  \subsection*{Problem}
      Consider the example of our endowment economy with two types of households. Assume utility function is
      \begin{gather*}
        u(c) = \frac{c^{1-\sigma}-1}{1-\sigma} \text{ for } \sigma > 0
      \end{gather*}
      Also assume endowments growth at constant rate, i.e.,
      \begin{gather*}
        e^1_t = \begin{cases}
          2\gamma^t & \text{if t is even}\\
          0 & \text{if t is odd}
        \end{cases}\\
        e^2_t = \begin{cases}
          0 & \text{if t is even}\\
          2\gamma^t & \text{if t is odd}
        \end{cases}
      \end{gather*}
      Let $0 < \beta < 1$ be discount rate and assume $\beta\gamma^{1-\sigma} < 1$. (Note: for $\sigma = 1$, $u(c) = \log(c)$. Therefore the example we studied in class was a special case of this problem with $\gamma = 1$ and $\sigma = 1$)
  \subsection*{Parts}
    \subsubsection*{(a)}
      Define (Arrow-Debreu) competitive equilibrium.
    \subsubsection*{(b)}
      Drive household's Euler equation.
    \subsubsection*{(c)}
      Write down a social planner problem for some Pareto weights and solve for Pareto efficient allocations (treat Pareto weights as parameters). hint: it is easier to solve everything in terms of the ratio of weights $\alpha \equiv \frac{\alpha_2}{\alpha_1}$.
    \subsubsection*{(d)}
      Use Negishi Method to solve for competitive equilibrium allocations and prices.
    \subsubsection*{(e)}
      Find Pareto weights that generate the following allocation $(c^1_t, c^2_t) = (\gamma^t, \gamma^t)$. Find transfers needed to implement this allocation in a competitive equilibrium.
    \subsubsection*{(f)}
      How does growth rate of consumption depend on $\gamma$ and $\sigma$? How do equilibrium prices depend on $\gamma$ and $\sigma$?
    \subsubsection*{(g)}
      Define a sequential market competitive equilibrium. Find interest rate in this equilibrium. How do equilibrium interest rate depend on $\gamma$ and $\sigma$?
  \subsection*{Solutions}
    \subsubsection*{(a)}
      \begin{gather*}
        {\max\atop{\{c^i_t\}^{\infty}_{t=0}}} \sum\limits_{t=0}^{\infty}\beta^t(\frac{c_t^i^{1-\sigma}-1}{1-\sigma})
      \end{gather*}
      s.t.
      \begin{gather*}
        \sum\limits^{\infty}_{t=0} \hat{p}_tc_t^i \leq \sum\limits_{t=0}^{\infty}\hat{p}_te_t^i \\
        \hat{c}_t^i \geq 0 \\
        0 < \beta < 1\\
      \end{gather*}
      Markets Clear:
      \begin{gather*}
        \hat{c}_t^1+\hat{c}_t^2 = \hat{e}_t^1+\hat{e}_t^2 \ \forall \ t
      \end{gather*}
    \subsubsection*{(b)}
      \begin{gather*}
        c^i_t: \frac{\beta^t}{{c_t^i}^{\sigma}} - \lambda p_t \\
        c_{t+1}^i: \frac{\beta^{t+1}}{{c_{t+1}^i}^{\sigma}} - \lambda p_{t+1} \\
        \beta\frac{{c_{t+1}^i}^{\sigma}}{{c_t^i}^{\sigma}} = \frac{p_t}{p_{t+1}} \\
        {c_{t+1}^i}^{\sigma}p_{t+1} = \beta {c_t^i}^{\sigma}p_t
      \end{gather*}
    \subsubsection*{(c)}
      \begin{gather*}
        {\max\atop{\{c^1_t, c^2_t\}_{t=0}^{\infty}}}\sum\limits_{t=0}^{\infty}\beta^t[\alpha_1\frac{c^1_t^{1-\sigma}-1}{1-\sigma}+\alpha_2\frac{c^2_t^{1-\sigma}}{1-\sigma}] \\
      \end{gather*}
      s.t. 
      \begin{gather*}
        c_t^1+c_t^2=e_t^1+e_t^2 = 2\gamma^t; \lambda
      \end{gather*}
      Lagrangian:
      \begin{gather*}
        \mathcal{L} = \sum\limits_{t=0}^{\infty}\alpha\sum\limits_{t=0}^{\infty}\beta^t\frac{{c_t^i}^{1-\sigma}-1}{1-\sigma} + \frac{\mu}{2}[\sum\limits_{t=0}^{\infty} e_t^i - \sum\limits_{t=0}^{\infty} c_t^i] \\
        \mathcal{L}_c = \alpha\beta c^{-\sigma}-\frac{\mu}{2} = 0 \\
        c_t^1: \alpha_1\beta {c_t^1}^{-\sigma}-\frac{\mu}{2} \\
        c_t^2: \alpha_2\beta {c_t^2}^{-\sigma}-\frac{\mu}{2} \\
      \end{gather*}
    \subsubsection*{(d)}
      Divide the two FOC's
      \begin{gather*}
        \frac{\alpha_1\beta {c_t^1}^{-\sigma}-\frac{\mu}{2}}{\alpha_2\beta {c_t^2}^{-\sigma}-\frac{\mu}{2}} \\
        1 = \frac{\alpha_1}{\alpha_2} (\frac{c_t^1}{c_t^2})^{-\sigma} \\
        (\frac{\alpha_2}{\alpha_1})^{\sigma} = \frac{c_t^1}{c_t^2} \\
        \alpha\equiv\frac{\alpha_2}{\alpha_1} \Rightarrow c_t^1 = c_t^2 \alpha^{\sigma} \\
        \alpha^{\sigma}c_t^2+c_t^2 = 2\gamma \\
        c_t^2 = \frac{2\gamma}{1+\alpha^{\sigma}} \\
        c_t^1 = \frac{2\gamma}{1+\alpha^{-\sigma}} \\
      \end{gather*}
      Solving for $\mu$
      \begin{gather*}
        \mu_1 = 2(\alpha\beta[\frac{2\gamma}{\alpha^{-\sigma}-1}]^{-\sigma}) \\
        \mu_2 = 2(\alpha\beta[\frac{2\gamma}{\alpha^{\sigma}-1}]^{-\sigma}) \\
      \end{gather*}
\section*{Question 2}
  \subsection*{Problem}
    Consider the economy in question 1. Assume $\sigma = 1$ (so log utility) and $\gamma = 1$. Assume the following endowments
    \begin{gather*}
      e^1_t = (2,0,2,0,2,\ldots)\\
      e^2_t = (2,2,2,2,2,\ldots)
    \end{gather*}

  \subsection*{Parts}
    \subsubsection*{(a)}
      Find equilibrium allocations and prices. Are prices different than the one we derived during lecture? Why?

    \subsubsection*{(b)}
      Define a sequential market competitive equilibrium. Find interest rate in this equilibrium.

  \subsection*{Solutions}
    \subsubsection*{(a)}
      \begin{gather*}
        {\max\atop{c_t^i}_{t=0}^{\infty}}\sum\limits_{t=0}^{\infty}\beta^t\text{ln}(c_t^i)
      \end{gather*}
      s.t.
      \begin{gather*}
        \sum\limits_{t=0}^{\infty}\hat{p}_tc_t^i\leq\sum\limits_{t=0}^{\infty}\hat{p}_te_t^i \\
        \hat{c}_t^i\geq0 \\
        0<\beta<1 \\
      \end{gather*}
      Markets Clear:
      \begin{gather*}
        e_t^1+e_t^2 = c_t^1+c_t^2 = \begin{cases}
          4, \text{if $t$ is odd} \\
          2, \text{if $t$ is even} \\
        \end{cases}
      \end{gather*}
\section*{Question 3}
  \subsection*{Problem}
    Consider the economy in question 2. Assume the following endowments
    \begin{gather*}
      e^1_t = (2,1,2,1,2,\ldots)\\
      e^2_t = (2,1,2,1,2,\ldots)
    \end{gather*}

    Find equilibrium allocations and prices. Does any trade happen in this equilibrium? Why?

\section*{Question 4}
\subsection*{Problem}
Consider an economy with two types of household with the following preferences
\begin{gather*}
  \sum_{t=0}^{\infty} \beta_i^t \log(c^i_t) \text{ for } i = 1,2
\end{gather*}

The households are different in their discount factor $0 < \beta_1 < \beta_2 < 1$. Both households have endowment of $e_t = 1$ every period.

\subsection*{Parts}
\subsubsection*{(a)}
Define (Arrow-Debreu) competitive equilibrium. And derive household's Euler equation.

\subsubsection*{(b)}
Write down a social planner problem for some Pareto weights and solve for Pareto efficient allocations (treat Pareto weights as parameters).

\subsubsection*{(c)}
Use Negishi Method to solve for competitive equilibrium allocations and prices. (hint: you can find weights that correspond to equilibrium by examining feasibility at $t = 0$ (or $t = \infty$))

\subsubsection*{(d)}
How does $\frac{c^1_t}{c^2_t}$ move over time? What it its limit?

\subsubsection*{(e)}
Explain what is happening here.

\section*{Question 5 (Bonus)}
\subsection*{Problem}
Consider a two period endowment economy with a single consumption good, $c$. The economy is comprised of 2 types of households with identical preferences over consumption in periods $t = 0,1$
\begin{gather*}
  u(c^k_0) + u(c^k_1), k = 1,2
\end{gather*}

Fraction $\eta$ of households are of type 1 and have endowments $(e^1_0, e^2_1) = (1,0)$. The rest of the households are of type 2 and have endowments $(e^2_1, e^2_1) = (0,1)$. There is no production technology and consumption good is non-storable.

\subsection*{Parts}
\subsubsection*{(a)}
Let $a_k$ be one period arrow security in this economy $(k = 1,2)$ with associate interest rate $i$. Households have no initial assets. There are no other assets in the economy. Define sequential market competitive equilibrium.

\subsubsection*{(b)}
Assume $u(c) = \log c$. Find interest rate $i$ as function of $\eta$. What happens to interest rate as $\eta$ increases?

\subsubsection*{(c)}
Discuss your answer to part (b). What is the intuition?

\end{document}
