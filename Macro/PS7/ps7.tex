%% ECON 8040 - Macro: Problem Set 7 %%
\documentclass[10pt, a4paper]{article}
\usepackage[top=3cm, bottom=4cm, left=3.5cm, right=3.5cm]{geometry}
\usepackage{amsmath,amsthm,amsfonts,amssymb,amscd, fancyhdr, color, comment, graphicx, environ}
\usepackage{float}
\usepackage{mathtools}
\usepackage{mathrsfs}
\usepackage[math-style=ISO]{unicode-math}
\DeclareSymbolFont{\mathnormal}{letters}
\usepackage{lastpage}

%%%%%%%%%%%%%%%%%%%%%%%%%%%%%%%%%%%%%%%%%%%%%%%%%%%%%%%%%%%%%%%%%%
%%%%%%%%%%%%%%%%%%%%%%%%%%%%%%%%%%%%%%%%%%%%%%%%%%%%%%%%%%%%%%%%%%
%Fill in the appropriate information below
\newcommand{\norm}[1]{\left\lVert#1\right\rVert}     
\newcommand\course{ECON - 8040}                            
\newcommand\hwnumber{7}                                 
\newcommand\Information{Tate Mason}                        
%%%%%%%%%%%%%%%%%%%%%%%%%%%%%%%%%%%%%%%%%%%%%%%%%%%%%%%%%%%%%%%%%%
%%%%%%%%%%%%%%%%%%%%%%%%%%%%%%%%%%%%%%%%%%%%%%%%%%%%%%%%%%%%%%%%%%
%Page setup
\pagestyle{fancy}
\headheight 35pt
\lhead{\today}
\rhead{}
\lfoot{}
\pagenumbering{arabic}
\cfoot{\small\thepage}
\rfoot{}
\headsep 1.2em
\renewcommand{\baselinestretch}{1.25}

%%%%%%%%%%%%%%%%%%%%%%%%%%%%%%%%%%%%%%%%%%%%%%%%%%%%%%%%%%%%%%%%%%
%%%%%%%%%%%%%%%%%%%%%%%%%%%%%%%%%%%%%%%%%%%%%%%%%%%%%%%%%%%%%%%%%%
%Add new commands here
\renewcommand{\labelenumi}{\arabic{enumi}.}
\newcommand{\Z}{\mathbb Z}
\newcommand{\R}{\mathbb R}
\newcommand{\Q}{\mathbb Q}
\newcommand{\NN}{\mathbb N}
\newcommand{\PP}{\mathbb P}
\newcommand{\sumt}{$\sum\limits_{t=0}^{\infty}$}
\DeclareMathOperator{\Mod}{Mod}
\newtheorem*{theorem}{Theorem}
\newtheorem*{lemma}{Lemma}
\newcommand{\assign}{:=}

\begin{document}
  \begin{titlepage}
    \begin{center}
      \vspace*{3cm}
            
      \vspace{1cm}
      \huge
      Homework \hwnumber
            
      \vspace{1.5cm}
      \Large
            
      \textbf{\Information}
            
      \vfill
      
      Collaboration to varying degrees with Timothy Duhon, Josephine Hughes, Abdul Khan, Kasra Lak, Rachel Lobo, Mingzhou Wang, Wenyi Wang
      \vspace{1cm}
      
      Due on Friday, November 22, by 11:59pm

      \vspace{1cm}

      An \course \ Homework Assignment
            
      \vspace{1cm}
      \Large
      
      \today
            
    \end{center}
  \end{titlepage}
\section*{Question 1}
  \subsection*{Problem}
    Consider the following infinite horizon production economy with a household sector and a business sector:

    \textbf{Business Sector:} Firms in the economy produce a composite good that can be used for either consumption or investment purposes according to the following technology:
    \begin{equation*}
    Y_t = K_{Mt}^\alpha N_{Mt}^{1-\alpha}
    \end{equation*}
    where $K_{Mt}$ is the amount of capital rented by the firm at date t and $N_{Mt}$ is the amount of labor hired by the firm at date t.

    \textbf{Household Sector:} There is a continuum of measure 1 of infinitely lived households.

    \textbf{Preferences:} Preferences are given by
    \begin{equation*}
    \sum_{t=0}^{\infty} \beta^t \log(c_t)
    \end{equation*}
    where the variable $c_t$ is an aggregator of the good produced by the business sector and a good produced by the household. More specifically:
    \begin{equation*}
    c_t = [\mu c_{Mt}^\rho + (1-\mu)c_{Ht}^\rho]^{\frac{1}{\rho}}
    \end{equation*}
    where $c_{Mt}$ is the good produced in the business sector and $c_{Ht}$ is the good produced at home.

    \textbf{Home Production:} Each household has access to the same technology to produce the home good. The use of this technology by a particular household requires that household's own capital and labor. This technology is:
    \begin{equation*}
    c_{Ht} = k_{Ht}^\alpha n_{Ht}^{1-\alpha}
    \end{equation*}

    \textbf{Endowments:} Each household is endowed with one unit of time. Additionally, each household is endowed with $k_{M0}$ units of capital it can rent out to firms in the economy and $k_{H0}$ units of capital that it can use to produce the home good. The two capital stocks depreciate at their respective rates $\delta_K$ and $\delta_H$. Capital is sector specific so home capital cannot be used in the business sector and vice versa.

  \subsection*{Questions}
    \begin{enumerate}
    \item Write down the Social Planner's problem.
    \item Write down the Social Planner's problem in recursive form (Bellman equation) – what are the state variables?
    \item Write down FOCs and envelope conditions for this Bellman equation.
    \item Write down equations that characterize the steady state.
    \end{enumerate}
  \subsection*{Solutions}
    \subsubsection*{(a)}
      \begin{gather*}
        {\max\atop{c_t, n_t, N_t, Y_t, k_{Ht}, K_{Mt}}} \ \sumt\beta^t\log(c_t)\\
        \left s.t. \\
        c_t = [\mu c_{Mt}^{\rho} + (1-\mu)c_{Ht}^{\rho}]^{\frac{1}{\rho}}\\
        c_{Ht} = k_{Ht}^{\alpha}n_{Ht}^{1-\alpha} \\
        Y_t = K_{Mt}^{\alpha}N_{Mt}^{1-\alpha} \\
        Y_t = c_{Mt} + x_{Mt} + x_{Ht} \\
        K_{Mt+1} = x_{Mt}-(1-\delta_M)K_{Mt} \\
        k_{Ht+1} = x_{Ht}-(1-\delta_H)k_{Ht} \\
        n_{Ht} + N_{Mt} = 1 \\
        k_{Ht}, K_{Mt}, c_{Mt}, c_{Ht}, n_{Ht}, N_{Mt}, Y_{t} > 0 \\
        k_{H0}, K_{M0} \ given
      \end{gather*}
    \subsubsection*{(b)}
      \begin{gather*}
        v(k_{Ht}, K_{Mt}) = {\max\atop{c_{Mt}, k_{Ht+1}, K_{Mt+1}}} \ \frac{1}{\rho}\log(\mu c_{m}^{\rho} + (1-\mu)[k_{ht}^{\alpha}(1-n_{mt})^{1-\alpha}]^{\rho}) + \beta v(k_{Ht+1},K_{Mt+1}) \\
        \left s.t. \\
        c_t = (\mu c_{m}^{\rho} + (1-\mu)[k_{ht}^{\alpha}(1-N_{Mt})^{1-\alpha}]^{\rho})^{\frac{1}{\rho}}\\
        c_{Ht} = k_{Ht}^{\alpha}(1-N_{Mt})^{1-\alpha} \\
        c_{Mt} + k_{Ht+1} + K_{Mt+1} = K_{MT}^{\alpha}N_{MT}^{1-\alpha} - (1-\delta_H)k_{Ht} - (1-\delta_M)K_{Mt} \\
      \end{gather*}
      In this case, the state variables are $K_{Mt}$ and $k_{Ht}$ as we have some sort of control over labor, consumption, and investment. 
    \subsubsection*{(c)}
      
\end{document}
