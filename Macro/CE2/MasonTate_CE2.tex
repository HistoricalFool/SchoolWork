%%% Computational Exercise 2- Advanced Macro %%%
\documentclass[10pt, a4paper]{article}
\usepackage[top=3cm, bottom=4cm, left=3.5cm, right=3.5cm]{geometry}
\usepackage{amsmath,amsthm,amsfonts,amssymb,amscd, fancyhdr, color, comment, graphicx, environ}
\usepackage{float}
\usepackage{mathtools}
\usepackage{mathrsfs}
\usepackage[math-style=ISO]{unicode-math}
\DeclareSymbolFont{\mathnormal}{letters}
\usepackage{lastpage}

%%%%%%%%%%%%%%%%%%%%%%%%%%%%%%%%%%%%%%%%%%%%%%%%%%%%%%%%%%%%%%%%%%
%%%%%%%%%%%%%%%%%%%%%%%%%%%%%%%%%%%%%%%%%%%%%%%%%%%%%%%%%%%%%%%%%%
%Fill in the appropriate information below
\newcommand{\norm}[1]{\left\lVert#1\right\rVert}     
\newcommand\course{ECON - 8040}                            % <-- course name   
\newcommand\hwnumber{ 3}                                 % <-- homework number
\newcommand\Information{Tate Mason}                        % <-- personal information
%%%%%%%%%%%%%%%%%%%%%%%%%%%%%%%%%%%%%%%%%%%%%%%%%%%%%%%%%%%%%%%%%%
%%%%%%%%%%%%%%%%%%%%%%%%%%%%%%%%%%%%%%%%%%%%%%%%%%%%%%%%%%%%%%%%%%
%Page setup
\pagestyle{fancy}
\headheight 35pt
\lhead{\today}
\rhead{}
\lfoot{}
\pagenumbering{arabic}
\cfoot{\small\thepage}
\rfoot{}
\headsep 1.2em
\renewcommand{\baselinestretch}{1.25}
%%%%%%%%%%%%%%%%%%%%%%%%%%%%%%%%%%%%%%%%%%%%%%%%%%%%%%%%%%%%%%%%%%
%%%%%%%%%%%%%%%%%%%%%%%%%%%%%%%%%%%%%%%%%%%%%%%%%%%%%%%%%%%%%%%%%%
%Add new commands here
\renewcommand{\labelenumi}{\alph{enumi})}
\newcommand{\Z}{\mathbb Z}
\newcommand{\R}{\mathbb R}
\newcommand{\Q}{\mathbb Q}
\newcommand{\NN}{\mathbb N}
\newcommand{\PP}{\mathbb P}
\DeclareMathOperator{\Mod}{Mod} 
\renewcommand\lstlistingname{Algorithm}
\renewcommand\lstlistlistingname{Algorithms}
\def\lstlistingautorefname{Alg.}
\newtheorem*{theorem}{Theorem}
\newtheorem*{lemma}{Lemma}
\newtheorem{case}{Case}
\newcommand{\assign}{:=}
\newcommand{\infixiff}{\text{ iff }}
\newcommand{\nobracket}{}
\newcommand{\backassign}{=:}
\newcommand{\tmmathbf}[1]{\ensuremath{\boldsymbol{#1}}}
\newcommand{\tmop}[1]{\ensuremath{\operatorname{#1}}}
\newcommand{\tmtextbf}[1]{\text{{\bfseries{#1}}}}
\newcommand{\tmtextit}[1]{\text{{\itshape{#1}}}}

\newenvironment{itemizedot}{\begin{itemize} \renewcommand{\labelitemi}{$\bullet$}\renewcommand{\labelitemii}{$\bullet$}\renewcommand{\labelitemiii}{$\bullet$}\renewcommand{\labelitemiv}{$\bullet$}}{\end{itemize}}
\catcode`\<=\active \def<{
\fontencoding{T1}\selectfont\symbol{60}\fontencoding{\encodingdefault}}
\catcode`\>=\active \def>{
\fontencoding{T1}\selectfont\symbol{62}\fontencoding{\encodingdefault}}
\catcode`\<=\active \def<{
\fontencoding{T1}\selectfont\symbol{60}\fontencoding{\encodingdefault}}

%%%%%%%%%%%%%%%%%%%%%%%%%%%%%%%%%%%%%%%%%%%%%%%%%%%%%%%%%%%%%%%%%%
%%%%%%%%%%%%%%%%%%%%%%%%%%%%%%%%%%%%%%%%%%%%%%%%%%%%%%%%%%%%%%%%%%
%Begin now!

\begin{document}
  \begin{titlepage}
    \begin{center}
      \vspace*{3cm}
            
        \vspace{1cm}
        \huge
        Homework \hwnumber
            
        \vspace{1.5cm}
        \Large
            
        \textbf{\Information}                      % <-- author
            
        \vfill
        Collaboration to varying degrees with Timothy Duhon, Josephine Hughes, Abdul Khan, Kasra Lak, Rachel Lobo, Mingzhou Wang, Wenyi Wang
        
        \vspace{1cm}

        An \course \ Homework Assignment
            
        \vspace{1cm}
        \Large

        
        \today
            
    \end{center}
  \end{titlepage}

  \newpage
\section{Question 1}
  \subsection{Problem}
    Consider the following two period planning problem
    \begin{center}
      $w(\bar k_1)={\max\atop{c_t,k_{t+1}\geq0}}\frac{c_1^{1-\sigma}}{1-\sigma}+\beta\frac{c_2^{1-\sigma}}{1-\sigma}$
    \end{center}
    s.t.
    \begin{center}
      $c_1+k_2=k_1^{\alpha}+(1-\delta)k_1$ \\ 
      $c_2=k_2^{\alpha}+(1-\delta)k_2$ \\ 
      $k_1=\bar k_1$ \\ 
    \end{center}
    The first order conditions for this problem is
    \begin{center}
      $c_1^{-\sigma}=\beta c_2^{-\sigma}(1-\delta+\alpha k_2^{\alpha-1})$.
    \end{center}
    Use the following parameters
    \begin{center}
      \begin{tabular}{|c c c c|}
        \hline
        \beta & \sigma & \alpha & \delta \\
        \hline\hline
        0.95 & 2 & 0.4 & 0.1 \\
        \hline
      \end{tabular}
    \end{center}
    Define 
    \begin{center}
      $k_{ss}=(\frac{\frac{1}{\beta}-1+\delta}{\alpha})^{\frac{1}{\alpha-1}}$
    \end{center}

    (a) Assume $\bar k_1=k_{ss}$. Solve allocation of consumption and capital stock $c_1, c_2, k_2$. Note, you need to solve the following system of equations 
    \begin{center}
      $c_1+k_2=k_1^{\alpha}+(1-\delta)k_1$ \\ 
      $c_2=k_2^{\alpha}+(1-\delta)k_2$ \\
      $c_1^{-\sigma}=c_2^{-\sigma}\beta(1-\delta+\alpha k_2^{\alpha-1})$ \\
    \end{center}
q   using the Newton method.

    (b) Now, make the following grid $\mathcal{K}=\{\frac{1}{2} k_{ss}, \frac{3}{4}k_{ss}, k_{ss}, \frac{3}{2}k_{ss}, 2k_{ss}\}$ for $\bar k_0$. Solve allocations $c_1, c_2, k_2$ for all points on the grid. Using your answers, find value of $w(\bar k_1)$ for every point on the grid and plot $w(\bar k_1)$.
  \subsection{Solution}
\section{Question 2}
  \subsection{Problem}
    



\end{document}
