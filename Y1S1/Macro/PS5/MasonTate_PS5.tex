%% ECON 8040 - Macro: Problem Set 5 %%
\documentclass[10pt, a4paper]{article}
\usepackage[top=3cm, bottom=4cm, left=3.5cm, right=3.5cm]{geometry}
\usepackage{amsmath,amsthm,amsfonts,amssymb,amscd, fancyhdr, color, comment, graphicx, environ}
\usepackage{float}
\usepackage{mathtools}
\usepackage{mathrsfs}
\usepackage[math-style=ISO]{unicode-math}
\DeclareSymbolFont{\mathnormal}{letters}
\usepackage{lastpage}

%%%%%%%%%%%%%%%%%%%%%%%%%%%%%%%%%%%%%%%%%%%%%%%%%%%%%%%%%%%%%%%%%%
%%%%%%%%%%%%%%%%%%%%%%%%%%%%%%%%%%%%%%%%%%%%%%%%%%%%%%%%%%%%%%%%%%
%Fill in the appropriate information below
\newcommand{\norm}[1]{\left\lVert#1\right\rVert}     
\newcommand\course{ECON - 8040}                            % <-- course name   
\newcommand\hwnumber{5}                                 % <-- homework number
\newcommand\Information{Tate Mason}                        % <-- personal information
%%%%%%%%%%%%%%%%%%%%%%%%%%%%%%%%%%%%%%%%%%%%%%%%%%%%%%%%%%%%%%%%%%
%%%%%%%%%%%%%%%%%%%%%%%%%%%%%%%%%%%%%%%%%%%%%%%%%%%%%%%%%%%%%%%%%%
%Page setup
\pagestyle{fancy}
\headheight 35pt
\lhead{\today}
\rhead{}
\lfoot{}
\pagenumbering{arabic}
\cfoot{\small\thepage}
\rfoot{}
\headsep 1.2em
\renewcommand{\baselinestretch}{1.25}
%%%%%%%%%%%%%%%%%%%%%%%%%%%%%%%%%%%%%%%%%%%%%%%%%%%%%%%%%%%%%%%%%%
%%%%%%%%%%%%%%%%%%%%%%%%%%%%%%%%%%%%%%%%%%%%%%%%%%%%%%%%%%%%%%%%%%
%Add new commands here
\renewcommand{\labelenumi}{\alph{enumi})}
\newcommand{\Z}{\mathbb Z}
\newcommand{\R}{\mathbb R}
\newcommand{\Q}{\mathbb Q}
\newcommand{\NN}{\mathbb N}
\newcommand{\PP}{\mathbb P}
\DeclareMathOperator{\Mod}{Mod} 
\renewcommand\lstlistingname{Algorithm}
\renewcommand\lstlistlistingname{Algorithms}
\def\lstlistingautorefname{Alg.}
\newtheorem*{theorem}{Theorem}
\newtheorem*{lemma}{Lemma}
\newtheorem{case}{Case}
\newcommand{\assign}{:=}
\newcommand{\infixiff}{\text{ iff }}
\newcommand{\nobracket}{}
\newcommand{\backassign}{=:}
\newcommand{\tmmathbf}[1]{\ensuremath{\boldsymbol{#1}}}
\newcommand{\tmop}[1]{\ensuremath{\operatorname{#1}}}
\newcommand{\tmtextbf}[1]{\text{{\bfseries{#1}}}}
\newcommand{\tmtextit}[1]{\text{{\itshape{#1}}}}

\newenvironment{itemizedot}{\begin{itemize} \renewcommand{\labelitemi}{$\bullet$}\renewcommand{\labelitemii}{$\bullet$}\renewcommand{\labelitemiii}{$\bullet$}\renewcommand{\labelitemiv}{$\bullet$}}{\end{itemize}}
\catcode`\<=\active \def<{
\fontencoding{T1}\selectfont\symbol{60}\fontencoding{\encodingdefault}}
\catcode`\>=\active \def>{
\fontencoding{T1}\selectfont\symbol{62}\fontencoding{\encodingdefault}}
\catcode`\<=\active \def<{
\fontencoding{T1}\selectfont\symbol{60}\fontencoding{\encodingdefault}}

%%%%%%%%%%%%%%%%%%%%%%%%%%%%%%%%%%%%%%%%%%%%%%%%%%%%%%%%%%%%%%%%%%
%%%%%%%%%%%%%%%%%%%%%%%%%%%%%%%%%%%%%%%%%%%%%%%%%%%%%%%%%%%%%%%%%%
%Begin now!

\begin{document}
  \begin{titlepage}
    \begin{center}
      \vspace*{3cm}
            
        \vspace{1cm}
        \huge
        Homework \hwnumber
            
        \vspace{1.5cm}
        \Large
            
        \textbf{\Information}                      % <-- author
            
        \vfill
        

        Collaboration to varying degrees with Timothy Duhon, Josephine Hughes, Abdul Khan, Kasra Lak, Rachel Lobo, Mingzhou Wang, Wenyi Wang
        
        \vspace{1cm}

        An \course \ Homework Assignment
            
        \vspace{1cm}
        \Large

        
        \today
            
    \end{center}
  \end{titlepage}

  \newpage
\section*{Question 1}
  \subsection*{Problem}
    Assume $F(k,n) = Ak^{\alpha}n^{1-\alpha}$, $k_{t+1} = (1-\delta)k_t + x_t$
    \subsubsection*{(a) Write down a recursive planning problem (Bellman equation) and derive first order and envelope conditions for each of these preferences:}
    \begin{itemize}
      \item $u(c,1-n) = \log c - \psi \frac{n^{1+\frac{1}{\epsilon}}}{1+\frac{1}{\epsilon}}$
      \item $u(c,1-n) = \phi \log c + (1-\phi) \log(1-n)$
      \item $u(c,1-n) = \frac{[c^{\phi}(1-n)^{1-\phi}]^{1-\sigma}}{1-\sigma}$
    \end{itemize}
    
    \subsubsection*{(b) For each case find steady state allocations.}
    (hint: find $\frac{k}{n}$ in steady state. Then use feasibility to find $\frac{c}{n}$. Finally, use labor supply choice to solve $n$ or $c$ in terms of the other one – go as far as you can go. If you cannot derive an explicit formula, write down equations that determine the steady state allocations.)

  \subsection*{Solution}
  \subsubsection*{(a)}
  \subsubsubsection{(i)}
    \begin{gather*}
      v(k) = \{\log(Ak^{\alpha}n^{1-\alpha}+(1-\delta)k-k')-\psi\frac{n^{1+\frac{1}{\epsilon}}}{1+\frac{1}{\epsilon}}+\betav(k')\}
    \end{gather*}
    FOC's
    
    k':
    \begin{gather*}
      \frac{1}{Ak^{\alpha}n^{1-\alpha}+(1-\delta)k-k'} = \beta v'(k')
    \end{gather*}
    
    n:
    \begin{gather*}
      \frac{(1-\alpha)Ak^{\alpha}n^{-\alpha}}{Ak^{\alpha}n^{1-\alpha}+(1-\delta)k-k'} = \psi n^{\frac{1}{\epsilon}}
    \end{gather*}

    Envelope Condition:
    \begin{gather*}
      v'(k) = \frac{\alpha Ak^{\alpha-1}n^{1-\alpha}+(1-\delta)}{Ak^{\alpha}n^{1-\alpha}+(1-\delta)k-k'}\\
      v'(k') = \frac{\alpha Ak'^{\alpha-1}n^{1-\alpha}+(1-\delta)}{Ak'^{\alpha}n^{1-\alpha}+(1-\delta)k'-k''}
    \end{gather*}
  \subsubsubsection{(ii)}
    \begin{gather*}
      v(k) = \{\phi\log(Ak^{\alpha}n^{1-\alpha}+(1-\delta)k-k')+(1-\phi)\log(1-n)+\beta v(k')\}
    \end{gather*}
    FOC's

    k':
    \begin{gather*}
      \frac{\phi}{Ak^{\alpha}n^{1-\alpha}+(1-\delta)k-k'}=\beta v(k')
    \end{gather*}
    n:
    \begin{gather*}
      \frac{(1-\alpha)Ak^{\alpha}n^{-\alpha}}{Ak^{\alpha}n^{1-\alpha}+(1-\delta)k-k'} = \frac{(1-\phi)}{(1-n)}
    \end{gather*}

    Envelope Condition:
    \begin{gather*}
      \frac{\alpha\phi Ak'^{\alpha-1}n^{1-\alpha}+(1-\delta)}{Ak'^{\alpha}n^{1-\alpha}+(1-\delta)k'-k''}
    \end{gather*}
    \subsubsubsection{(iii)}
    FOC's
    
    k':
    \begin{gather*}
      \phi((Ak^{\alpha}n^{1-\alpha}+(1-\delta)k-k')^{\phi}(1-n)^{1-\phi})^{-\sigma} (Ak^{\alpha}n^{1-\alpha}+(1-\delta)k-k')^{\phi-1}(1-n)^{1-\phi} = \beta v(k')
    \end{gather*}
    n:
    \begin{gather*}
      \phi(1-\alpha)Ak^{\alpha}n^{-\alpha}(1-n)=(1-\phi)(Ak^{\alpha}n^{1-\alpha}+(1-\delta)k-k')
    \end{gather*}

    Envelope Condition:
    \begin{gather*}
      v'(k') = \phi ((Ak^\alpha n^{1-\alpha} + (1-\delta)k - k')^\phi (1-n)^{1-\phi})^{-\sigma}\\ \times(Ak^\alpha n^{1-\alpha} + (1-\delta)k - k')^{\phi-1} (1-n)^{1-\phi} [\alpha Ak^{\alpha-1}n^{1-\alpha} + (1-\delta)]
    \end{gather*}
  \subsubsection*{(b)}
  \subsubsubsection{(i)}
    \begin{gather*}
      k=k'=k''\\
      1 = \beta(\alpha Ak^{\alpha-1}n^{1-\alpha}+(1-\delta))\\
      \frac{1}{\beta} +\delta - 1 = \alpha Ak^{\alpha-1}n^{1-\alpha}\\
      \boxed{k = n(\frac{1 + \beta(\delta-1)}{\beta\alpha A})^{\frac{1}{\alpha-1}}} \\
      \boxed{\therefore\frac{k}{n}=\frac{1 + \beta(\delta-1)}{\beta\alpha A})^{\frac{1}{\alpha-1}}}
    \end{gather*}
    With this result, we can find $\frac{c}{n}$ using feasibility 
    \begin{gather*}
      \boxed{\frac{c}{n} = A(\frac{k}{n}^{\alpha})-\delta(\frac{k}{n})}
    \end{gather*}
    To find steady state n we can use
    \begin{gather*}
      \boxed{(1-\alpha)A(\frac{k}{n})^{\alpha}=\psi n^{1+\frac{1}{\epsilon}}(A(\frac{k}{n})^{\alpha}-\delta(\frac{k}{n}))}
    \end{gather*}
  \subsubsubsection{(ii)}
    \begin{gather*}
      \frac{\phi}{Ak^{\alpha}n^{1-\alpha}+(1-\delta)k-k} = \frac{\phi\beta}{Ak^{\alpha}n^{1-\alpha}+(1-\delta)k-k'} \cdot A\alphak^{\alpha-1}n^{1-\alpha}+(1-\delta)\\
      1 = \beta A\alpha k^{\alpha-1}n^{1-\alpha}+(1-\delta)\\
      \frac{1}{\beta}-1+\delta = A\alpha (\frac{k}{n})^{\alpha-1} \\
      \boxed{\frac{k}{n} = (\frac{1+\beta(\delta-1)}{A\alpha\beta})^{\frac{1}{\alpha-1}}}
    \end{gather*}
    As in the first equation:
    \begin{gather*}
      \boxed{\frac{c}{n} = A(\frac{k}{n})^{\alpha}-\delta(\frac{k}{n})}
    \end{gather*}
    Steady state n:
    \begin{gather*}
      \frac{\phi(1-\alpha)A}{n}(\frac{k}{n})^{\alpha} = \frac{1-\phi}{1-n}(A(\frac{k}{n})^{\alpha}-\delta(\frac{k}{n}))\\
      \boxed{n = \frac{\phi A(1-\alpha)(\frac{k}{n})^{\alpha}}{(1-\alpha+\delta(\frac{k}{n})(1-\delta))}}
    \end{gather*}
  \subsubsubsection{(iii)}
    \begin{gather*}
      \boxed{\frac{k}{n} = (\frac{1+\beta(\delta-1)}{A\alpha\beta})^{\frac{1}{\alpha-1}}}
    \end{gather*}
    \begin{gather*}
      \boxed{\frac{c}{n} = A(\frac{k}{n})^{\alpha}-\delta(\frac{k}{n})}
    \end{gather*}
    \begin{gather*}
      \phi(1-\alpha)A(\frac{k}{n})^{\alpha}(1-n) = (1-\phi)(Ak^{\alpha}n^{1-\alpha}+(1-\delta)k-k')\\
      \boxed{n = 1-\frac{(1-\phi)(A(\frac{k}{n})^{\alpha}-\delta{k}{n})}{\phi(1-\alpha)A(\frac{k}{n})^{\alpha}}}
    \end{gather*}

\section*{Question 2}
  \subsection*{Problem}
    Consider the following problem faced by Martin, who is selling his BMW. Martin places an ad in the local newspaper with his telephone number asking interested individuals to make him offers he can't refuse. Individuals making offers are drawn independently and identically from a distribution given by the cumulative distribution function $F(p) = \text{Prob}(p_t \leq p)$. Assume $F(0) = 0$ and $F(B) = 1$ for $B < \infty$. Martin receives one offer each day, and must decide whether to accept or reject it. Martin knows he misplaces things so he does not bother to write down the callers' phone numbers. For this reason, he cannot call a person back the next day and tell them he changed his mind. If he accepts the offer, he received a one time payment $p$ and the car is sold. If he rejects, he will have a chance to entertain future offers.

    Martin's wife is an economist and she insists that he sells their BMW to maximize the expected present value of the sale's price. Martin discounts the future using discount factor $0 < \beta < 1$. Assume the car does not depreciate.

    \subsubsection*{(a) Write down a Bellman equation that describes Martin's decision problem after receiving an offer $p$.}

    \subsubsection*{(b) Show that the Bellman equation you wrote down in part (a) is a contraction mapping.}

    \subsubsection*{(c) Characterize Martin's optimal policy function assuming that offers are uniformly distributed over the interval, $[0, B]$.}
    (Remember for a uniformly distributed variable over the interval $[a, b]$, $F(p) = (p-a)/(b-a)$, and the density function $f(p) = 1/(b-a)$).

  \subsection*{Solution}
  \subsubsection*{(a)}
    \begin{gather*}
      v(p) = \max\limits_p\{p, \beta\mathbb{E}[v(p')]\}
    \end{gather*}
  \subsubsection*{(b)}
    Monotonicity:
    \begin{proof}
      $T(v(p)) = \{p, \beta\mathbb{E}[f(p')]\}\geq\{p,\beta\mathbb{E}[u(p')]\}=T(u(p))$
    \end{proof}
    Discounting:
    \begin{proof}
      For any $T(v(p)+a) = \max\limits_p\{p, \beta\mathbb{E}[f(p')]+\beta a\}\leq\{p, \beta\mathbb{E}[v(p')]\}+\beta a=T(v(p)+a)$
    \end{proof}
  \subsubsection*{(c)}
    \begin{gather*}
      p^* = \beta\mathbb{E}[v(p')]\\
      \mathbb{E}[v(p')] = \beta\int\limits_0^{p^*}\mathbb{E}[v(p')]\frac{1}{B}dp + \int\limits_{p^*}^{B}p(\frac{1}{B})dp\\
      \beta\mathbb{E}[v(p')](\frac{p^*}{B}) + (\frac{B}{2}-\frac{p^{*2}}{2B}) \\
      \frac{p^*}{\beta} = p^*(\frac{p^*}{B})+(\frac{B}{2}-\frac{p^{*2}}{2B}) \\
      \beta p^{*2} -2Bp^*+\beta B^2 = 0 \\
      p^* = \frac{2B - \sqrt{-2B^2 - 4(\beta - (\beta B)^2 )}}{2\beta} \\
      p^* = \frac{2B - 2B\sqrt{(1-\beta^2)}}{2\beta}\\
      \boxed{p^* = \frac{B\sqrt{1-(1-\beta^2)}}{\beta}}
    \end{gather*}
\end{document}
