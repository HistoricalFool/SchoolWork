\documentclass[10pt,a4paper]{article}
\usepackage[top=3cm,bottom=4cm,left=3.5cm,right=3.5cm]{geometry}
\usepackage{amsmath,amsthm,amsfonts,amssymb,amscd}
\usepackage{fancyhdr,color,comment,graphicx,environ,float,mathtools,mathrsfs,bbm,listings}
\newcommand{\norm}[1]{\left\lVert#1\right\rVert}

% Custom headers
\pagestyle{fancy}
\lhead{ECON - 8050}
\chead{}
\rhead{Tate Mason}
\lfoot{}
\cfoot{Homework 3}
\rfoot{\thepage}

\begin{document}

\title{Homework 3}
\author{ECON 8050: Macroeconomics II \\ Tate Mason}
\date{}
\maketitle

\section*{Problem 1: Dynamic Programming}
Consider the following model with a disability shock. There are three sources of uncertainty:
\begin{itemize}
    \item Out-of-pocket medical shock evolving according to transition matrix $\Psi(x_t|x_{t-1})$.
    \item Productivity evolving according to $T(z_t|z_{t-1})$.
    \item Disability shock.
\end{itemize}

The timing of events is as follows: At the beginning of the period, an individual with savings $k_t$ learns their productivity $z_t$ and medical shock $x_t$. Then they decide whether to work ($l_t = 0$ or $l_t > 0$). If working, labor income is $w z_t l_t$. Then, they decide about consumption $c_t$ and savings $k_{t+1}$.

At the end of the period, the disability shock is realized with probability $d$. Disabled individuals stay permanently disabled, do not work, receive constant benefits $DI$, and make only consumption/savings decisions. Medical spending for disabled individuals is fully covered by public insurance.

\begin{enumerate}
    \item[(1)] Write down the dynamic programming problem of a non-disabled individual, denoting the value function as $V_t$.
    \item[(2)] Write down the dynamic programming problem of a disabled individual, denoting the value function as $V_t^d$.
    \item[(3)] Modify the problem assuming disabled individuals can recover with probability $f$. Recovered individuals draw new productivity realizations from the invariant distribution.
    \item[(4)] Extend the model to allow non-disabled individuals to falsely claim disability benefits, introducing the value function for falsely disabled $V_t^{fd}$.
\end{enumerate}

\section*{Problem 2: Consumption-Savings Model}
A consumer with infinite life maximizes quadratic utility:
\begin{align*}
    u(c_t) &= -\frac{1}{2} (c_t - \bar{c})^2
\end{align*}
where future utility is discounted at rate $\beta$ and borrowing/savings occur at interest rate $r$ with $\beta(1+r) = 1$.

The consumer’s endowment $y_t$ is i.i.d. with values $y_H$ and $y_L$ occurring with probabilities $p_H$ and $p_L$ respectively. The budget constraint is:
\begin{align*}
    c_t = a_t(1+r) + y_t - a_{t+1}.
\end{align*}

\begin{enumerate}
    \item[(1)] Solve for the consumption and saving functions. Provide intuition on when savings are positive or negative.
    \item[(2)] Introduce a borrowing constraint $a_{t+1} \geq 0$. Solve the consumer’s problem in recursive form numerically using given parameters.
    \item[(3)] Plot policy functions $a_{t+1}$ and $c_t$ as functions of current assets $a_t$ for cases with and without borrowing constraints.
    \item[(4)] Simulate the income process and optimal decision rules over $T=100$ periods. Compare results with and without borrowing constraints.
\end{enumerate}

\section*{Solution 1}
  \subsection*{Part (i)}
    The Bellman for an able bodied person with probability of becoming disabled $d$ is as follows:
    \begin{gather*}
      V_t(k_t,x_t,z_t) = {\max\atop{c_t,l_t,k_{t+1}}}\{u(c_t,l_t) + 
      \beta(1-d)\sum\limits_{x_t}\sum\limits_{z_t}\Psi(x_t|x_{t-1})
      \text{T}(z_t|z_{t-1})V_{t+1}(k_{t+1},x_{t+1},z_{t+1}) \\
      + \beta d\sum\limits_{x_t}\Psi(x_t|x_{t-1})V_{t+1}(k_{t+1},x_{t+1},z_{t+1})\} \\
      \shortintertext{s.t.} \\
      c_t + k_{t+1} + x_t = wz_tl_t + k_t(1+r) \\
    \end{gather*}
  \subsection*{Part (ii)}
    The Bellman for an individual who is disabled and has no probability of recovery can be represented as follows:
    \begin{gather*}
      V_t^d(k_t,x_t) = {\max\atop{c_t,k_{t+1}}}\{u(c_t,0) + \beta\sum
      \limits_{x_t}\Psi(x_t|x_{t-1})V^d_{t+1}(k_{t+1},x_{t+1})\} \\
      \shortintertext{s.t.} \\
      c_t + k_{t+1} = DI + k_t(1+r)
    \end{gather*}
  \subsection*{Part (iii)}
    The Bellman equation for an individual who is disabled but has a probability of recovery is as follows:
    \begin{gather*}
      V_t^{df}(k_t,x_t) = {\max\atop{c_t,k_{t+1}}}\{u(c_t,l_t) +\beta 
      f\sum\limits_{x_t|x_{t-1}}\sum\limits_{z_t|z_{t-1}}\Psi(x_t|x_
      {t-1})\text{T}(z_t|z_{t-1})V^{d}_{t+1}(k_{t+1}, x_{t+1}, z_{t+1}) \\
      + \beta(1-f)\sum\limits_{x_t}\Psi(x_t|x_{t-1})V^d_{t+1}(k_{t+1},x_{t+1})\} \\
      \shortintertext{s.t.} \\
      c_t + k_{t+1} + x_t = DI + k_t(1+r) \\
    \end{gather*}
  \subsection*{Part (iv)}
    Finally, the Bellman for someone who has the option to fake disability is as follows:
    \begin{gather*}
      V_t^{df}(k_t,x_t,z_t) = {\max\atop{c_t,k_{t+1},l_t}}\{u(c_t,l_t) +\beta 
      f\sum\limits_{x_t|x_{t-1}}\sum\limits_{z_t|z_{t-1}}\Psi(x_t|x_
      {t-1})\text{T}(z_t|z_{t-1})V^{d}_{t+1}(k_{t+1}, x_{t+1}, z_{t+1}) \\
      + \beta(1-f)\sum\limits_{x_t}\Psi(x_t|x_{t-1})V^d_{t+1}(k_{t+1},x_{t+1}) \\
      + \beta f \mathbbm{1}_{fake = 1} \sum_{x_t}(x_t|x_{t-1})V_{t+1}^{fd}(k_{t+1},x_{t+1})\} \\
      \shortintertext{s.t.} \\
      c_t + k_{t+1} + x_t = wz_tl_t + DI\mathbbm{1}_{D = 1 or fake = 1} + k_t(1+r)
    \end{gather*}
\section*{Solution 2:} 
  \subsection*{Part (i)}
    \begin{gather*}
      \sum\limits_{t=0}^{\infty}\beta^t[-\frac{1}{2}(c_t -
      \bar{c})^2] \\
      \shortintertext{s.t.} \\
      c_t + a_{t+1} = (1+r)a_t + y_t \\
      \shortintertext{Because the utility is quadratic and $\beta(1+r) = 1$, we know the Euler is as follows:} \\
      c_t = \mathbb{E}_t[c_{t+1}] \\
      \therefore \ u'(c_t) = \mathbb{E}_t[u'(c_{t+1})] \\
      \shortintertext{Plugging in the utility function given above,} \\
      (c_t - \bar{c}) = \mathbb{E}_t[(c_{t+1} - \bar{c})] \\
      \shortintertext{Assuming $\bar{c}$ is constant,} \\
      c_t = \mathbb{E}_t[c_{t+1}] \rightarrow c_t = c_{t+1} = c \\
      \shortintertext{Plugging this back into the budget constraint will allow us to solve for the intertemporal budget constraint} \\
      \sum\limits_{t=0}^{\infty}\frac{c}{(1+r)^t} = \sum\limits_{t=0}^{\infty}\frac{y_t}{(1+r)^t} + a_0 \\
      \shortintertext{As given in the question, $y_t$ is i.i.d and thus we can state the following } \\
      \mathbb{E}_t[y_t] = p_Ly_L + p_Hy_H \\
      \shortintertext{Now, after simplifying terms,}
      c = r(a_0 + \frac{\mathbb{E}_t[y_t]}{r}) \\
      c_t = ra_t + \mathbb{E}_t[y_t] \\
      \shortintertext{Now, using this, we can find the savings function} \\
      a_{t+1} = a_t(1+r) + y_t - c_t \\
      a_{t+1} = a_t(1+r) + y_t - ra_t - \mathbb{E}_t[y_t] \\
      a_{t+1} = a_t + y_t - \mathbb{E}_t[y_t] \\
    \end{gather*}
    Now, we can see that the agent will consume based on their assets and expected income. This is to be expected and standard. However, their savings function will depend on assets and the difference between actual and expected consumption. If $y_t > \mathbb{E}_t[y_t]$, the agent will save, trying to bolster savings to smooth consumption in periods of lower income. When $y_t < \mathbb{E}_t[y_t]$, the agent will borrow, again attempting to smooth consumption due to their lower income. 
  \subsection*{Parts (ii-iv)}

  Outputs from code are as follows:

  VFI converged after 1644 iterations, diff = 1.16e-10

  Unconstrained VFI converged after 2000 iterations, diff = 1.00e+00

  Borrowing constraint was binding in 1.00\% of periods.

  Here are plots for the associated parts (sorry for poor quality i do not know why this is the case):

  \begin{centering}
  \includegraphics*[width=18cm]{VFI.png}

    \includegraphics*[width=18cm]{SIM.png}
  \end{centering}

    \definecolor{dkgreen}{rgb}{0,0.6,0}
    \definecolor{gray}{rgb}{0.5,0.5,0.5}
    \definecolor{mauve}{rgb}{0.58,0,0.82}
    \lstset{frame=tb,
            language=MatLab,
            aboveskip=3mm,
            belowskip=3mm,
            showstringspaces=false,
            columns=flexible,
            basicstyle={\small\ttfamily},
            numbers=none,
            numberstyle=\tiny\color{gray},
            keywordstyle=\color{blue},
            commentstyle=\color{dkgreen},
            stringstyle=\color{mauve},
            breaklines=true,
            breakatwhitespace=true,
            tabsize=3
          }
    \begin{lstlisting}

    % Solving Consumer's Problem %
    clear;
    clc;

    % Definition of Parameters %
    r = 0.02;
    beta = 1/(1+r);
    cbar = 100;
    yL = 0.05;
    yH = 0.5;
    pL = 0.6;
    pH = 0.4;

    % Bringing in Agrid %
    r=0.02;
    yvec=[0.05,0.5];
    min=0.0001;
    amax=2;
    k=20;
    gamma=(1+r)^(1/(k-1));
    step1=(yvec(1)-min)*(gamma-1)/r;
    n=floor(1+log(amax*(gamma-1)/step1+1)/log(gamma));
    f=@(x)step1*(gamma.^(x-1)-1)/(gamma-1);
    agrid=f(1:n)

    % VFI Params %
    tol = 1e-10;
    maxiter = 2000;
    diff = 999.0;
    iter = 0;

    % Init Matrices %
    n = length(agrid);
    v0 = zeros(n,2);
    v1 = zeros(n,2);
    aopt = zeros(n,2);
    copt = zeros(n,2);
    jopt = zeros(n,2);

    % VFI %
    while diff > tol && iter<maxiter
      iter = iter + 1;
      diff = 0;
      for i = 1:n
        ai = agrid(i)
        for m = 1:2
          if m == 1
            ym = yL;
          else 
            ym = yH;
          end
          res = (1+r)*ai + ym;
          max_val = -Inf;
          opt_index = 1
          for j = 1:n
            a_next = agrid(j);
            if a_next > res
              break;
            end
            c = res - a_next;
            u = -0.5*(c-cbar)^2;
            EV = beta*(pL*v0(j,1) + pH*v0(j,2));
            val = u+EV;
            if val > max_val
              max_val = val;
              opt_index = j;
            end
          end
        v1(i,m) = max_val;
        jopt(i,m) = opt_index;
        aopt(i,m) = agrid(opt_index);
        copt(i,m) = res-aopt(i,m);
        end
      end
      diff = max(abs(v1-v0));
      v0 = v1;
    end
    fprint('VFI converged after %d iterations, diff = %2.2e\n', count, diff);

    % Unconstrained

    % Definition of unconstrained grid
    agrid_UC = -2:01:10;
    nUC = length(agrid_free);

    % initialize params
    vUC0 = zeros(nU,2);
    vUC1 = zeros(nU,2);
    aopt_UC = zeros(nU,2);
    copt_UC = zeros(nU,2);
    jopt_UC = zeros(nU,2);

    % VFI init
    tol = 1e-10;
    maxiter = 2000;
    diff = 1;
    iter = 0;

    while diff > tol && iter<maxiter
      iter = iter+1;
      diffv = 0;
      for i = 1:nUC
        ai = agrid_UC(i);
        for m = 1:2
          if m == 1
            y_m = yL;
          else
            y_m = yH;
          end
          res = (1+r)*ai + ym;
          max_val = -Inf;
          opt_index = 1
          for j = 1:n
            a_next = agrid_UC(j);
            if a_next > res
              break;
            end
            cUC = res - a_next;
            u = -0.5*(cUC-cbar)^2;
            EV = beta*(pL*v0(j,1) + pH*v0(j,2));
            val = u+EV;
            if val > max_val
              max_val = val;
              opt_index = j;
            end
          end
        vUC1(i,m) = max_val;
        jopt_UC(i,m) = opt_index;
        aopt_UC(i,m) = agrid_UC(opt_index);
        copt_UC(i,m) = res-aopt_UC(i,m);
        end
      end
      diffUC = max(abs(vUC1-vUC0));
      vUC0 = vUC1;
    end
    fprint('Unconstrained VFI converged after %d iterations, diff = %2.2e\n', count, diff);

    figure;
    subplot(2,2,1);
    plot(agrid_UC, aopt_UC(:,1), 'b-', agrid, aopt(:,1), 'k--');
    title('Assets Policy Function (yL)');
    xlabel('Current Assets');
    ylabel('Next Period Assets');

    subplot(2,2,2);
    plot(agrid_UC, copt_UC(:,1), 'b-', agrid, copt(:,1), 'k--');
    title('Consumption Policy Function (yL)');
    xlabel('Current Assets');
    ylabel('Consumption');

    subplot(2,2,3);
    plot(agrid_UC, aopt_UC(:,2), 'b-', agrid, aopt(:,1), 'k--');
    title('Assets Policy Function (yH)');
    xlabel('Current Assets');
    ylabel('Next Period Assets');

    subplot(2,2,4);
    plot(agrid_UC, copt_UC(:,2), 'r-', agrid, copt(:,2), 'k--');
    title('Consumption Policy Function (yH)');
    xlabel('Current Assets');
    ylabel('Consumption');

    % Simulations %

    %% Unconstrained
    T = 100;
    a_sim = zeros(T,1);
    c_sim = zeros(T,1);

    %% Constrained
    a_con = zeros(T+1, 1);
    c_con = zeros(T,1)

    a_sim(1) = 0

    rng(1);

    for t = 1:T
      if rand < pL
        y_sim(t) = yL;
        m = 1;
      else
        y_sim(t) = yH;
        m = 2;
      end
      %% UC
      [~, iU] = min(abs(agrid-a_sim(t)));
      a_sim(t+1) = aopt_UC(idx, m);
      c_sim(t) = copt_UC(idx, m);
      %% C
      a_sim(t+1)=interp1(agrid,aopt(:,(y_sim(t)==yH)+1), a_sim(t), 'linear','extrap');
      c_sim(t)=interp1(agrid, copt(:,(y_sim(t)==yH)+1), a_sim(t), 'linear', 'extrap');
    end

    figure('Name', 'Simulation', 'NumberTitle', 'off');
    subplot(2,1,1);
    plot(1:T, a_un(2:end), 'b-o', 1:T, a_con(2:end), 'r--o', 'LineWidth',1);
    xlabel('Time');ylabel('a_{t+1}');
    title('Simulated Assets');
    grid on;

    subplot(2,1,2);
    plot(1:T, c_un, 'b-o', 1:T, c_con, 'r--o','LineWidth',1);
    xlabel('Time');ylabel('c_t');
    title('Simulated Consumption');
    grid on;

    % Count of how often constraint binds
    bind = sum(abs(a_con(2:end))<1e-10);
    pctbind = (100*bind)/T;
    fprintf('Borrowing constraint was binding in %.2f%% of periods.\n', pctbind);
  \end{lstlisting}
\end{document}
