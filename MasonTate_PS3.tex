%%% Problem Set 3 - Advanced Macro %%%
\documentclass[10pt, a4paper]{article}
\usepackage[top=3cm, bottom=4cm, left=3.5cm, right=3.5cm]{geometry}
\usepackage{amsmath,amsthm,amsfonts,amssymb,amscd, fancyhdr, color, comment, graphicx, environ}
\usepackage{float}
\usepackage{mathtools}
\usepackage{mathrsfs}
\usepackage[math-style=ISO]{unicode-math}
\DeclareSymbolFont{\mathnormal}{letters}
\usepackage{lastpage}

%%%%%%%%%%%%%%%%%%%%%%%%%%%%%%%%%%%%%%%%%%%%%%%%%%%%%%%%%%%%%%%%%%
%%%%%%%%%%%%%%%%%%%%%%%%%%%%%%%%%%%%%%%%%%%%%%%%%%%%%%%%%%%%%%%%%%
%Fill in the appropriate information below
\newcommand{\norm}[1]{\left\lVert#1\right\rVert}     
\newcommand\course{ECON - 8040}                            % <-- course name   
\newcommand\hwnumber{ 3}                                 % <-- homework number
\newcommand\Information{Tate Mason}                        % <-- personal information
%%%%%%%%%%%%%%%%%%%%%%%%%%%%%%%%%%%%%%%%%%%%%%%%%%%%%%%%%%%%%%%%%%
%%%%%%%%%%%%%%%%%%%%%%%%%%%%%%%%%%%%%%%%%%%%%%%%%%%%%%%%%%%%%%%%%%
%Page setup
\pagestyle{fancy}
\headheight 35pt
\lhead{\today}
\rhead{}
\lfoot{}
\pagenumbering{arabic}
\cfoot{\small\thepage}
\rfoot{}
\headsep 1.2em
\renewcommand{\baselinestretch}{1.25}
%%%%%%%%%%%%%%%%%%%%%%%%%%%%%%%%%%%%%%%%%%%%%%%%%%%%%%%%%%%%%%%%%%
%%%%%%%%%%%%%%%%%%%%%%%%%%%%%%%%%%%%%%%%%%%%%%%%%%%%%%%%%%%%%%%%%%
%Add new commands here
\renewcommand{\labelenumi}{\alph{enumi})}
\newcommand{\Z}{\mathbb Z}
\newcommand{\R}{\mathbb R}
\newcommand{\Q}{\mathbb Q}
\newcommand{\NN}{\mathbb N}
\newcommand{\PP}{\mathbb P}
\DeclareMathOperator{\Mod}{Mod} 
\renewcommand\lstlistingname{Algorithm}
\renewcommand\lstlistlistingname{Algorithms}
\def\lstlistingautorefname{Alg.}
\newtheorem*{theorem}{Theorem}
\newtheorem*{lemma}{Lemma}
\newtheorem{case}{Case}
\newcommand{\assign}{:=}
\newcommand{\infixiff}{\text{ iff }}
\newcommand{\nobracket}{}
\newcommand{\backassign}{=:}
\newcommand{\tmmathbf}[1]{\ensuremath{\boldsymbol{#1}}}
\newcommand{\tmop}[1]{\ensuremath{\operatorname{#1}}}
\newcommand{\tmtextbf}[1]{\text{{\bfseries{#1}}}}
\newcommand{\tmtextit}[1]{\text{{\itshape{#1}}}}

\newenvironment{itemizedot}{\begin{itemize} \renewcommand{\labelitemi}{$\bullet$}\renewcommand{\labelitemii}{$\bullet$}\renewcommand{\labelitemiii}{$\bullet$}\renewcommand{\labelitemiv}{$\bullet$}}{\end{itemize}}
\catcode`\<=\active \def<{
\fontencoding{T1}\selectfont\symbol{60}\fontencoding{\encodingdefault}}
\catcode`\>=\active \def>{
\fontencoding{T1}\selectfont\symbol{62}\fontencoding{\encodingdefault}}
\catcode`\<=\active \def<{
\fontencoding{T1}\selectfont\symbol{60}\fontencoding{\encodingdefault}}

%%%%%%%%%%%%%%%%%%%%%%%%%%%%%%%%%%%%%%%%%%%%%%%%%%%%%%%%%%%%%%%%%%
%%%%%%%%%%%%%%%%%%%%%%%%%%%%%%%%%%%%%%%%%%%%%%%%%%%%%%%%%%%%%%%%%%
%Begin now!

\begin{document}
  \begin{titlepage}
    \begin{center}
      \vspace*{3cm}
            
        \vspace{1cm}
        \huge
        Homework \hwnumber
            
        \vspace{1.5cm}
        \Large
            
        \textbf{\Information}                      % <-- author
            
        \vfill
        Collaboration to varying degrees with Timothy Duhon, Josephine Hughes, Abdul Khan, Kasra Lak, Rachel Lobo, Mingzhou Wang, Wenyi Wang
        
        \vspace{1cm}

        An \course \ Homework Assignment
            
        \vspace{1cm}
        \Large

        
        \today
            
    \end{center}
  \end{titlepage}

  \newpage
\section{Question 1}
  \subsection{Problem}
    Consider the following finite horizon planning problem
    \begin{center}
      ${\max\atop{\{k_{t+1}_{t=0}^T\}, k_{t+1}\geq0,k_0=\bar{k_0}}}\sum^T_{t=0}\beta^t\log(Ak^{\alpha}_t-k_{t+1})$
    \end{center}

    (a) Solve for the optimal path of $k_t$. First, derive the Euler equation from the first order conditions. Then make the following change of variable
    \begin{center}
      $z_t=\frac{k_{t+1}}{Ak_t^{\alpha}}$
    \end{center}
    Note that $z_T=0$. So you find all $z_t$ by backward induction.

    (b) Find $\lim_{T\rightarrow\infty}z_0$ and $\lim_{T\rightarrow\infty}z_1$. What do you notice?

  \subsection{Solution}
    (a) First, let's find the Euler equation from FOC with respect to $k_{t+1}$

    \begin{center}
      $\log(Ak_t^{\alpha}-k_{t+1})+\beta^t\log(Ak_{t+1}^{\alpha}-k_{t+2})$ \\
      $-\frac{1}{Ak^{\alpha}_t-k_{t+1}}+\beta\frac{1}{Ak^{\alpha}_{t+1}-k_{t+2}}\cdot\alpha Ak_{t+1}^{\alpha-1}$  \\
      $\Aboxed{\frac{1}{Ak_t^{\alpha}-k_{t+1}}=\alpha\beta\frac{1}{Ak_{t+1}^{\alpha}-k_{t+2}}\cdot Ak_{t+1}^{\alpha-1}}$
    \end{center}
    
    Next, we are given the identity $z_t=\frac{k_{t+1}}{Ak_t^{\alpha}}$. We should find all $z_t$ by backwards induction 
    
    To solve, we need to rearrange the Euler equation to get $z_t=\frac{k_{t+1}}{Ak_t^{\alpha}}$
    \begin{center}
      $Ak_{t+1}^{\alpha}-k_{t+2}=\beta(Ak_t^{\alpha}-k_{t+1})(\alpha Ak_{t+1}^{\alpha-1})$ \\
      $1 = \frac{\beta(Ak_t^{\alpha}-k_{t+1})(\alpha Ak_{t+1}^{\alpha-1})}{Ak_{t+1}^{\alpha}}+\frac{k_{t+2}}{Ak_{t+!}^{\alpha}}$ \\
      $1 = \alpha\beta\frac{Ak^{\alpha}_t-k_{t+1}}{k_{t+1}}+\frac{k_{t+2}}{Ak_{t+!}^{\alpha}}$ \\
      $1 = \alpha\beta(\frac{1}{z_t})+z_{t+1} \Rightarrow 1 = \frac{\alpha\beta}{z_t}-\alpha\beta+z_{t+1}$ \\
      $z_t = \frac{\alpha\beta}{1+\alpha\beta-z_{t+1}}$ \\
      $z_T = 0$ \\
      $z_{T-1} = \frac{\alpha\beta}{1+\alpha\beta}$ \\
      $z_{T-2} = \frac{\alpha\beta}{1+\alpha\beta-\frac{\alpha\beta}{1+\alpha\beta}}$ \\
      $ = \frac{\alpha\beta(1+\alpha\beta)}{1+\alpha\beta+(\alpha\beta)^2}$ \\
      $\therefore z_{T-t}=\frac{\alpha\beta(1+\alpha\beta+...+(\alpha\beta)^t)}{1+\alpha\beta+...+(\alpha\beta)^{t+1}}$ \\
      $ = \frac{\alpha\beta(1-(\alpha\beta)^t)}{1-\alpha\beta^{T-t}}$ \\
      $\Aboxed{z_{t+1} = \frac{\alpha\beta(1-(\alpha\beta)^{T-t})}{1-(\alpha\beta)^{T-t+1}}}$ \\ 
    \end{center} 

    (b) Now, finding the limits of $z_0$ and $z_1$ as $T\rightarrow\infty$ should be informative:
    \begin{center}
      ${\lim\atop{T\rightarrow\infty}}z_1 = {\lim\atop{T\rightarrow\infty}}\frac{\alpha\beta(1-(\alpha\beta)^{T-1})}{1-(\alpha\beta)^{T-2}}$ \\
      $ = \frac{\alpha\beta(1-0)}{(1-0)} = \Aboxed{\alpha\beta}$ \\
      ${\lim\atop{T\rightarrow\infty}}z_0 = {\lim\atop{T\rightarrow\infty}} \frac{\alpha\beta(1-(\alpha\beta)^T)}{1-(\alpha\beta)^{T+1}}$ \\
      $ = \frac{\alpha\beta(1-0)}{(1-0)} = \Aboxed{\alpha\beta}$ \\
    \end{center}

    As the $T\rightarrow\infty$, the savings rate, $z_t$, becomes a constant $\alpha\beta$. This makes sense, as in the long run, saving would become constant once steady state is reached. 
\section{Question 2}
  \subsection{Problem}
    Consider a more general version of consumption-saving problem we studied in class. Let $a_t$ be asset position in period $t$. Let $r$ be exogenous net return on asset, and let $e_t$ be exogenous endowment in period $t$. The consumer solves the following decision problem
    \begin{center}
      ${\max\atop{\{c_t,a_{t+1}\}_{t=0}^{\infty}}}\sum_{t=0}^T\beta^t\frac{c_t^{1-\sigma}}{1-\sigma}$
    \end{center}
    s.t.
    \begin{center}
      $c_t+a_{t+1}=(1+r)a_t+e_t$ \\
      $c_t\geq0$ \\
      $a_0$ given
    \end{center}

    where $\sigma>0$ and $0<\beta<1$. Basically, every period the consumer has endowment $e_t$, assets $a_t$, and asset income $r\cdot a_t$. She can split these resources to consumption $c_t$ or future asset position $a_{t+1}$ Ignore the last constraint for now (assume $\bar{A}$ is large enough that it never binds).

    (a) Derive the Euler equation, i.e., the equation that describes intertemporal optimality condition.
    
    \vspace(9mm)
    
    (b) Show that
    \begin{center}
      $\frac{\partial\log(\frac{c_{t+1}}{c_t})}{\partial\log(1+r)}=\frac{1}{\sigma}$
    \end{center}
    $\frac{1}{\sigma}$ is called "inter-temporal elasticity of substitution". Can you explain in one sentence why this label is appropriate?

    (c) Find the optimal sequence of consumption as function of initial asset $a_0$ and present value of endowment (hint: write the "lifetime budget constraint". You only need to solve for $c_0$. The future consumption follows from Euler equation).

  \subsection{Solution}
    (a) Derivation of the Euler equation is as follows:
    \begin{center}
      ${\max\atop{\{c_t,a_{t+1}\}_{t=0}^{\infty}}}\sum_{t=0}^T\beta^t\frac{c_t^{1-\sigma}}{1-\sigma}$ \\
      $\frac{\partial{\max\atop{\{c_t,a_{t+1}\}_{t=0}^{\infty}}}\sum_{t=0}^T\beta^t\frac{c_t^{1-\sigma}}{1-\sigma}}{\partial a_{t+1}}$ \\
      $-((1+r)a_t+e_t-a_{t+1})^{-\sigma}+\beta(1+r)((1+r)a_{t+1}+e_{t+1}-a_{t+2})^{-\sigma}$ \\
      $\Aboxed{((1+r)a_t+e_t-a_{t+1})^{-\sigma}=\beta(1+r)((1+r)a_{t+1}+e_{t+1}-a_{t+2})^{-\sigma}}$
    \end{center}

    (b) To prove 
    \begin{center}
      $\frac{\partial\log(\frac{c_{t+1}}{c_t})}{\partial\log(1+r)}=\frac{1}{\sigma}$
    \end{center}

    we must first define $\frac{c_{t+1}}{c_t}$

    \begin{center}
      $\frac{c_{t+1}}{c_t}=(\beta^t)^{\frac{1}{\sigma}}(1+r)^{\frac{1}{\sigma}}$ \\
    \end{center}

    Next, let's plug this into the desired function
    \begin{center}
      $\frac{\partial\log{\frac{c_{t+1}}{c_t}}}{\partial\log(1+r)}$ \\
      $\frac{\partial\log(\beta^t)^{\frac{1}{\sigma}}(1+r)^{\frac{1}{\sigma}}}{\partial\log(1+r)}$ \\
      $\frac{\partial\frac{1}{\sigma}\log(\beta^t)+\frac{1}{\sigma}\log(1+r)}{\partial\log(1+r)}$  \\
      $\Aboxed{\frac{1}{\sigma}}$
    \end{center}
    $\frac{1}{\sigma}$ is called the intertemporal elasticity of substitution because it relies upon consumption choices as well as the exogenous return on asset across time periods, thus making it intertemporal.

    (c) Now, lets find the lifetime budget constraint by solving for consumption as a function of $a_0$ and $e_0$.

    \begin{center}
      $c_0+a_1=e_0+(1+r)a_0$ \\
      $\frac{c_1}{1+r}+\frac{a_2}{1+r}=\frac{e_1}{1+r}+a_1$ \\
      $\frac{c_2}{(1+r)^2}+\frac{a_3}{(1+r)^2}=\frac{e}{(1+r)^2}+\frac{a_2}{(1+r)}$ \\
    \end{center}
    and so on. But, we can sum across these and cancel terms yielding us
    \begin{center}
      $\Aboxed{\frac{c_t}{(1+r)^t}=\sum^T_{t=0}\frac{e_t}{(1+r)^t}+(1+r)a_0-\frac{a_{t+1}}{(1+r)^t}}$
    \end{center}

\end{document}
